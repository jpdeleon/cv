% The formatting of this CV is based on @davidwhogg's layout.

\documentclass[12pt,letterpaper]{article}

\newcommand{\fullname}{Jerome Pitogo de Leon, Ph.D}
\newcommand{\currentposition}{Postdoctoral researcher at The University of Tokyo}
\newcommand{\address}{3-5-3 Nishi-koiwa Edogawa-ku Tokyo} %133-0057
\newcommand{\phonenumber}{+81 (0) 80 8712 7159}
\newcommand{\email}{jpdeleon@g.ecc.u-tokyo.ac.jp}
\newcommand{\githuburl}{https://jpdeleon.github.io}
\newcommand{\scholarurl}{https://scholar.google.com/citations?hl=en&user=_Z8ialwAAAAJ&view_op=list_works&sortby=pubdate}
\newcommand{\orcid}{0000-0002-6424-3410}
\newcommand{\orcidurl}{https://orcid.org/my-orcid?orcid=0000-0002-6424-3410}
\newcommand{\linkedinurl}{https://www.linkedin.com/in/jpdeleonbsap/}
\newcommand{\paperthree}{https://ui.adsabs.harvard.edu/abs/2023MNRAS.522..750D/abstract}
\newcommand{\papertwo}{https://ui.adsabs.harvard.edu/abs/2021MNRAS.508..195D/abstract}
\newcommand{\paperone}{https://ui.adsabs.harvard.edu/abs/2015ApJ...806L..10D/abstract}
\newcommand{\spieurl}{https://www.spiedigitallibrary.org/conference-proceedings-of-spie/10679/106790Z/Real-time-high-speed-motion-blur-compensation-method-using-galvanometer/10.1117/12.2306621.short}
\newcommand{\naritaurl}{https://www.u-tokyo.ac.jp/focus/en/people/k0001_00784.html}
\newcommand{\tamuraurl}{https://www.u-tokyo.ac.jp/focus/en/people/people003289.htm}
\newcommand{\seseurl}{https://unisec-global.org/rogel.html}
\newcommand{\takamiurl}{https://www.asiaa.sinica.edu.tw/people/cv.php?i=hiro}
\newcommand{\hayakawaurl}{https://ishikawa-vision.org/members/hayakawa/hayakawa-e.html}
\newcommand{\mayamaurl}{https://researchmap.jp/mayama_satoshi?lang=en}
\newcommand{\phdthesisurl}{https://repository.dl.itc.u-tokyo.ac.jp/record/2006678/files/A37944.pdf}
% \newcommand{}{}
\usepackage{color}
\usepackage{fancyhdr}
\usepackage{hyperref}
\usepackage{ifthen}

% \usepackage[yyyymmdd]{datetime}
% \renewcommand{\dateseparator}{-}

% Link formatting.
\definecolor{numcolor}{rgb}{0.5,0.5,0.5}
\definecolor{linkcolor}{rgb}{0,0,0.4}
\hypersetup{%
    colorlinks=true,        % false: boxed links; true: colored links
    linkcolor=linkcolor,    % color of internal links
    citecolor=linkcolor,    % color of links to bibliography
    filecolor=linkcolor,    % color of file links
    urlcolor=linkcolor      % color of external links
}

% Text formatting.
\newcommand{\foreign}[1]{\textit{#1}}
\newcommand{\etal}{\foreign{et~al.}}
\newcommand{\project}[1]{\textsl{#1}}
\definecolor{grey}{rgb}{0.5,0.5,0.5}
\newcommand{\deemph}[1]{\textcolor{grey}{\footnotesize{#1}}}

% literature links--use doi if you can
  \newcommand{\doi}[2]{\emph{\href{http://dx.doi.org/#1}{{#2}}}}
  \newcommand{\ads}[2]{\href{http://adsabs.harvard.edu/abs/#1}{{#2}}}
  \newcommand{\isbn}[1]{{\footnotesize(\textsc{isbn:}{#1})}}
  \newcommand{\arxiv}[1]{{\href{http://arxiv.org/abs/#1}{arXiv:{#1}}}}

% Section headings.
\renewcommand\familydefault{\sfdefault}
\usepackage{titlesec}

\titleformat{\subsection}
{\normalfont\sffamily\large\bfseries}
{}{0pt}{}

\titleformat{\subsubsection}
{\normalfont\sffamily\bfseries}
{}{0pt}{}

\titlespacing{\subsection}{0pt}{2\parskip}{0pt}
\titlespacing{\subsubsection}{0pt}{\parskip}{0pt}

\newcommand{\cvheading}[1]{\addvspace{1ex}\pagebreak[2]\par\textbf{#1}\nopagebreak\vspace{-0.4em}}

% Set up the custom unordered list.
\newcounter{refpubnum}
\newcommand{\cvlist}{%
    \rightmargin=0in
    \leftmargin=0.15in
    \topsep=0ex
    \partopsep=0pt
    \itemsep=0.2ex
    \parsep=0pt
    \itemindent=-1.0\leftmargin
    \listparindent=0.0\leftmargin
    \settowidth{\labelsep}{~}
    \usecounter{refpubnum}
}

% Margins and spaces.
\raggedright
\setlength{\oddsidemargin}{0in}
\setlength{\topmargin}{0in}
\setlength{\headsep}{0.20in}
\setlength{\headheight}{0.25in}
\setlength{\textheight}{9.1in}
\addtolength{\topmargin}{-\headsep}
\addtolength{\topmargin}{-\headheight}
\setlength{\textwidth}{6.50in}
\setlength{\parindent}{0in}
\setlength{\parskip}{1ex}

% Headings and footings.
\renewcommand{\headrulewidth}{0pt}
\pagestyle{fancy}
\lhead{\deemph{Jerome Pitogo de Leon}}
\chead{\deemph{Curriculum Vitae}}
\rhead{\deemph{\thepage}}
\cfoot{\deemph{Last updated: \today}}

% Journal names.
\newcommand{\aj}{AJ}
\newcommand{\apj}{ApJ}
\newcommand{\pasp}{PASP}
\newcommand{\mnras}{MNRAS}

\newcommand{\pubsdate}{2024-12-12}
\newcommand{\pubsfirst}{3}
\newcommand{\pubsnumber}{75}
\newcommand{\pubscitations}{1,468}
\newcommand{\pubshindex}{24}

\begin{document}\thispagestyle{empty}\sloppy\sloppypar\raggedbottom

\textbf{\Large \fullname} \\[0.5ex]
\currentposition \\
\textsf{\small 
    \href{mailto:\email}{jpdeleon [at] g.ecc.u-tokyo.ac.jp} | %
    Phone: {\phonenumber} | %
    % \href{\scholarurl}{google scholar} | %
    \href{\orcidurl}{ORCID: \orcid} | %
    \href{\githuburl}{github.com/jpdeleon} | %
    \href{\scholarurl}{scholar.google.com} | %
    \href{\linkedinurl}{linkedin.com/in/jpdeleonbsap}
}\\[0.5ex]

\subsection{Overview}
\begin{list}{}{\cvlist}
      \item As of \pubsdate\footnote{Publication data collected from \href{https://ui.adsabs.harvard.edu/}{NASA Astrophysics Data System}}, published \pubsnumber\ refereed papers with \pubscitations\ total citations, and h-index of \pubshindex.
      \item Taught 2 introductory courses in Astronomy for undergraduates. 
      \item Led a research group composed of 1 PhD, 3 master and 2 undergraduate students. 
      \item Gave several invited colloquium talks for research and public outreach.
\end{list}

\subsection{Education}
\begin{list}{}{\cvlist}
    \item
          PhD 2021, Department of Astronomy, The University of Tokyo, Japan.\\Advisor: \href{\tamuraurl}{Prof. Motohide Tamura}
    \item
          MSc 2018, Department of Astronomy, The University of Tokyo, Japan.\\Advisor: \href{\tamuraurl}{Prof. Motohide Tamura}
    \item
          BSc 2013, Department of Physics, University of the Philippines, Los Ba\~nos, Laguna, Philippines.\\Advisor: \href{\seseurl}{Dr. Rogel Mari Sese}      
\end{list}

\subsection{Full-time Work Experience}
\begin{list}{}{\cvlist}
    \item Project Research Assistant (2023/4 - current) \\
        Advisor: Asst prof. Norio Narita \\
        Department of Multi-Disciplinary Sciences, Graduate School of Arts and Sciences, The University of Tokyo, 3-8-1 Komaba, Meguro, Tokyo 153-8902, Japan \\
        Theme: Discovery and characterization of young transiting exoplanets with TESS telescope (See \href{\paperthree}{de Leon et al. 2023})\\
    \item Project Research Assistant (2021/4 - 2023/3) \\
        Department of Astronomy, Graduate School of Science, The University of Tokyo, 7-3-1 Hongo, Bunkyo-ku, Tokyo 113-0033, Japan \\
        Advisor: Prof. Motohide Tamura \\
        Theme: Discovery and characterization of transiting exoplanets with Kepler telescope (See \href{\papertwo}{de Leon et al. 2021}) \\
    \item Research Development Assistant (2013/4-2014/6)\\
        Regulus SpaceTech, Inc., Los Ba\~nos, Laguna, Philippines\\
        Theme: Data pipeline development for basic operation and data analysis of automated solar telescope
\end{list}

\subsection{Temporary Research Positions}
\begin{list}{}{\cvlist}
    \item Technical Assistant (2018/10-2019/3)\\
          Engineering Department, The University of Tokyo, Japan\\
          Developed image processing software for high resolution mosaic/stitching and analysis (See \href{\spieurl}{Murakami et al. 2018})
    \item Research Assistant (2017/4-2018/3)\\	
          Astronomy Department, The University of Tokyo, Japan\\
          Developed pipeline for modeling and analysis of multi-wavelength transit light curves used in our research group and mentored new members
    \item Research Student (2015/9-2016/3)\\
          Department of Astronomy, Graduate University for Advanced Studies (aka SOKENDAI), Japan\\
          Image processing and data reduction of high resolution astronomical images and scientific report writing
    \item Intern (2014/7-2014/9)\\
          Academia Sinica Institute for Astronomy and Astrophysics, Taipei, Taiwan\\	
          Developed image processing pipeline for analysis of high resolution astronomical images (See \href{\paperone}{de Leon et al. 2015})
\end{list}

% \subsection{Popular open-source software}
% \begin{list}{}{\cvlist}
%   \item \href{https://github.com/jpdeleon/cv}{{\bf cv}} --- 0 stars / 0 forks \\
Jerome's Curriculum Vitae \href{https://raw.githubusercontent.com/jpdeleon/cv/main-pdf/tex/cv_pubs.pdf}{[docs]}

\item \href{https://github.com/jpdeleon/chronos}{{\bf chronos}} --- 5 stars / 2 forks \\
analysis of Tess data focused on young host stars \href{None}{[docs]}

\item \href{https://github.com/jpdeleon/quicklook}{{\bf quicklook}} --- 8 stars / 0 forks \\
Quick look lightcurve generator \href{None}{[docs]}

\item \href{https://github.com/jpdeleon/cluster-gallery}{{\bf cluster-gallery}} --- 1 stars / 0 forks \\
Check it out at https://jpdeleon.github.io/cluster-gallery/ \href{None}{[docs]}

\item \href{https://github.com/jpdeleon/epoch}{{\bf epoch}} --- 0 stars / 0 forks \\
This code is used to over plot the position of target on archival (survey) images \href{None}{[docs]}

\item \href{https://github.com/jpdeleon/tfop_code}{{\bf tfop{\_}code}} --- 0 stars / 0 forks \\
code for analysing MuSCAT1/2/3/4 photometry data for TFOP \href{None}{[docs]}
% \end{list}

\ifdefined\withpubs
  \subsection{Publications}
  refereed: 76 / first author: 3 / citations: 1,593 / h-index: 25 (as of 2025-04-04)
  % \subsubsection{Refereed publications}
  \begin{list}{}{\cvlist}
    \item[{\color{numcolor}\scriptsize77}] Barkaoui, K.; Korth, J.; Gaidos, E.; Agol, E.; \etal\ (incl.\ \textbf{de Leon, J. P.}), 2025, \doi{10.1051/0004-6361/202452916}{TOI-2015 b: A sub-Neptune in strong gravitational interaction with an outer non-transiting planet}, Astronomy and Astrophysics, \textbf{695} (\arxiv{2502.07074})

\item[{\color{numcolor}\scriptsize76}] Masuda, Kento; Libby-Roberts, Jessica E.; Livingston, John H.; Stevenson, Kevin B.; \etal\ (incl.\ \textbf{de Leon, J. P.}), 2024, \doi{10.3847/1538-3881/ad83d3}{A Fourth Planet in the Kepler-51 System Revealed by Transit Timing Variations}, The Astronomical Journal, \textbf{168}, 294 (\arxiv{2410.01625}) [\href{https://ui.adsabs.harvard.edu/abs/2024AJ....168..294M}{4 citations}]

\item[{\color{numcolor}\scriptsize75}] Ehrhardt, Juliana; Thomas, Luis; Kellermann, Hanna; Freitag, Christine; \etal\ (incl.\ \textbf{de Leon, J. P.}), 2024, \doi{10.1051/0004-6361/202451404}{Confirmation of four hot Jupiters detected by TESS using follow-up spectroscopy from MaHPS at Wendelstein together with NEID and TRES}, Astronomy and Astrophysics, \textbf{692} (\arxiv{2501.04383})

\item[{\color{numcolor}\scriptsize74}] Ghachoui, M.; Rackham, B. V.; D{\'e}vora-Pajares, M.; Chouqar, J.; \etal\ (incl.\ \textbf{de Leon, J. P.}), 2024, \doi{10.1051/0004-6361/202451120}{TESS discovery of two super-Earths orbiting the M-dwarf stars TOI-6002 and TOI-5713 near the radius valley}, Astronomy and Astrophysics, \textbf{690} (\arxiv{2408.00709})

\item[{\color{numcolor}\scriptsize73}] Pel{\'a}ez-Torres, A.; Esparza-Borges, E.; Pall{\'e}, E.; Parviainen, H.; \etal\ (incl.\ \textbf{de Leon, J. P.}), 2024, \doi{10.1051/0004-6361/202347251}{Validation of up to seven TESS planet candidates through multi-colour transit photometry using MuSCAT2 data}, Astronomy and Astrophysics, \textbf{690} (\arxiv{2409.07400})

\item[{\color{numcolor}\scriptsize72}] Hayashi, Yuya; Narita, Norio; Fukui, Akihiko; Changeat, Quentin; \etal\ (incl.\ \textbf{de Leon, J. P.}), 2024, \doi{10.1093/pasj/psae075}{Low abundances of TiO and VO on the dayside of KELT-9 b: Insights from ground-based photometric observations}, Publications of the Astronomical Society of Japan, \textbf{76}, 1131 (\arxiv{2408.15698})

\item[{\color{numcolor}\scriptsize71}] Orell-Miquel, J.; Murgas, F.; Pall{\'e}, E.; Mallorqu{\'\i}n, M.; \etal\ (incl.\ \textbf{de Leon, J. P.}), 2024, \doi{10.1051/0004-6361/202449411}{The MOPYS project: A survey of 70 planets in search of extended He I and H atmospheres: No evidence of enhanced evaporation in young planets}, Astronomy and Astrophysics, \textbf{689} (\arxiv{2404.16732}) [\href{https://ui.adsabs.harvard.edu/abs/2024A&A...689A.179O}{18 citations}]

\item[{\color{numcolor}\scriptsize70}] Gaidos, E.; Parviainen, H.; Esparza-Borges, E.; Fukui, A.; \etal\ (incl.\ \textbf{de Leon, J. P.}), 2024, \doi{10.1051/0004-6361/202451332}{Climate change in hell: Long-term variation in transits of the evaporating planet K2-22b}, Astronomy and Astrophysics, \textbf{688} (\arxiv{2407.17372})

\item[{\color{numcolor}\scriptsize69}] Schulte, Jack; Rodriguez, Joseph E.; Bieryla, Allyson; Quinn, Samuel N.; \etal\ (incl.\ \textbf{de Leon, J. P.}), 2024, \doi{10.3847/1538-3881/ad4a57}{Migration and Evolution of giant ExoPlanets (MEEP). I. Nine Newly Confirmed Hot Jupiters from the TESS Mission}, The Astronomical Journal, \textbf{168}, 32 (\arxiv{2401.05923}) [\href{https://ui.adsabs.harvard.edu/abs/2024AJ....168...32S}{4 citations}]

\item[{\color{numcolor}\scriptsize68}] Gillon, Micha{\"e}l; Pedersen, Peter P.; Rackham, Benjamin V.; Dransfield, Georgina; \etal\ (incl.\ \textbf{de Leon, J. P.}), 2024, \doi{10.1038/s41550-024-02271-2}{Detection of an Earth-sized exoplanet orbiting the nearby ultracool dwarf star SPECULOOS-3}, Nature Astronomy, \textbf{8}, 865 (\arxiv{2406.00794}) [\href{https://ui.adsabs.harvard.edu/abs/2024NatAs...8..865G}{6 citations}]

\item[{\color{numcolor}\scriptsize67}] Barkaoui, K.; Schwarz, R. P.; Narita, N.; Mistry, P.; \etal\ (incl.\ \textbf{de Leon, J. P.}), 2024, \doi{10.1051/0004-6361/202349127}{Three short-period Earth-sized planets around M dwarfs discovered by TESS: TOI-5720 b, TOI-6008 b, and TOI-6086 b}, Astronomy and Astrophysics, \textbf{687} (\arxiv{2405.06350})

\item[{\color{numcolor}\scriptsize66}] Kuzuhara, Masayuki; Fukui, Akihiko; Livingston, John H.; Caballero, Jos{\'e} A.; \etal\ (incl.\ \textbf{de Leon, J. P.}), 2024, \doi{10.3847/2041-8213/ad3642}{Gliese 12 b: A Temperate Earth-sized Planet at 12 pc Ideal for Atmospheric Transmission Spectroscopy}, The Astrophysical Journal, \textbf{967} (\arxiv{2405.14708}) [\href{https://ui.adsabs.harvard.edu/abs/2024ApJ...967L..21K}{5 citations}]

\item[{\color{numcolor}\scriptsize65}] Hori, Yasunori; Fukui, Akihiko; Hirano, Teruyuki; Narita, Norio; \etal\ (incl.\ \textbf{de Leon, J. P.}), 2024, \doi{10.3847/1538-3881/ad4115}{The Discovery and Follow-up of Four Transiting Short-period Sub-Neptunes Orbiting M Dwarfs}, The Astronomical Journal, \textbf{167}, 289 (\arxiv{2405.12637})

\item[{\color{numcolor}\scriptsize64}] Hord, Benjamin J.; Kempton, Eliza M. -R.; Evans-Soma, Thomas M.; Latham, David W.; \etal\ (incl.\ \textbf{de Leon, J. P.}), 2024, \doi{10.3847/1538-3881/ad3068}{Identification of the Top TESS Objects of Interest for Atmospheric Characterization of Transiting Exoplanets with JWST}, The Astronomical Journal, \textbf{167}, 233 (\arxiv{2308.09617}) [\href{https://ui.adsabs.harvard.edu/abs/2024AJ....167..233H}{17 citations}]

\item[{\color{numcolor}\scriptsize63}] Mori, Mayuko; Ikuta, Kai; Fukui, Akihiko; Narita, Norio; \etal\ (incl.\ \textbf{de Leon, J. P.}), 2024, \doi{10.1093/mnras/stae841}{Characterization of starspots on a young M-dwarf K2-25: multiband observations of stellar photometric variability and planetary transits}, Monthly Notices of the Royal Astronomical Society, \textbf{530}, 167 (\arxiv{2403.13946}) [\href{https://ui.adsabs.harvard.edu/abs/2024MNRAS.530..167M}{4 citations}]

\item[{\color{numcolor}\scriptsize62}] Murgas, F.; Pall{\'e}, E.; Orell-Miquel, J.; Carleo, I.; \etal\ (incl.\ \textbf{de Leon, J. P.}), 2024, \doi{10.1051/0004-6361/202348813}{Wolf 327b: A new member of the pack of ultra-short-period super-Earths around M dwarfs}, Astronomy and Astrophysics, \textbf{684} (\arxiv{2401.12150}) [\href{https://ui.adsabs.harvard.edu/abs/2024A&A...684A..83M}{5 citations}]

\item[{\color{numcolor}\scriptsize61}] Serrano Bell, J.; D{\'\i}az, R. F.; H{\'e}brard, G.; Martioli, E.; \etal\ (incl.\ \textbf{de Leon, J. P.}), 2024, \doi{10.1051/0004-6361/202348288}{TOI-1199 b and TOI-1273 b: Two new transiting hot Saturns detected and characterized with SOPHIE and TESS}, Astronomy and Astrophysics, \textbf{684} (\arxiv{2402.07861}) [\href{https://ui.adsabs.harvard.edu/abs/2024A&A...684A...6S}{2 citations}]

\item[{\color{numcolor}\scriptsize60}] Parviainen, H.; Murgas, F.; Esparza-Borges, E.; Pel{\'a}ez-Torres, A.; \etal\ (incl.\ \textbf{de Leon, J. P.}), 2024, \doi{10.1051/0004-6361/202347431}{TOI-2266 b: A keystone super-Earth at the edge of the M dwarf radius valley}, Astronomy and Astrophysics, \textbf{683} (\arxiv{2401.11879}) [\href{https://ui.adsabs.harvard.edu/abs/2024A&A...683A.170P}{2 citations}]

\item[{\color{numcolor}\scriptsize59}] Kang, H.; Chen, G.; Pall{\'e}, E.; Murgas, F.; \etal\ (incl.\ \textbf{de Leon, J. P.}), 2024, \doi{10.1093/mnras/stae072}{Simultaneous multicolour transit photometry of hot Jupiters HAT-P-19b, HAT-P-51b, HAT-P-55b, and HAT-P-65b}, Monthly Notices of the Royal Astronomical Society, \textbf{528}, 1930 (\arxiv{2401.03715}) [\href{https://ui.adsabs.harvard.edu/abs/2024MNRAS.528.1930K}{2 citations}]

\item[{\color{numcolor}\scriptsize58}] Mallorqu{\'\i}n, M.; Goffo, E.; Pall{\'e}, E.; Lodieu, N.; \etal\ (incl.\ \textbf{de Leon, J. P.}), 2023, \doi{10.1051/0004-6361/202347346}{TOI-1801 b: A temperate mini-Neptune around a young M0.5 dwarf}, Astronomy and Astrophysics, \textbf{680} (\arxiv{2310.10244}) [\href{https://ui.adsabs.harvard.edu/abs/2023A&A...680A..76M}{7 citations}]

\item[{\color{numcolor}\scriptsize57}] Luque, R.; Osborn, H. P.; Leleu, A.; Pall{\'e}, E.; \etal\ (incl.\ \textbf{de Leon, J. P.}), 2023, \doi{10.1038/s41586-023-06692-3}{A resonant sextuplet of sub-Neptunes transiting the bright star HD 110067}, Nature, \textbf{623}, 932 (\arxiv{2311.17775}) [\href{https://ui.adsabs.harvard.edu/abs/2023Natur.623..932L}{29 citations}]

\item[{\color{numcolor}\scriptsize56}] Gan, Tianjun; Cadieux, Charles; Jahandar, Farbod; Vazan, Allona; \etal\ (incl.\ \textbf{de Leon, J. P.}), 2023, \doi{10.3847/1538-3881/acf56d}{A Massive Hot Jupiter Orbiting a Metal-rich Early M Star Discovered in the TESS Full-frame Images}, The Astronomical Journal, \textbf{166}, 165 (\arxiv{2307.07329}) [\href{https://ui.adsabs.harvard.edu/abs/2023AJ....166..165G}{11 citations}]

\item[{\color{numcolor}\scriptsize55}] Hartman, J. D.; Bakos, G. {\'A}.; Csubry, Z.; Howard, A. W.; \etal\ (incl.\ \textbf{de Leon, J. P.}), 2023, \doi{10.3847/1538-3881/acf56e}{TOI 4201 b and TOI 5344 b: Discovery of Two Transiting Giant Planets around M-dwarf Stars and Revised Parameters for Three Others}, The Astronomical Journal, \textbf{166}, 163 (\arxiv{2307.06809}) [\href{https://ui.adsabs.harvard.edu/abs/2023AJ....166..163H}{14 citations}]

\item[{\color{numcolor}\scriptsize54}] Palle, E.; Orell-Miquel, J.; Brady, M.; Bean, J.; \etal\ (incl.\ \textbf{de Leon, J. P.}), 2023, \doi{10.1051/0004-6361/202244261}{GJ 806 (TOI-4481): A bright nearby multi-planetary system with a transiting hot low-density super-Earth}, Astronomy and Astrophysics, \textbf{678} (\arxiv{2301.06873}) [\href{https://ui.adsabs.harvard.edu/abs/2023A&A...678A..80P}{12 citations}]

\item[{\color{numcolor}\scriptsize53}] Orell-Miquel, J.; Lamp{\'o}n, M.; L{\'o}pez-Puertas, M.; Mallorqu{\'\i}n, M.; \etal\ (incl.\ \textbf{de Leon, J. P.}), 2023, \doi{10.1051/0004-6361/202346445}{Confirmation of an He I evaporating atmosphere around the 650-Myr-old sub-Neptune HD 235088 b (TOI-1430 b) with CARMENES}, Astronomy and Astrophysics, \textbf{677} (\arxiv{2307.05191}) [\href{https://ui.adsabs.harvard.edu/abs/2023A&A...677A..56O}{16 citations}]

\item[{\color{numcolor}\scriptsize52}] Kagetani, Taiki; Narita, Norio; Kimura, Tadahiro; Hirano, Teruyuki; \etal\ (incl.\ \textbf{de Leon, J. P.}), 2023, \doi{10.1093/pasj/psad031}{The mass of TOI-519 b: A close-in giant planet transiting a metal-rich mid-M dwarf}, Publications of the Astronomical Society of Japan, \textbf{75}, 713 (\arxiv{2304.14703}) [\href{https://ui.adsabs.harvard.edu/abs/2023PASJ...75..713K}{14 citations}]

\item[{\color{numcolor}\scriptsize51}] \textbf{de Leon, J. P.}; Livingston, J. H.; Jenkins, J. S.; Vines, J. I.; \etal, 2023, \doi{10.1093/mnras/stad894}{A sub-Neptune transiting the young field star HD 18599 at 40 pc}, Monthly Notices of the Royal Astronomical Society, \textbf{522}, 750 (\arxiv{2210.08179}) [\href{https://ui.adsabs.harvard.edu/abs/2023MNRAS.522..750D}{4 citations}]

\item[{\color{numcolor}\scriptsize50}] Morello, G.; Parviainen, H.; Murgas, F.; Pall{\'e}, E.; \etal\ (incl.\ \textbf{de Leon, J. P.}), 2023, \doi{10.1051/0004-6361/202243592}{TOI-1442 b and TOI-2445 b: Two potentially rocky ultra-short period planets around M dwarfs}, Astronomy and Astrophysics, \textbf{673} (\arxiv{2201.13274}) [\href{https://ui.adsabs.harvard.edu/abs/2023A&A...673A..32M}{9 citations}]

\item[{\color{numcolor}\scriptsize49}] Vines, Jose I.; Jenkins, James S.; Berdi{\~n}as, Zaira; Soto, Maritza G.; \etal\ (incl.\ \textbf{de Leon, J. P.}), 2023, \doi{10.1093/mnras/stac2845}{A dense mini-Neptune orbiting the bright young star HD 18599}, Monthly Notices of the Royal Astronomical Society, \textbf{518}, 2627 (\arxiv{2210.07945}) [\href{https://ui.adsabs.harvard.edu/abs/2023MNRAS.518.2627V}{8 citations}]

\item[{\color{numcolor}\scriptsize48}] Kawauchi, K.; Murgas, F.; Palle, E.; Narita, N.; \etal\ (incl.\ \textbf{de Leon, J. P.}), 2022, \doi{10.1051/0004-6361/202243381}{Validation and atmospheric exploration of the sub-Neptune TOI-2136b around a nearby M3 dwarf}, Astronomy and Astrophysics, \textbf{666} (\arxiv{2202.10182}) [\href{https://ui.adsabs.harvard.edu/abs/2022A&A...666A...4K}{7 citations}]

\item[{\color{numcolor}\scriptsize47}] Carmichael, Theron W.; Irwin, Jonathan M.; Murgas, Felipe; Pall{\'e}, Enric; \etal\ (incl.\ \textbf{de Leon, J. P.}), 2022, \doi{10.1093/mnras/stac1666}{TOI-2119: a transiting brown dwarf orbiting an active M-dwarf from NASA's TESS mission}, Monthly Notices of the Royal Astronomical Society, \textbf{514}, 4944 (\arxiv{2202.08842}) [\href{https://ui.adsabs.harvard.edu/abs/2022MNRAS.514.4944C}{12 citations}]

\item[{\color{numcolor}\scriptsize46}] Mori, Mayuko; Livingston, John H.; \textbf{de Leon, J. P.}; Narita, Norio; \etal, 2022, \doi{10.3847/1538-3881/ac6bf8}{TOI-1696: A Nearby M4 Dwarf with a 3 R {\_}{\ensuremath{\oplus}}{\_} Planet in the Neptunian Desert}, The Astronomical Journal, \textbf{163}, 298 (\arxiv{2203.02694}) [\href{https://ui.adsabs.harvard.edu/abs/2022AJ....163..298M}{12 citations}]

\item[{\color{numcolor}\scriptsize45}] Christian, Sam; Vanderburg, Andrew; Becker, Juliette; Yahalomi, Daniel A.; \etal\ (incl.\ \textbf{de Leon, J. P.}), 2022, \doi{10.3847/1538-3881/ac517f}{A Possible Alignment Between the Orbits of Planetary Systems and their Visual Binary Companions}, The Astronomical Journal, \textbf{163}, 207 (\arxiv{2202.00042}) [\href{https://ui.adsabs.harvard.edu/abs/2022AJ....163..207C}{33 citations}]

\item[{\color{numcolor}\scriptsize44}] Watanabe, Noriharu; Narita, Norio; Palle, Enric; Fukui, Akihiko; \etal\ (incl.\ \textbf{de Leon, J. P.}), 2022, \doi{10.1093/mnras/stac620}{Nodal precession of WASP-33b for 11 yr by Doppler tomographic and transit photometric observations}, Monthly Notices of the Royal Astronomical Society, \textbf{512}, 4404 (\arxiv{2203.02003}) [\href{https://ui.adsabs.harvard.edu/abs/2022MNRAS.512.4404W}{11 citations}]

\item[{\color{numcolor}\scriptsize43}] Fukui, Akihiko; Kimura, Tadahiro; Hirano, Teruyuki; Narita, Norio; \etal\ (incl.\ \textbf{de Leon, J. P.}), 2022, \doi{10.1093/pasj/psab106}{TOI-2285b: A 1.7 Earth-radius planet near the habitable zone around a nearby M dwarf}, Publications of the Astronomical Society of Japan, \textbf{74} (\arxiv{2110.10215}) [\href{https://ui.adsabs.harvard.edu/abs/2022PASJ...74L...1F}{10 citations}]

\item[{\color{numcolor}\scriptsize42}] Castro-Gonz{\'a}lez, A.; D{\'\i}ez Alonso, E.; Men{\'e}ndez Blanco, J.; Livingston, J.; \etal\ (incl.\ \textbf{de Leon, J. P.}), 2022, \doi{10.1093/mnras/stab2669}{The K2-OjOS Project: New and revisited planets and candidates in K2 campaigns 5, 16, {\&} 18}, Monthly Notices of the Royal Astronomical Society, \textbf{509}, 1075 (\arxiv{2109.03346}) [\href{https://ui.adsabs.harvard.edu/abs/2022MNRAS.509.1075C}{18 citations}]

\item[{\color{numcolor}\scriptsize41}] Wong, Ian; Shporer, Avi; Zhou, George; Kitzmann, Daniel; \etal\ (incl.\ \textbf{de Leon, J. P.}), 2021, \doi{10.3847/1538-3881/ac26bd}{TOI-2109: An Ultrahot Gas Giant on a 16 hr Orbit}, The Astronomical Journal, \textbf{162}, 256 (\arxiv{2111.12074}) [\href{https://ui.adsabs.harvard.edu/abs/2021AJ....162..256W}{32 citations}]

\item[{\color{numcolor}\scriptsize40}] Garai, Z.; Pribulla, T.; Parviainen, H.; Pall{\'e}, E.; \etal\ (incl.\ \textbf{de Leon, J. P.}), 2021, \doi{10.1093/mnras/stab2929}{Is the orbit of the exoplanet WASP-43b really decaying? TESS and MuSCAT2 observations confirm no detection}, Monthly Notices of the Royal Astronomical Society, \textbf{508}, 5514 (\arxiv{2110.04761}) [\href{https://ui.adsabs.harvard.edu/abs/2021MNRAS.508.5514G}{13 citations}]

\item[{\color{numcolor}\scriptsize39}] \textbf{de Leon, J. P.}; Livingston, J.; Endl, M.; Cochran, W. D.; \etal, 2021, \doi{10.1093/mnras/stab2305}{37 new validated planets in overlapping K2 campaigns}, Monthly Notices of the Royal Astronomical Society, \textbf{508}, 195 (\arxiv{2108.05621}) [\href{https://ui.adsabs.harvard.edu/abs/2021MNRAS.508..195D}{25 citations}]

\item[{\color{numcolor}\scriptsize38}] Scarsdale, Nicholas; Murphy, Joseph M. Akana; Batalha, Natalie M.; Crossfield, Ian J. M.; \etal\ (incl.\ \textbf{de Leon, J. P.}), 2021, \doi{10.3847/1538-3881/ac18cb}{TESS-Keck Survey. V. Twin Sub-Neptunes Transiting the Nearby G Star HD 63935}, The Astronomical Journal, \textbf{162}, 215 (\arxiv{2110.06885}) [\href{https://ui.adsabs.harvard.edu/abs/2021AJ....162..215S}{15 citations}]

\item[{\color{numcolor}\scriptsize37}] Hirano, Teruyuki; Livingston, John H.; Fukui, Akihiko; Narita, Norio; \etal\ (incl.\ \textbf{de Leon, J. P.}), 2021, \doi{10.3847/1538-3881/ac0fdc}{Two Bright M Dwarfs Hosting Ultra-Short-Period Super-Earths with Earth-like Compositions}, The Astronomical Journal, \textbf{162}, 161 (\arxiv{2103.12760}) [\href{https://ui.adsabs.harvard.edu/abs/2021AJ....162..161H}{28 citations}]

\item[{\color{numcolor}\scriptsize36}] Teske, Johanna; Wang, Sharon Xuesong; Wolfgang, Angie; Gan, Tianjun; \etal\ (incl.\ \textbf{de Leon, J. P.}), 2021, \doi{10.3847/1538-4365/ac0f0a}{The Magellan-TESS Survey. I. Survey Description and Midsurvey Results}, The Astrophysical Journal Supplement Series, \textbf{256}, 33 (\arxiv{2011.11560}) [\href{https://ui.adsabs.harvard.edu/abs/2021ApJS..256...33T}{38 citations}]

\item[{\color{numcolor}\scriptsize35}] Fukui, A.; Korth, J.; Livingston, J. H.; Twicken, J. D.; \etal\ (incl.\ \textbf{de Leon, J. P.}), 2021, \doi{10.3847/1538-3881/ac13a5}{TOI-1749: an M dwarf with a Trio of Planets including a Near-resonant Pair}, The Astronomical Journal, \textbf{162}, 167 (\arxiv{2107.05430}) [\href{https://ui.adsabs.harvard.edu/abs/2021AJ....162..167F}{7 citations}]

\item[{\color{numcolor}\scriptsize34}] Trifonov, T.; Caballero, J. A.; Morales, J. C.; Seifahrt, A.; \etal\ (incl.\ \textbf{de Leon, J. P.}), 2021, \doi{10.1126/science.abd7645}{A nearby transiting rocky exoplanet that is suitable for atmospheric investigation}, Science, \textbf{371}, 1038 (\arxiv{2103.04950}) [\href{https://ui.adsabs.harvard.edu/abs/2021Sci...371.1038T}{64 citations}]

\item[{\color{numcolor}\scriptsize33}] Parviainen, H.; Palle, E.; Zapatero-Osorio, M. R.; Nowak, G.; \etal\ (incl.\ \textbf{de Leon, J. P.}), 2021, \doi{10.1051/0004-6361/202038934}{TOI-519 b: A short-period substellar object around an M dwarf validated using multicolour photometry and phase curve analysis}, Astronomy and Astrophysics, \textbf{645} (\arxiv{2011.11458}) [\href{https://ui.adsabs.harvard.edu/abs/2021A&A...645A..16P}{23 citations}]

\item[{\color{numcolor}\scriptsize32}] Chen, G.; Pall{\'e}, E.; Parviainen, H.; Wang, H.; \etal\ (incl.\ \textbf{de Leon, J. P.}), 2021, \doi{10.1093/mnras/staa3555}{An enhanced slope in the transmission spectrum of the hot Jupiter WASP-104b}, Monthly Notices of the Royal Astronomical Society, \textbf{500}, 5420 (\arxiv{2011.06329}) [\href{https://ui.adsabs.harvard.edu/abs/2021MNRAS.500.5420C}{17 citations}]

\item[{\color{numcolor}\scriptsize31}] Castro Gonz{\'a}lez, A.; D{\'\i}ez Alonso, E.; Men{\'e}ndez Blanco, J.; Livingston, John H.; \etal\ (incl.\ \textbf{de Leon, J. P.}), 2020, \doi{10.1093/mnras/staa2353}{Planetary candidates transiting cool dwarf stars from campaigns 12 to 15 of K2}, Monthly Notices of the Royal Astronomical Society, \textbf{499}, 5416 (\arxiv{2007.12744}) [\href{https://ui.adsabs.harvard.edu/abs/2020MNRAS.499.5416C}{14 citations}]

\item[{\color{numcolor}\scriptsize30}] Narita, Norio; Fukui, Akihiko; Yamamuro, Tomoyasu; Harbeck, Daniel; \etal\ (incl.\ \textbf{de Leon, J. P.}), 2020, \doi{10.1117/12.2559947}{MuSCAT3: a 4-color simultaneous camera for the 2m Faulkes Telescope North}, Proceedings of the SPIE, \textbf{11447} [\href{https://www.spiedigitallibrary.org/conference-proceedings-of-spie/10925/1092507/Focus-adjustable-motion-blur-compensation-method-using-deformable-mirror/10.1117/12.2509567.short}{23 citations}]

\item[{\color{numcolor}\scriptsize29}] Bouma, L. G.; Hartman, J. D.; Brahm, R.; Evans, P.; \etal\ (incl.\ \textbf{de Leon, J. P.}), 2020, \doi{10.3847/1538-3881/abb9ab}{Cluster Difference Imaging Photometric Survey. II. TOI 837: A Young Validated Planet in IC 2602}, The Astronomical Journal, \textbf{160}, 239 (\arxiv{2009.07845}) [\href{https://ui.adsabs.harvard.edu/abs/2020AJ....160..239B}{56 citations}]

\item[{\color{numcolor}\scriptsize28}] Nowak, G.; Luque, R.; Parviainen, H.; Pall{\'e}, E.; \etal\ (incl.\ \textbf{de Leon, J. P.}), 2020, \doi{10.1051/0004-6361/202037867}{The CARMENES search for exoplanets around M dwarfs. Two planets on opposite sides of the radius gap transiting the nearby M dwarf LTT 3780}, Astronomy and Astrophysics, \textbf{642} (\arxiv{2003.01140}) [\href{https://ui.adsabs.harvard.edu/abs/2020A&A...642A.173N}{66 citations}]

\item[{\color{numcolor}\scriptsize27}] Kemmer, J.; Stock, S.; Kossakowski, D.; Kaminski, A.; \etal\ (incl.\ \textbf{de Leon, J. P.}), 2020, \doi{10.1051/0004-6361/202038967}{Discovery of a hot, transiting, Earth-sized planet and a second temperate, non-transiting planet around the M4 dwarf GJ 3473 (TOI-488)}, Astronomy and Astrophysics, \textbf{642} (\arxiv{2009.10432}) [\href{https://ui.adsabs.harvard.edu/abs/2020A&A...642A.236K}{36 citations}]

\item[{\color{numcolor}\scriptsize26}] Luque, R.; Casasayas-Barris, N.; Parviainen, H.; Chen, G.; \etal\ (incl.\ \textbf{de Leon, J. P.}), 2020, \doi{10.1051/0004-6361/202038703}{Obliquity measurement and atmospheric characterisation of the WASP-74 planetary system}, Astronomy and Astrophysics, \textbf{642} (\arxiv{2007.11851}) [\href{https://ui.adsabs.harvard.edu/abs/2020A&A...642A..50L}{17 citations}]

\item[{\color{numcolor}\scriptsize25}] Nowak, Grzegorz; Palle, Enric; Gandolfi, Davide; Deeg, Hans J.; \etal\ (incl.\ \textbf{de Leon, J. P.}), 2020, \doi{10.1093/mnras/staa2077}{K2-280 b - a low density warm sub-Saturn around a mildly evolved star}, Monthly Notices of the Royal Astronomical Society, \textbf{497}, 4423 (\arxiv{2007.07939}) [\href{https://ui.adsabs.harvard.edu/abs/2020MNRAS.497.4423N}{6 citations}]

\item[{\color{numcolor}\scriptsize24}] Nielsen, L. D.; Brahm, R.; Bouchy, F.; Espinoza, N.; \etal\ (incl.\ \textbf{de Leon, J. P.}), 2020, \doi{10.1051/0004-6361/202037941}{Three short-period Jupiters from TESS. HIP 65Ab, TOI-157b, and TOI-169b}, Astronomy and Astrophysics, \textbf{639} (\arxiv{2003.05932}) [\href{https://ui.adsabs.harvard.edu/abs/2020A&A...639A..76N}{30 citations}]

\item[{\color{numcolor}\scriptsize23}] {\v{S}}ubjak, J{\'a}n; Sharma, Rishikesh; Carmichael, Theron W.; Johnson, Marshall C.; \etal\ (incl.\ \textbf{de Leon, J. P.}), 2020, \doi{10.3847/1538-3881/ab7245}{TOI-503: The First Known Brown-dwarf Am-star Binary from the TESS Mission}, The Astronomical Journal, \textbf{159}, 151 (\arxiv{1909.07984}) [\href{https://ui.adsabs.harvard.edu/abs/2020AJ....159..151S}{43 citations}]

\item[{\color{numcolor}\scriptsize22}] Hidalgo, D.; Pall{\'e}, E.; Alonso, R.; Gandolfi, D.; \etal, 2020, \doi{10.1051/0004-6361/201937080}{Three planets transiting the evolved star EPIC 249893012: a hot 8.8-M{\_}{\ensuremath{\oplus}}{\_} super-Earth and two warm 14.7 and 10.2-M{\_}{\ensuremath{\oplus}}{\_} sub-Neptunes}, Astronomy and Astrophysics, \textbf{636} (\arxiv{2002.01755}) [\href{https://ui.adsabs.harvard.edu/abs/2020A&A...636A..89H}{19 citations}]

\item[{\color{numcolor}\scriptsize21}] Lam, Kristine W. F.; Korth, Judith; Masuda, Kento; Csizmadia, Szil{\'a}rd; \etal\ (incl.\ \textbf{de Leon, J. P.}), 2020, \doi{10.3847/1538-3881/ab66c9}{It Takes Two Planets in Resonance to Tango around K2-146}, The Astronomical Journal, \textbf{159}, 120 (\arxiv{1907.11141}) [\href{https://ui.adsabs.harvard.edu/abs/2020AJ....159..120L}{18 citations}]

\item[{\color{numcolor}\scriptsize20}] Parviainen, H.; Palle, E.; Zapatero-Osorio, M. R.; Montanes Rodriguez, P.; \etal\ (incl.\ \textbf{de Leon, J. P.}), 2020, \doi{10.1051/0004-6361/201935958}{MuSCAT2 multicolour validation of TESS candidates: an ultra-short-period substellar object around an M dwarf}, Astronomy and Astrophysics, \textbf{633} (\arxiv{1911.04366}) [\href{https://ui.adsabs.harvard.edu/abs/2020A&A...633A..28P}{38 citations}]

\item[{\color{numcolor}\scriptsize19}] Mayama, Satoshi; P{\'e}rez, Sebasti{\'a}n; Kusakabe, Nobuhiko; Muto, Takayuki; \etal\ (incl.\ \textbf{de Leon, J. P.}), 2020, \doi{10.3847/1538-3881/ab5850}{Subaru Near-infrared Imaging Polarimetry of Misaligned Disks around the SR 24 Hierarchical Triple System}, The Astronomical Journal, \textbf{159}, 12 (\arxiv{1911.10941}) [\href{https://ui.adsabs.harvard.edu/abs/2020AJ....159...12M}{8 citations}]

\item[{\color{numcolor}\scriptsize18}] Jenkins, James S.; D{\'\i}az, Mat{\'\i}as R.; Kurtovic, Nicol{\'a}s T.; Espinoza, N{\'e}stor; \etal\ (incl.\ \textbf{de Leon, J. P.}), 2020, \doi{10.1038/s41550-020-1142-z}{An ultrahot Neptune in the Neptune desert}, Nature Astronomy, \textbf{4}, 1148 (\arxiv{2009.12832}) [\href{https://ui.adsabs.harvard.edu/abs/2020NatAs...4.1148J}{63 citations}]

\item[{\color{numcolor}\scriptsize17}] Barrag{\'a}n, O.; Aigrain, S.; Kubyshkina, D.; Gandolfi, D.; \etal\ (incl.\ \textbf{de Leon, J. P.}), 2019, \doi{10.1093/mnras/stz2569}{Radial velocity confirmation of K2-100b: a young, highly irradiated, and low-density transiting hot Neptune}, Monthly Notices of the Royal Astronomical Society, \textbf{490}, 698 (\arxiv{1909.05252}) [\href{https://ui.adsabs.harvard.edu/abs/2019MNRAS.490..698B}{63 citations}]

\item[{\color{numcolor}\scriptsize16}] Fukui, A.; Suzuki, D.; Koshimoto, N.; Bachelet, E.; \etal\ (incl.\ \textbf{de Leon, J. P.}), 2019, \doi{10.3847/1538-3881/ab487f}{Kojima-1Lb Is a Mildly Cold Neptune around the Brightest Microlensing Host Star}, The Astronomical Journal, \textbf{158}, 206 (\arxiv{1909.11802}) [\href{https://ui.adsabs.harvard.edu/abs/2019AJ....158..206F}{29 citations}]

\item[{\color{numcolor}\scriptsize15}] Quinn, Samuel N.; Becker, Juliette C.; Rodriguez, Joseph E.; Hadden, Sam; \etal\ (incl.\ \textbf{de Leon, J. P.}), 2019, \doi{10.3847/1538-3881/ab3f2b}{Near-resonance in a System of Sub-Neptunes from TESS}, The Astronomical Journal, \textbf{158}, 177 (\arxiv{1901.09092}) [\href{https://ui.adsabs.harvard.edu/abs/2019AJ....158..177Q}{47 citations}]

\item[{\color{numcolor}\scriptsize14}] Persson, Carina M.; Csizmadia, Szil{\'a}rd; Mustill, Alexander J.; Fridlund, Malcolm; \etal\ (incl.\ \textbf{de Leon, J. P.}), 2019, \doi{10.1051/0004-6361/201935505}{Greening of the brown-dwarf desert. EPIC 212036875b: a 51 M{\_}J{\_} object in a 5-day orbit around an F7 V star}, Astronomy and Astrophysics, \textbf{628} (\arxiv{1906.05048}) [\href{https://ui.adsabs.harvard.edu/abs/2019A&A...628A..64P}{26 citations}]

\item[{\color{numcolor}\scriptsize13}] Hjorth, M.; Justesen, A. B.; Hirano, T.; Albrecht, S.; \etal\ (incl.\ \textbf{de Leon, J. P.}), 2019, \doi{10.1093/mnras/stz139}{K2-290: a warm Jupiter and a mini-Neptune in a triple-star system}, Monthly Notices of the Royal Astronomical Society, \textbf{484}, 3522 (\arxiv{1901.03716}) [\href{https://ui.adsabs.harvard.edu/abs/2019MNRAS.484.3522H}{25 citations}]

\item[{\color{numcolor}\scriptsize12}] Akiyama, Eiji; Vorobyov, Eduard I.; Liu, Hauyu Baobabu; Dong, Ruobing; \etal\ (incl.\ \textbf{de Leon, J. P.}), 2019, \doi{10.3847/1538-3881/ab0ae4}{A Tail Structure Associated with a Protoplanetary Disk around SU Aurigae}, The Astronomical Journal, \textbf{157}, 165 (\arxiv{1902.10306}) [\href{https://ui.adsabs.harvard.edu/abs/2019AJ....157..165A}{38 citations}]

\item[{\color{numcolor}\scriptsize11}] Hayakawa, Tomohiko; Murakami, Kenichi; \textbf{de Leon, J. P.}; \& Ishikawa, Masatoshi, 2019, \doi{10.1117/12.2509567}{Focus adjustable motion-blur compensation method using deformable mirror}, Proc. SPIE 10925, Photonic Instrumentation Engineering VI [\href{https://www.spiedigitallibrary.org/conference-proceedings-of-spie/10925/1092507/Focus-adjustable-motion-blur-compensation-method-using-deformable-mirror/10.1117/12.2509567.short}{2 citations}]

\item[{\color{numcolor}\scriptsize10}] Esposito, M.; Armstrong, D. J.; Gandolfi, D.; Adibekyan, V.; \etal\ (incl.\ \textbf{de Leon, J. P.}), 2019, \doi{10.1051/0004-6361/201834853}{HD 219666 b: a hot-Neptune from TESS Sector 1}, Astronomy and Astrophysics, \textbf{623} (\arxiv{1812.05881}) [\href{https://ui.adsabs.harvard.edu/abs/2019A&A...623A.165E}{34 citations}]

\item[{\color{numcolor}\scriptsize9}] Livingston, John H.; Dai, Fei; Hirano, Teruyuki; Gandolfi, Davide; \etal\ (incl.\ \textbf{de Leon, J. P.}), 2019, \doi{10.1093/mnras/sty3464}{K2-264: a transiting multiplanet system in the Praesepe open cluster}, Monthly Notices of the Royal Astronomical Society, \textbf{484}, 8 (\arxiv{1809.01968}) [\href{https://ui.adsabs.harvard.edu/abs/2019MNRAS.484....8L}{33 citations}]

\item[{\color{numcolor}\scriptsize8}] Narita, Norio; Fukui, Akihiko; Kusakabe, Nobuhiko; Watanabe, Noriharu; \etal\ (incl.\ \textbf{de Leon, J. P.}), 2019, \doi{10.1117/1.JATIS.5.1.015001}{MuSCAT2: four-color simultaneous camera for the 1.52-m Telescopio Carlos S{\'a}nchez}, Journal of Astronomical Telescopes, Instruments, and Systems, \textbf{5}, 15001 (\arxiv{1807.01908}) [\href{https://ui.adsabs.harvard.edu/abs/2019JATIS...5a5001N}{118 citations}]

\item[{\color{numcolor}\scriptsize7}] Mayama, Satoshi; Akiyama, Eiji; Pani{\'c}, Olja; Miley, James; \etal\ (incl.\ \textbf{de Leon, J. P.}), 2018, \doi{10.3847/2041-8213/aae88b}{ALMA Reveals a Misaligned Inner Gas Disk inside the Large Cavity of a Transitional Disk}, The Astrophysical Journal, \textbf{868} (\arxiv{1810.06941}) [\href{https://ui.adsabs.harvard.edu/abs/2018ApJ...868L...3M}{35 citations}]

\item[{\color{numcolor}\scriptsize6}] Takami, Michihiro; Fu, Guangwei; Liu, Hauyu Baobab; Karr, Jennifer L.; \etal\ (incl.\ \textbf{de Leon, J. P.}), 2018, \doi{10.3847/1538-4357/aad2e1}{Near-infrared High-resolution Imaging Polarimetry of FU Ori-type Objects: Toward a Unified Scheme for Low-mass Protostellar Evolution}, The Astrophysical Journal, \textbf{864}, 20 (\arxiv{1807.03499}) [\href{https://ui.adsabs.harvard.edu/abs/2018ApJ...864...20T}{46 citations}]

\item[{\color{numcolor}\scriptsize5}] Livingston, John H.; Endl, Michael; Dai, Fei; Cochran, William D.; \etal\ (incl.\ \textbf{de Leon, J. P.}), 2018, \doi{10.3847/1538-3881/aaccde}{44 Validated Planets from K2 Campaign 10}, The Astronomical Journal, \textbf{156}, 78 (\arxiv{1806.11504}) [\href{https://ui.adsabs.harvard.edu/abs/2018AJ....156...78L}{60 citations}]

\item[{\color{numcolor}\scriptsize4}] Uyama, Taichi; Hashimoto, Jun; Muto, Takayuki; Akiyama, Eiji; \etal\ (incl.\ \textbf{de Leon, J. P.}), 2018, \doi{10.3847/1538-3881/aacbd1}{Subaru/HiCIAO HK {\_}s{\_} Imaging of LKHa 330: Multi-band Detection of the Gap and Spiral-like Structures}, The Astronomical Journal, \textbf{156}, 63 (\arxiv{1804.05934}) [\href{https://ui.adsabs.harvard.edu/abs/2018AJ....156...63U}{27 citations}]

\item[{\color{numcolor}\scriptsize3}] Murakami, Kenichi; Hayakawa, Tomohiko; \textbf{de Leon, J. P.}; \& Ishikawa, Masatoshi, 2018, \doi{10.1117/12.2306621}{Real-time high-speed motion blur compensation method using galvanometer mirror for shape sensing of microfabricated objects}, Proceedings Volume 10679, Optics, Photonics, and Digital Technologies for Imaging Applications V [\href{https://www.spiedigitallibrary.org/conference-proceedings-of-spie/10679/2306621/Real-time-high-speed-motion-blur-compensation-method-using-galvanometer/10.1117/12.2306621.short}{7 citations}]

\item[{\color{numcolor}\scriptsize2}] Yang, Yi; Hashimoto, Jun; Hayashi, Saeko S.; Tamura, Motohide; \etal\ (incl.\ \textbf{de Leon, J. P.}), 2017, \doi{10.3847/1538-3881/153/1/7}{Near-infrared Imaging Polarimetry of Inner Region of GG Tau A Disk}, The Astronomical Journal, \textbf{153}, 7 (\arxiv{1610.09134}) [\href{https://ui.adsabs.harvard.edu/abs/2017AJ....153....7Y}{14 citations}]

\item[{\color{numcolor}\scriptsize1}] \textbf{de Leon, J. P.}; Takami, Michihiro; Karr, Jennifer L.; Hashimoto, Jun; \etal, 2015, \doi{10.1088/2041-8205/806/1/L10}{Near-IR High-resolution Imaging Polarimetry of the SU Aur Disk: Clues for Tidal Tails?}, The Astrophysical Journal, \textbf{806} (\arxiv{1505.03610}) [\href{https://ui.adsabs.harvard.edu/abs/2015ApJ...806L..10D}{15 citations}]
  \end{list}

  % \subsubsection{Preprints \& white papers}
  % \begin{list}{}{\cvlist}
  %   \item[{\color{numcolor}\scriptsize8}] , 2024, \doi{10.48550/arXiv.2401.03715}{Simultaneous multicolour transit photometry of hot Jupiters HAT-P-19b, HAT-P-51b, HAT-P-55b, and HAT-P-65b}, ArXiv (\arxiv{2401.03715})

\item[{\color{numcolor}\scriptsize7}] , 2024, \doi{10.48550/arXiv.2401.05923}{Migration and Evolution of giant ExoPlanets (MEEP) I: Nine Newly Confirmed Hot Jupiters from the TESS Mission}, ArXiv (\arxiv{2401.05923})

\item[{\color{numcolor}\scriptsize6}] , 2023, \doi{10.48550/arXiv.2311.01971}{Photometry of the Didymos system across the DART impact apparition}, ArXiv (\arxiv{2311.01971}) [\href{https://ui.adsabs.harvard.edu/abs/2023arXiv231101971M}{3 citations}]

\item[{\color{numcolor}\scriptsize5}] , 2023, \doi{10.48550/arXiv.2308.09617}{Identification of the Top TESS Objects of Interest for Atmospheric Characterization of Transiting Exoplanets with JWST}, ArXiv (\arxiv{2308.09617})

\item[{\color{numcolor}\scriptsize4}] Yumoto, K.; Tatsumi, E.; Kouyama, T.; Golish, D. R.; \etal\ (incl.\ \textbf{jpdeleon}), 2023, \doi{10.48550/arXiv.2306.13321}{Cross calibration between Hayabusa2/ONC-T and OSIRIS-REx/MapCam for comparative analyses between asteroids Ryugu and Bennu}, ArXiv (\arxiv{2306.13321})

\item[{\color{numcolor}\scriptsize3}] Rizos, J. L.; Asensio-Ramos, A.; Golish, D. R.; DellaGiustina, D. N.; \etal\ (incl.\ \textbf{jpdeleon}), 2021, \doi{10.48550/arXiv.2106.01363}{Using artificial neural networks to improve photometric modeling in airless bodies}, ArXiv (\arxiv{2106.01363})

\item[{\color{numcolor}\scriptsize2}] , 2019, \doi{10.48550/arXiv.1904.11831}{ASIME 2018 White Paper. In-Space Utilisation of Asteroids: Asteroid Composition -- Answers to Questions from the Asteroid Miners}, ArXiv (\arxiv{1904.11831}) [\href{https://ui.adsabs.harvard.edu/abs/2019arXiv190411831G}{3 citations}]

\item[{\color{numcolor}\scriptsize1}] , 2018, \doi{10.48550/arXiv.1809.01148}{Compositional Diversity Among Primitive Asteroids}, ArXiv (\arxiv{1809.01148}) [\href{https://ui.adsabs.harvard.edu/abs/2018arXiv180901148C}{4 citations}]
  % \end{list}
\fi

\subsection{Scholarships}
\begin{list}{}{\cvlist}
    \item 2015/4-2021/3: Ministry of Education, Culture, Sports (MEXT) Scholarship (JPY 40.4M)
    \item 2014/9: SanDisk Japan Scholarship (USD 3.75k, deferred)
    \item 2013/3: International Space University Scholarship (EUR 9k, deferred)
    \item 2011-now: now multiple travel grants to attend workshops in the US and Taiwan related to data science and research-related conferences and observation runs in the U.S., Europe, and South-east Asia
\end{list}

\subsection{Observing Experiences}
\begin{list}{}{\cvlist}
    \item 2021-now: monthly remote observations with Subaru/IRD intensive program (PI: Norio Narita) 
    \item 2018-now: bi-monthly observations with TCS/MuSCAT2
    \item 2018-2020: MuSCAT2/Telescopio Carlo Sanchez, Teide Observatory, Tenerife, Spain (on-site and remote, on-going)
    \item 2018: SIRIUS/IRSF, Sutherland, South Africa (on-site, 2 weeks)
    \item 2018: IRCS/Subaru Telescope, Mauna Kea observatory, Hawaii, USA (on-site, 3 nights)
    \item 2017: CHARIS/Subaru Telescope, Mauna Kea observatory, Hawaii, USA (on-site, 2 nights)
    \item 2017: HDS/Gemini Telescope, Mauna Kea observatory, Hawaii, USA (on-site, 1 night)
    \item 2017-2019: MuSCAT2/Okayama astronomical observatory, Japan (on-site, 20+ nights)
\end{list}

\subsection{Teaching and Mentorships}
\begin{list}{}{\cvlist}
    \item I taught 2 elective courses (Introduction to Astronomy \& Data Analysis) to 4th year undergraduate Aerospace Engineering students at the \href{http://sea.addu.edu.ph/programs/aerospace-engineering/}{Ateneo de Davao University} in 2nd semester of 2022. \\
    \item M. Mori (Current postdoc) and I worked on the discovery and validation of a rare planet found in the so-called Neptune desert (See \href{https://ui.adsabs.harvard.edu/abs/2022AJ....163..298M/abstract}{Mori et al. 2022}; \href{https://github.com/jpdeleon/toi1696}{github.com/jpdeleon/toi1696})
    \item H. Kobayashi (Former undergraduate student) and I worked on creating a pipeline for finding young transiting planets in nearby star clusters and found no planets but several young binary stars (See \href{https://github.com/hiremasa/ytps}{github.com/hiremasa/ytps})
\end{list}

\subsection{Affiliations}
\begin{list}{}{\cvlist}
    \item Association of Filipino Students in Japan (\href{https://www.facebook.com/afsjpage/}{AFSJ}; President, 2017-2018)
    \item Association of Filipino Scholars in Taiwan  (\href{https://www.facebook.com/AssocIskolar/}{AFST}; Founding member)
    \item Science and Technology Advisory Council- Japan chapter (\href{https://www.facebook.com/profile.php?id=100083271798519}{STAC-J}; Member)
\end{list}

\subsection{Professional service \& activities}
\begin{list}{}{\cvlist}
  \item Active referee at the Monthly Notices of the Royal Astronomical Society (\href{https://academic.oup.com/mnras}{MNRAS}; Impact Factor: 4.8)
\end{list}

\end{document}
