% The formatting of this CV is based on @davidwhogg's layout.

\documentclass[12pt,letterpaper]{article}

\usepackage{fontawesome5}
\newcommand{\makefield}[2]{\makebox[1.5em]{#1} #2}

\newcommand{\fullname}{Jerome Pitogo de Leon, Ph.D}
\newcommand{\currentposition}{Postdoctoral researcher at The University of Tokyo}
\newcommand{\address}{3-5-3 Nishi-koiwa Edogawa-ku Tokyo} %133-0057
\newcommand{\phonenumber}{+81 (0) 80 8712 7159}
\newcommand{\email}{jpdeleon@g.ecc.u-tokyo.ac.jp}
\newcommand{\githuburl}{https://jpdeleon.github.com}
\newcommand{\scholarurl}{https://scholar.google.com/citations?hl=en&user=_Z8ialwAAAAJ&view_op=list_works&sortby=pubdate}
\newcommand{\orcid}{0000-0002-6424-3410}
\newcommand{\orcidurl}{https://orcid.org/my-orcid?orcid=0000-0002-6424-3410}
\newcommand{\linkedinurl}{https://www.linkedin.com/in/jpdeleonbsap/}
\newcommand{\paperthree}{https://ui.adsabs.harvard.edu/abs/2023MNRAS.522..750D/abstract}
\newcommand{\papertwo}{https://ui.adsabs.harvard.edu/abs/2021MNRAS.508..195D/abstract}
\newcommand{\paperone}{https://ui.adsabs.harvard.edu/abs/2015ApJ...806L..10D/abstract}
\newcommand{\spieurl}{https://www.spiedigitallibrary.org/conference-proceedings-of-spie/10679/106790Z/Real-time-high-speed-motion-blur-compensation-method-using-galvanometer/10.1117/12.2306621.short}
\newcommand{\naritaurl}{https://www.u-tokyo.ac.jp/focus/en/people/k0001_00784.html}
\newcommand{\tamuraurl}{https://www.u-tokyo.ac.jp/focus/en/people/people003289.htm}
\newcommand{\seseurl}{https://unisec-global.org/rogel.html}
\newcommand{\takamiurl}{https://www.asiaa.sinica.edu.tw/people/cv.php?i=hiro}
\newcommand{\hayakawaurl}{https://ishikawa-vision.org/members/hayakawa/hayakawa-e.html}
\newcommand{\mayamaurl}{https://researchmap.jp/mayama_satoshi?lang=en}
\newcommand{\phdthesisurl}{https://repository.dl.itc.u-tokyo.ac.jp/record/2006678/files/A37944.pdf}
% \newcommand{}{}
\input{cvstyle}
\newcommand{\pubsdate}{2025-11-01}
\newcommand{\pubsfirst}{3}
\newcommand{\pubsnumber}{86}
\newcommand{\pubscitations}{1,908}
\newcommand{\pubshindex}{28}

\begin{document}\thispagestyle{empty}\sloppy\sloppypar\raggedbottom

\textbf{\Large \fullname} \\[0.5ex]
\currentposition \\[1.0ex]
\textsf{\small 
    \makefield{\faMapPin}{\address} %
    \makefield{\faEnvelope[regular]}{\href{mailto:\email}{\texttt{jpdeleon[at]g.ecc.u-tokyo.ac.jp}}}\\%
    \makefield{\faPhone}{\url{\phonenumber}} %
    \makefield{\faBook}{\href{\scholarurl}{Google Scholar}} %
    % \href{\orcidurl}{ORCID: \orcid} | %
    \makefield{\faLinkedin}{\href{\linkedinurl}{\texttt{linkedin.com/in/jpdeleonbsap}}}\\%
    \makefield{\faGlobe}{\url{\githuburl}} %
}\\[0.5ex]

\subsection{Overview}
\begin{list}{}{\cvlist}
    \item \textbf{Research:} Author of \pubsnumber\ refereed publications in exoplanet science with \pubscitations\ total citations and an h-index of \pubshindex\ (as of \pubsdate).\footnote{Publication metrics from NASA ADS.} Research focuses on the discovery and characterization of transiting exoplanets using space- and ground-based telescopes and advanced data analyses.

    \item \textbf{Teaching:} Designed and delivered introductory astronomy courses for undergraduates, five of whom subsequently pursued graduate studies in the Philippines and abroad.

    \item \textbf{Mentorship:} Supervised 2 PhD, 2 master's, and 4 undergraduate students on astronomy research projects. Developed publicly available software tools to support research and instruction.

    \item \textbf{Outreach:} Delivered invited talks for academic and public audiences and founded an astronomy community initiative to promote engagement in astronomy.

\end{list}

\subsection{Education}
\begin{list}{}{\cvlist}

    \item
        \textbf{PhD in Astronomy, 2021} \\
        Department of Astronomy, The University of Tokyo, Japan \\
        Dissertation: \textit{Discovery and Characterization of Transiting Exoplanets with Diverse Radii and Ages} (\href{\phdthesisurl}{online}) \\
        Advisor: \href{\tamuraurl}{Prof.\ Motohide Tamura}

    \item
        \textbf{MSc in Astronomy, 2018} \\
        Department of Astronomy, The University of Tokyo, Japan \\
        Thesis: \textit{Multi-color Simultaneous Transit Observations of Low-density Hot Jupiters} (\href{https://github.com/jpdeleon/thesis-master/tree/master}{online}) \\
        Advisor: Prof.\ Motohide Tamura

    \item
        \textbf{BSc in Physics, 2013} \\
        Department of Physics, University of the Philippines Los Ba\~nos, Laguna, Philippines \\
        Thesis: \textit{Thermal Differential and Image Analysis of a Fabricated 11-cm Solar Telescope Using an Active-Passive Cooling System} \\
        Advisor: \href{\seseurl}{Dr.\ Rogel Mari Sese}

\end{list}

\subsection{Full-time Work Experience}
\begin{list}{}{\cvlist}

    \item
        \textbf{Project Assistant Professor (2025/4--Present)} \\
        \textit{Komaba Institute for Science, Graduate School of Arts and Sciences, The University of Tokyo, Japan} \\
        \textit{Role: Lead coordination of observations and data analysis with \href{\muscatLCOurl}{MuSCAT4} under the JSPS \href{\kibanSurl}{Kiban S} project, including planning observing campaigns, overseeing data reduction, and maintaining analysis workflows.}
        
    \item
        \textbf{Project Research Assistant (2023/4--2025/3)} \\
        \textit{Department of Multi-Disciplinary Sciences, Graduate School of Arts and Sciences, The University of Tokyo, Japan} \\
        \textit{Role: Led the discovery and detailed characterization of young transiting exoplanets using TESS photometry, including target vetting, light-curve modeling, and follow-up coordination.}\\
        \textit{Advisor: \href{\naritaurl}{Prof.\ Norio Narita}} (see \href{\paperthree}{de Leon et al.\ 2023})

    \item
        \textbf{Project Research Assistant (2021/4--2023/3)} \\
        \textit{Department of Astronomy, Graduate School of Science, The University of Tokyo, Japan} \\
        \textit{Role: Coordinated an international collaboration that identified 37 transiting exoplanets from Kepler data, performing transit detection, validation, and system characterization.}\\
        \textit{Advisor: \href{\tamuraurl}{Prof.\ Motohide Tamura}} (see \href{\papertwo}{de Leon et al.\ 2021})
        
    \item
        \textbf{Research Development Assistant (2013/4--2014/6)}\\
        \textit{Regulus SpaceTech, Inc., Los Ba\~nos, Laguna, Philippines}\\
        \textit{Role: Initiated and implemented software for automated solar telescope control and data analysis.}\\
        \textit{Advisor: \href{\seseurl}{Dr.\ Rogel Sese}}

\end{list}

\subsection{Temporary Positions}
\begin{list}{}{\cvlist}
    \item 
        \textbf{Consultant (2024/6--Present)} \\
        \textit{Telos Media (www.telos.so)} \\
        \textit{Role: Accuracy reviewer of Astronomy-related long-form educational videos published on youtube.}
        
    \item
        \textbf{Remote Lecturer (2021/6--2021/10)} \\
        \textit{Faculty of Aerospace Engineering, School of Engineering \& Architecture, Ateneo de Davao University, Philippines} \\
        \textit{Role: Designed and delivered lectures on introductory astronomy and astronomical data analysis for senior undergraduates.}
        
    \item 
        \textbf{Technical Assistant (2018/10--2019/3)}\\
        \textit{Engineering Department, The University of Tokyo, Japan}\\
        \textit{Role: Designed and implemented image-processing software for stitching featureless high-resolution images.}\\
        \textit{Advisor: \href{hayakawaurl}{Assoc.\ Prof.\ Tomohiko Hayakawa}} (see \href{\spieurl}{Murakami et al.\ 2018})
        
    \item 
        \textbf{Research Assistant (2017/4--2018/3)}\\	
        \textit{Department of Astronomy, The University of Tokyo, Japan}\\
        \textit{Role: Designed a pipeline for modeling and analyzing multi-wavelength transit light curves and mentored new group members in its use.}
        
    \item 
        \textbf{Research Student (2015/9--2016/3)}\\
        \textit{Department of Astronomy, Graduate University for Advanced Studies (SOKENDAI), Japan}\\
        \textit{Role: Conducted data reduction and analysis of high-resolution astronomical images and prepared technical content for scientific reports.}\\
        \textit{Advisor: \href{\mayamaurl}{Dr.\ Satoshi Mayama}}
        
    \item 
        \textbf{Intern (2014/7--2014/9)}\\
        \textit{Academia Sinica Institute for Astronomy and Astrophysics (ASIAA), Taipei, Taiwan}\\	
        \textit{Role: Developed an image-processing pipeline to analyze high-resolution images of a young forming star using the 8-m Subaru Telescope.}\\
        \textit{Advisor: \href{takamiurl}{Dr.\ Michihiro Takami}} (see \href{\paperone}{de Leon et al.\ 2015})
\end{list}

\ifdefined\withpubs
  \subsection{Publications}
  refereed: 82 / first author: 3 / citations: 1,767 / h-index: 26 (as of 2025-08-04)
  % \subsubsection{Refereed publications}
  \begin{list}{}{\cvlist}
    \item[{\color{numcolor}\scriptsize192}] de la Fuente Marcos, R.; de la Fuente Marcos, C.; \textbf{de Leon, Jerome}; Alarcon, M. R.; \etal, 2024, \doi{10.1051/0004-6361/202347663}{When the horseshoe fits: Characterizing 2023 FY<SUB>3</SUB> with the 10.4 m Gran Telescopio Canarias and the Two-meter Twin Telescope}, Astronomy and Astrophysics, \textbf{681} (\arxiv{2310.08724})

\item[{\color{numcolor}\scriptsize191}] Mallorqu{\'\i}n, M.; Goffo, E.; Pall{\'e}, E.; Lodieu, N.; \etal\ (incl.\ \textbf{jpdeleon}), 2023, \doi{10.1051/0004-6361/202347346}{TOI-1801 b: A temperate mini-Neptune around a young M0.5 dwarf}, Astronomy and Astrophysics, \textbf{680} (\arxiv{2310.10244})

\item[{\color{numcolor}\scriptsize190}] , 2023, \doi{10.3847/PSJ/ad0a87}{VLT/MUSE Characterization of Dimorphos Ejecta from the DART Impact}, The Planetary Science Journal, \textbf{4}, 238 (\arxiv{2311.09977})

\item[{\color{numcolor}\scriptsize189}] , 2023, \doi{10.3847/PSJ/ad04d8}{Near to Mid-infrared Spectroscopy of (65803) Didymos as Observed by JWST: Characterization Observations Supporting the Double Asteroid Redirection Test}, The Planetary Science Journal, \textbf{4}, 214 (\arxiv{2310.11168}) [\href{https://ui.adsabs.harvard.edu/abs/2023PSJ.....4..214R}{2 citations}]

\item[{\color{numcolor}\scriptsize188}] , 2023, \doi{10.3390/life13112156}{Therapeutic Potential of P110 Peptide: New Insights into Treatment of Alzheimer's Disease}, Life, \textbf{13}, 2156

\item[{\color{numcolor}\scriptsize187}] , 2023, \doi{10.3847/1538-3881/acf56d}{A Massive Hot Jupiter Orbiting a Metal-rich Early M Star Discovered in the TESS Full-frame Images}, The Astronomical Journal, \textbf{166}, 165 (\arxiv{2307.07329}) [\href{https://ui.adsabs.harvard.edu/abs/2023AJ....166..165G}{3 citations}]

\item[{\color{numcolor}\scriptsize186}] Hartman, J. D.; Bakos, G. {\'A}.; Csubry, Z.; Howard, A. W.; \etal\ (incl.\ \textbf{jpdeleon}), 2023, \doi{10.3847/1538-3881/acf56e}{TOI 4201 b and TOI 5344 b: Discovery of Two Transiting Giant Planets around M-dwarf Stars and Revised Parameters for Three Others}, The Astronomical Journal, \textbf{166}, 163 (\arxiv{2307.06809})

\item[{\color{numcolor}\scriptsize185}] Palle, E.; Orell-Miquel, J.; Brady, M.; Bean, J.; \etal\ (incl.\ \textbf{jpdeleon}), 2023, \doi{10.1051/0004-6361/202244261}{GJ 806 (TOI-4481): A bright nearby multi-planetary system with a transiting hot low-density super-Earth}, Astronomy and Astrophysics, \textbf{678} (\arxiv{2301.06873}) [\href{https://ui.adsabs.harvard.edu/abs/2023A&A...678A..80P}{3 citations}]

\item[{\color{numcolor}\scriptsize184}] , 2023, \doi{10.1088/1361-6560/acf5c6}{Towards simulation-free MR-linac treatment: utilizing male pelvis PSMA-PET/CT and population-based electron density assignments}, Physics in Medicine and Biology, \textbf{68}, 195012

\item[{\color{numcolor}\scriptsize183}] Orell-Miquel, J.; Lamp{\'o}n, M.; L{\'o}pez-Puertas, M.; Mallorqu{\'\i}n, M.; \etal\ (incl.\ \textbf{jpdeleon}), 2023, \doi{10.1051/0004-6361/202346445}{Confirmation of an He I evaporating atmosphere around the 650-Myr-old sub-Neptune HD 235088 b (TOI-1430 b) with CARMENES}, Astronomy and Astrophysics, \textbf{677} (\arxiv{2307.05191}) [\href{https://ui.adsabs.harvard.edu/abs/2023A&A...677A..56O}{3 citations}]

\item[{\color{numcolor}\scriptsize182}] Popescu, Marcel M.; V{\u{a}}duvescu, O.; \textbf{de Leon, Jerome}; de la Fuente Marcos, C.; \etal, 2023, \doi{10.1051/0004-6361/202346751}{Discovery and physical characterization as the first response to a potential asteroid collision: The case of 2023 DZ<SUB>2</SUB>}, Astronomy and Astrophysics, \textbf{676} (\arxiv{2306.11347})

\item[{\color{numcolor}\scriptsize181}] , 2023, \doi{10.1093/pasj/psad031}{The mass of TOI-519 b: A close-in giant planet transiting a metal-rich mid-M dwarf}, Publications of the Astronomical Society of Japan, \textbf{75}, 713 (\arxiv{2304.14703}) [\href{https://ui.adsabs.harvard.edu/abs/2023PASJ...75..713K}{7 citations}]

\item[{\color{numcolor}\scriptsize180}] \textbf{de Leon, Jerome}; Livingston, J. H.; Jenkins, J. S.; Vines, J. I.; \etal, 2023, \doi{10.1093/mnras/stad894}{A sub-Neptune transiting the young field star HD 18599 at 40 pc}, Monthly Notices of the Royal Astronomical Society, \textbf{522}, 750 (\arxiv{2210.08179}) [\href{https://ui.adsabs.harvard.edu/abs/2023MNRAS.522..750D}{3 citations}]

\item[{\color{numcolor}\scriptsize179}] Morello, G.; Parviainen, H.; Murgas, F.; Pall{\'e}, E.; \etal\ (incl.\ \textbf{jpdeleon}), 2023, \doi{10.1051/0004-6361/202243592}{TOI-1442 b and TOI-2445 b: Two potentially rocky ultra-short period planets around M dwarfs}, Astronomy and Astrophysics, \textbf{673} (\arxiv{2201.13274}) [\href{https://ui.adsabs.harvard.edu/abs/2023A&A...673A..32M}{4 citations}]

\item[{\color{numcolor}\scriptsize178}] , 2023, \doi{10.1093/mnras/stad708}{Observations of two superfast rotator NEAs: 2021 NY<SUB>1</SUB> and 2022 AB}, Monthly Notices of the Royal Astronomical Society, \textbf{521}, 3784 (\arxiv{2303.05099}) [\href{https://ui.adsabs.harvard.edu/abs/2023MNRAS.521.3784L}{5 citations}]

\item[{\color{numcolor}\scriptsize177}] Hromakina, T.; Birlan, M.; Barucci, M. A.; Fulchignoni, M.; \etal\ (incl.\ \textbf{jpdeleon}), 2023, \doi{10.1093/mnras/stad330}{NEOROCKS project: surface properties of small near-Earth asteroids}, Monthly Notices of the Royal Astronomical Society, \textbf{520}, 3143 (\arxiv{2302.01165})

\item[{\color{numcolor}\scriptsize176}] \textbf{de Leon, Jerome}; Licandro, J.; Pinilla-Alonso, N.; Moskovitz, N.; \etal, 2023, \doi{10.1051/0004-6361/202346278}{Characterisation of the new target of the NASA Lucy mission: Asteroid 152830 Dinkinesh (1999 VD57)}, Astronomy and Astrophysics, \textbf{672} (\arxiv{2303.05918}) [\href{https://ui.adsabs.harvard.edu/abs/2023A&A...672A.174D}{3 citations}]

\item[{\color{numcolor}\scriptsize175}] Tatsumi, E.; Vilas, F.; \textbf{de Leon, Jerome}; Popescu, M.; \etal, 2023, \doi{10.1051/0004-6361/202244499}{Near-ultraviolet absorption distribution of primitive asteroids from spectrophotometric surveys. I. Radial distribution}, Astronomy and Astrophysics, \textbf{672} (\arxiv{2302.08863})

\item[{\color{numcolor}\scriptsize174}] de la Fuente Marcos, R.; \textbf{de Leon, Jerome}; de la Fuente Marcos, C.; Licandro, J.; \etal, 2023, \doi{10.1051/0004-6361/202245514}{Mini-moons from horseshoes: A physical characterization of 2022 NX<SUB>1</SUB> with OSIRIS at the 10.4 m Gran Telescopio Canarias}, Astronomy and Astrophysics, \textbf{670} (\arxiv{2301.10797}) [\href{https://ui.adsabs.harvard.edu/abs/2023A&A...670L..10D}{2 citations}]

\item[{\color{numcolor}\scriptsize173}] Morate, D.; Popescu, M.; Licandro, J.; Tinaut-Ruano, F.; \etal\ (incl.\ \textbf{jpdeleon}), 2023, \doi{10.1093/mnras/stac3530}{Mineralogical analysis of 14 PHAs from ViNOS data}, Monthly Notices of the Royal Astronomical Society, \textbf{519}, 1677

\item[{\color{numcolor}\scriptsize172}] , 2023, \doi{10.1093/mnras/stac2845}{A dense mini-Neptune orbiting the bright young star HD 18599}, Monthly Notices of the Royal Astronomical Society, \textbf{518}, 2627 (\arxiv{2210.07945}) [\href{https://ui.adsabs.harvard.edu/abs/2023MNRAS.518.2627V}{4 citations}]

\item[{\color{numcolor}\scriptsize171}] Tinaut-Ruano, F.; Tatsumi, E.; Tanga, P.; \textbf{de Leon, Jerome}; \etal, 2023, \doi{10.1051/0004-6361/202245134}{Asteroids' reflectance from Gaia DR3: Artificial reddening at near-UV wavelengths}, Astronomy and Astrophysics, \textbf{669} (\arxiv{2301.02157})

\item[{\color{numcolor}\scriptsize170}] , 2023, \doi{10.1371/journal.pone.0278929}{Assessing the role of collectivism and individualism on COVID-19 beliefs and behaviors in the Southeastern United States}, PLoS ONE, \textbf{18}

\item[{\color{numcolor}\scriptsize169}] Kawauchi, K.; Murgas, F.; Palle, E.; Narita, N.; \etal\ (incl.\ \textbf{jpdeleon}), 2022, \doi{10.1051/0004-6361/202243381}{Validation and atmospheric exploration of the sub-Neptune TOI-2136b around a nearby M3 dwarf}, Astronomy and Astrophysics, \textbf{666} (\arxiv{2202.10182}) [\href{https://ui.adsabs.harvard.edu/abs/2022A&A...666A...4K}{6 citations}]

\item[{\color{numcolor}\scriptsize168}] , 2022, \doi{10.1093/mnras/stac1666}{TOI-2119: a transiting brown dwarf orbiting an active M-dwarf from NASA's TESS mission}, Monthly Notices of the Royal Astronomical Society, \textbf{514}, 4944 (\arxiv{2202.08842}) [\href{https://ui.adsabs.harvard.edu/abs/2022MNRAS.514.4944C}{7 citations}]

\item[{\color{numcolor}\scriptsize167}] Ieva, Simone; Mazzotta Epifani, E.; Perna, D.; Dall'Ora, M.; \etal\ (incl.\ \textbf{jpdeleon}), 2022, \doi{10.3847/PSJ/ac7f34}{Spectral Rotational Characterization of the Didymos System prior to the DART Impact}, The Planetary Science Journal, \textbf{3}, 183 [\href{https://ui.adsabs.harvard.edu/abs/2022PSJ.....3..183I}{9 citations}]

\item[{\color{numcolor}\scriptsize166}] Tatsumi, E.; Tinaut-Ruano, F.; \textbf{de Leon, Jerome}; Popescu, M.; \& Licandro, J., 2022, \doi{10.1051/0004-6361/202243806}{Near-ultraviolet to visible spectroscopy of the Themis and Polana-Eulalia complex families}, Astronomy and Astrophysics, \textbf{664} (\arxiv{2205.13917}) [\href{https://ui.adsabs.harvard.edu/abs/2022A&A...664A.107T}{6 citations}]

\item[{\color{numcolor}\scriptsize165}] Pravec, P.; Thomas, C. A.; Rivkin, A. S.; Scheirich, P.; \etal\ (incl.\ \textbf{jpdeleon}), 2022, \doi{10.3847/PSJ/ac7be1}{Photometric Observations of the Binary Near-Earth Asteroid (65803) Didymos in 2015-2021 Prior to DART Impact}, The Planetary Science Journal, \textbf{3}, 175 [\href{https://ui.adsabs.harvard.edu/abs/2022PSJ.....3..175P}{15 citations}]

\item[{\color{numcolor}\scriptsize164}] , 2022, \doi{10.3847/PSJ/ac6f52}{The ESA Hera Mission: Detailed Characterization of the DART Impact Outcome and of the Binary Asteroid (65803) Didymos}, The Planetary Science Journal, \textbf{3}, 160 [\href{https://ui.adsabs.harvard.edu/abs/2022PSJ.....3..160M}{50 citations}]

\item[{\color{numcolor}\scriptsize163}] , 2022, \doi{10.3847/1538-3881/ac517f}{A Possible Alignment Between the Orbits of Planetary Systems and their Visual Binary Companions}, The Astronomical Journal, \textbf{163}, 207 (\arxiv{2202.00042}) [\href{https://ui.adsabs.harvard.edu/abs/2022AJ....163..207C}{19 citations}]

\item[{\color{numcolor}\scriptsize162}] , 2022, \doi{10.3847/PSJ/ac66eb}{Apophis Planetary Defense Campaign}, The Planetary Science Journal, \textbf{3}, 123 [\href{https://ui.adsabs.harvard.edu/abs/2022PSJ.....3..123R}{5 citations}]

\item[{\color{numcolor}\scriptsize161}] , 2022, \doi{10.1093/mnras/stac620}{Nodal precession of WASP-33b for 11 yr by Doppler tomographic and transit photometric observations}, Monthly Notices of the Royal Astronomical Society, \textbf{512}, 4404 (\arxiv{2203.02003}) [\href{https://ui.adsabs.harvard.edu/abs/2022MNRAS.512.4404W}{6 citations}]

\item[{\color{numcolor}\scriptsize160}] , 2022, \doi{10.1093/pasj/psab106}{TOI-2285b: A 1.7 Earth-radius planet near the habitable zone around a nearby M dwarf}, Publications of the Astronomical Society of Japan, \textbf{74} (\arxiv{2110.10215}) [\href{https://ui.adsabs.harvard.edu/abs/2022PASJ...74L...1F}{6 citations}]

\item[{\color{numcolor}\scriptsize159}] Tinaut-Ruano, F.; \textbf{de Leon, Jerome}; Tatsumi, E.; Rousseau, B.; \etal, 2022, \doi{10.1051/0004-6361/202141815}{Spectroscopic study of Ceres' collisional family candidates}, Astronomy and Astrophysics, \textbf{658} (\arxiv{2112.02396})

\item[{\color{numcolor}\scriptsize158}] Castro-Gonz{\'a}lez, A.; D{\'\i}ez Alonso, E.; Men{\'e}ndez Blanco, J.; Livingston, J.; \etal\ (incl.\ \textbf{jpdeleon}), 2022, \doi{10.1093/mnras/stab2669}{The K2-OjOS Project: New and revisited planets and candidates in K2 campaigns 5, 16, {\&} 18}, Monthly Notices of the Royal Astronomical Society, \textbf{509}, 1075 (\arxiv{2109.03346}) [\href{https://ui.adsabs.harvard.edu/abs/2022MNRAS.509.1075C}{6 citations}]

\item[{\color{numcolor}\scriptsize157}] , 2021, \doi{10.3847/1538-3881/ac26bd}{TOI-2109: An Ultrahot Gas Giant on a 16 hr Orbit}, The Astronomical Journal, \textbf{162}, 256 (\arxiv{2111.12074}) [\href{https://ui.adsabs.harvard.edu/abs/2021AJ....162..256W}{21 citations}]

\item[{\color{numcolor}\scriptsize156}] Garai, Z.; Pribulla, T.; Parviainen, H.; Pall{\'e}, E.; \etal\ (incl.\ \textbf{jpdeleon}), 2021, \doi{10.1093/mnras/stab2929}{Is the orbit of the exoplanet WASP-43b really decaying? TESS and MuSCAT2 observations confirm no detection}, Monthly Notices of the Royal Astronomical Society, \textbf{508}, 5514 (\arxiv{2110.04761}) [\href{https://ui.adsabs.harvard.edu/abs/2021MNRAS.508.5514G}{11 citations}]

\item[{\color{numcolor}\scriptsize155}] , 2021, \doi{10.1093/mnras/stab2548}{Widely distributed exogenic materials of varying compositions and morphologies on asteroid (101955) Bennu}, Monthly Notices of the Royal Astronomical Society, \textbf{508}, 2053 (\arxiv{2109.01449}) [\href{https://ui.adsabs.harvard.edu/abs/2021MNRAS.508.2053T}{9 citations}]

\item[{\color{numcolor}\scriptsize154}] , 2021, \doi{10.1126/science.abh1924}{Physiological costs of undocumented human migration across the southern United States border}, Science, \textbf{374}, 1496 [\href{https://ui.adsabs.harvard.edu/abs/2021Sci...374.1496C}{3 citations}]

\item[{\color{numcolor}\scriptsize153}] \textbf{de Leon, Jerome}; Livingston, J.; Endl, M.; Cochran, W. D.; \etal, 2021, \doi{10.1093/mnras/stab2305}{37 new validated planets in overlapping K2 campaigns}, Monthly Notices of the Royal Astronomical Society, \textbf{508}, 195 (\arxiv{2108.05621}) [\href{https://ui.adsabs.harvard.edu/abs/2021MNRAS.508..195D}{15 citations}]

\item[{\color{numcolor}\scriptsize152}] , 2021, \doi{10.3847/1538-3881/ac18cb}{TESS-Keck Survey. V. Twin Sub-Neptunes Transiting the Nearby G Star HD 63935}, The Astronomical Journal, \textbf{162}, 215 (\arxiv{2110.06885}) [\href{https://ui.adsabs.harvard.edu/abs/2021AJ....162..215S}{12 citations}]

\item[{\color{numcolor}\scriptsize151}] Simion, N. G.; Popescu, M.; Licandro, J.; Vaduvescu, O.; \etal\ (incl.\ \textbf{jpdeleon}), 2021, \doi{10.1093/mnras/stab2561}{Spectral properties of near-Earth objects with low-Jovian Tisserand invariant}, Monthly Notices of the Royal Astronomical Society, \textbf{508}, 1128 (\arxiv{2109.12193}) [\href{https://ui.adsabs.harvard.edu/abs/2021MNRAS.508.1128S}{6 citations}]

\item[{\color{numcolor}\scriptsize150}] , 2021, \doi{10.1016/j.icarus.2021.114619}{Spectral diversity of the inner belt primitive asteroid background population}, Icarus, \textbf{368}, 114619

\item[{\color{numcolor}\scriptsize149}] , 2021, \doi{10.3847/1538-3881/ac0fdc}{Two Bright M Dwarfs Hosting Ultra-Short-Period Super-Earths with Earth-like Compositions}, The Astronomical Journal, \textbf{162}, 161 (\arxiv{2103.12760}) [\href{https://ui.adsabs.harvard.edu/abs/2021AJ....162..161H}{22 citations}]

\item[{\color{numcolor}\scriptsize148}] , 2021, \doi{10.3847/1538-4365/ac0f0a}{The Magellan-TESS Survey. I. Survey Description and Midsurvey Results}, The Astrophysical Journal Supplement Series, \textbf{256}, 33 (\arxiv{2011.11560}) [\href{https://ui.adsabs.harvard.edu/abs/2021ApJS..256...33T}{24 citations}]

\item[{\color{numcolor}\scriptsize147}] Fukui, A.; Korth, J.; Livingston, J. H.; Twicken, J. D.; \etal\ (incl.\ \textbf{jpdeleon}), 2021, \doi{10.3847/1538-3881/ac13a5}{TOI-1749: an M dwarf with a Trio of Planets including a Near-resonant Pair}, The Astronomical Journal, \textbf{162}, 167 (\arxiv{2107.05430}) [\href{https://ui.adsabs.harvard.edu/abs/2021AJ....162..167F}{7 citations}]

\item[{\color{numcolor}\scriptsize146}] , 2021, \doi{10.1038/s41467-021-26071-8}{Spectrally blue hydrated parent body of asteroid (162173) Ryugu}, Nature Communications, \textbf{12}, 5837 [\href{https://ui.adsabs.harvard.edu/abs/2021NatCo..12.5837T}{22 citations}]

\item[{\color{numcolor}\scriptsize145}] , 2021, \doi{10.1088/1361-6560/ac25d5}{Automatic radiotherapy delineation quality assurance on prostate MRI with deep learning in a multicentre clinical trial}, Physics in Medicine and Biology, \textbf{66}, 195008

\item[{\color{numcolor}\scriptsize144}] Rizos, J. L.; \textbf{de Leon, Jerome}; Licandro, J.; Golish, D. R.; \etal, 2021, \doi{10.1016/j.icarus.2021.114467}{Bennu's global surface and two candidate sample sites characterized by spectral clustering of OSIRIS-REx multispectral images}, Icarus, \textbf{364}, 114467 (\arxiv{2104.02435}) [\href{https://ui.adsabs.harvard.edu/abs/2021Icar..36414467R}{7 citations}]

\item[{\color{numcolor}\scriptsize143}] Licandro, J.; \textbf{de Leon, Jerome}; Moreno, F.; de la Fuente Marcos, C.; \etal, 2021, \doi{10.1051/0004-6361/202038842}{Activity of the Jupiter co-orbital comet P/2019 LD<SUB>2</SUB> (ATLAS) observed with OSIRIS at the 10.4 m GTC}, Astronomy and Astrophysics, \textbf{650} (\arxiv{2103.14613})

\item[{\color{numcolor}\scriptsize142}] de la Fuente Marcos, C.; de la Fuente Marcos, R.; Licandro, J.; Serra-Ricart, M.; \etal\ (incl.\ \textbf{jpdeleon}), 2021, \doi{10.1051/0004-6361/202039117}{The active centaur 2020 MK<SUB>4</SUB>}, Astronomy and Astrophysics, \textbf{649} (\arxiv{2104.01668}) [\href{https://ui.adsabs.harvard.edu/abs/2021A&A...649A..85D}{9 citations}]

\item[{\color{numcolor}\scriptsize141}] , 2021, \doi{10.3390/ma14102609}{One-Dimensional (1D) Nanostructured Materials for Energy Applications}, Materials, \textbf{14}, 2609 [\href{https://ui.adsabs.harvard.edu/abs/2021Mate...14.2609M}{13 citations}]

\item[{\color{numcolor}\scriptsize140}] , 2021, \doi{10.1016/j.icarus.2020.114210}{Near-infrared spectroscopy of the Sulamitis asteroid family: Surprising similarities in the inner belt primitive asteroid population}, Icarus, \textbf{358}, 114210

\item[{\color{numcolor}\scriptsize139}] Trifonov, T.; Caballero, J. A.; Morales, J. C.; Seifahrt, A.; \etal\ (incl.\ \textbf{jpdeleon}), 2021, \doi{10.1126/science.abd7645}{A nearby transiting rocky exoplanet that is suitable for atmospheric investigation}, Science, \textbf{371}, 1038 (\arxiv{2103.04950}) [\href{https://ui.adsabs.harvard.edu/abs/2021Sci...371.1038T}{44 citations}]

\item[{\color{numcolor}\scriptsize138}] Parviainen, H.; Palle, E.; Zapatero-Osorio, M. R.; Nowak, G.; \etal\ (incl.\ \textbf{jpdeleon}), 2021, \doi{10.1051/0004-6361/202038934}{TOI-519 b: A short-period substellar object around an M dwarf validated using multicolour photometry and phase curve analysis}, Astronomy and Astrophysics, \textbf{645} (\arxiv{2011.11458}) [\href{https://ui.adsabs.harvard.edu/abs/2021A&A...645A..16P}{17 citations}]

\item[{\color{numcolor}\scriptsize137}] Chen, G.; Pall{\'e}, E.; Parviainen, H.; Wang, H.; \etal\ (incl.\ \textbf{jpdeleon}), 2021, \doi{10.1093/mnras/staa3555}{An enhanced slope in the transmission spectrum of the hot Jupiter WASP-104b}, Monthly Notices of the Royal Astronomical Society, \textbf{500}, 5420 (\arxiv{2011.06329}) [\href{https://ui.adsabs.harvard.edu/abs/2021MNRAS.500.5420C}{15 citations}]

\item[{\color{numcolor}\scriptsize136}] , 2021, \doi{10.1016/j.icarus.2020.114028}{Near-infrared spectroscopy of the Chaldaea asteroid family: Possible link to the Klio family}, Icarus, \textbf{354}, 114028 [\href{https://ui.adsabs.harvard.edu/abs/2021Icar..35414028A}{2 citations}]

\item[{\color{numcolor}\scriptsize135}] DellaGiustina, D. N.; Kaplan, H. H.; Simon, A. A.; Bottke, W. F.; \etal\ (incl.\ \textbf{jpdeleon}), 2021, \doi{10.1038/s41550-020-1195-z}{Exogenic basalt on asteroid (101955) Bennu}, Nature Astronomy, \textbf{5}, 31 [\href{https://ui.adsabs.harvard.edu/abs/2021NatAs...5...31D}{55 citations}]

\item[{\color{numcolor}\scriptsize134}] Tatsumi, E.; Sugimoto, C.; Riu, L.; Sugita, S.; \etal\ (incl.\ \textbf{jpdeleon}), 2021, \doi{10.1038/s41550-020-1179-z}{Collisional history of Ryugu's parent body from bright surface boulders}, Nature Astronomy, \textbf{5}, 39 [\href{https://ui.adsabs.harvard.edu/abs/2021NatAs...5...39T}{34 citations}]

\item[{\color{numcolor}\scriptsize133}] , 2020, \doi{10.1093/mnras/staa2353}{Planetary candidates transiting cool dwarf stars from campaigns 12 to 15 of K2}, Monthly Notices of the Royal Astronomical Society, \textbf{499}, 5416 (\arxiv{2007.12744}) [\href{https://ui.adsabs.harvard.edu/abs/2020MNRAS.499.5416C}{10 citations}]

\item[{\color{numcolor}\scriptsize132}] Bouma, L. G.; Hartman, J. D.; Brahm, R.; Evans, P.; \etal\ (incl.\ \textbf{jpdeleon}), 2020, \doi{10.3847/1538-3881/abb9ab}{Cluster Difference Imaging Photometric Survey. II. TOI 837: A Young Validated Planet in IC 2602}, The Astronomical Journal, \textbf{160}, 239 (\arxiv{2009.07845}) [\href{https://ui.adsabs.harvard.edu/abs/2020AJ....160..239B}{40 citations}]

\item[{\color{numcolor}\scriptsize131}] DellaGiustina, D. N.; Burke, K. N.; Walsh, K. J.; Smith, P. H.; \etal\ (incl.\ \textbf{jpdeleon}), 2020, \doi{10.1126/science.abc3660}{Variations in color and reflectance on the surface of asteroid (101955) Bennu}, Science, \textbf{370} [\href{https://ui.adsabs.harvard.edu/abs/2020Sci...370.3660D}{76 citations}]

\item[{\color{numcolor}\scriptsize130}] De Pr{\'a}, M. N.; Pinilla-Alonso, N.; Carvano, J.; Licandro, J.; \etal\ (incl.\ \textbf{jpdeleon}), 2020, \doi{10.1051/0004-6361/202038536}{A comparative analysis of the outer-belt primitive families}, Astronomy and Astrophysics, \textbf{643} (\arxiv{2105.11994}) [\href{https://ui.adsabs.harvard.edu/abs/2020A&A...643A.102D}{6 citations}]

\item[{\color{numcolor}\scriptsize129}] , 2020, \doi{10.1051/0004-6361/202039263}{Spectral characterisation of 14 V-type candidate asteroids from the MOVIS catalogue}, Astronomy and Astrophysics, \textbf{643} (\arxiv{2010.13855}) [\href{https://ui.adsabs.harvard.edu/abs/2020A&A...643A.107M}{5 citations}]

\item[{\color{numcolor}\scriptsize128}] , 2020, \doi{10.3390/s20236741}{A Sensor Fusion Method for Pose Estimation of C-Legged Robots}, Sensors, \textbf{20}, 6741

\item[{\color{numcolor}\scriptsize127}] Nowak, G.; Luque, R.; Parviainen, H.; Pall{\'e}, E.; \etal\ (incl.\ \textbf{jpdeleon}), 2020, \doi{10.1051/0004-6361/202037867}{The CARMENES search for exoplanets around M dwarfs. Two planets on opposite sides of the radius gap transiting the nearby M dwarf LTT 3780}, Astronomy and Astrophysics, \textbf{642} (\arxiv{2003.01140}) [\href{https://ui.adsabs.harvard.edu/abs/2020A&A...642A.173N}{55 citations}]

\item[{\color{numcolor}\scriptsize126}] Kemmer, J.; Stock, S.; Kossakowski, D.; Kaminski, A.; \etal\ (incl.\ \textbf{jpdeleon}), 2020, \doi{10.1051/0004-6361/202038967}{Discovery of a hot, transiting, Earth-sized planet and a second temperate, non-transiting planet around the M4 dwarf GJ 3473 (TOI-488)}, Astronomy and Astrophysics, \textbf{642} (\arxiv{2009.10432}) [\href{https://ui.adsabs.harvard.edu/abs/2020A&A...642A.236K}{30 citations}]

\item[{\color{numcolor}\scriptsize125}] Luque, R.; Casasayas-Barris, N.; Parviainen, H.; Chen, G.; \etal\ (incl.\ \textbf{jpdeleon}), 2020, \doi{10.1051/0004-6361/202038703}{Obliquity measurement and atmospheric characterisation of the WASP-74 planetary system}, Astronomy and Astrophysics, \textbf{642} (\arxiv{2007.11851}) [\href{https://ui.adsabs.harvard.edu/abs/2020A&A...642A..50L}{15 citations}]

\item[{\color{numcolor}\scriptsize124}] , 2020, \doi{10.1093/mnras/staa2077}{K2-280 b - a low density warm sub-Saturn around a mildly evolved star}, Monthly Notices of the Royal Astronomical Society, \textbf{497}, 4423 (\arxiv{2007.07939}) [\href{https://ui.adsabs.harvard.edu/abs/2020MNRAS.497.4423N}{2 citations}]

\item[{\color{numcolor}\scriptsize123}] Barnouin, O. S.; Daly, M. G.; Palmer, E. E.; Gaskell, R. W.; \etal\ (incl.\ \textbf{jpdeleon}), 2020, \doi{10.1038/s41561-020-0643-9}{Author Correction: Shape of (101955) Bennu indicative of a rubble pile with internal stiffness}, Nature Geoscience, \textbf{13}, 764

\item[{\color{numcolor}\scriptsize122}] Popescu, M.; \textbf{de Leon, Jerome}; de la Fuente Marcos, C.; Vaduvescu, O.; \etal, 2020, \doi{10.1093/mnras/staa1728}{Physical characterization of 2020 AV<SUB>2</SUB>, the first known asteroid orbiting inside Venus orbit}, Monthly Notices of the Royal Astronomical Society, \textbf{496}, 3572 (\arxiv{2006.08304}) [\href{https://ui.adsabs.harvard.edu/abs/2020MNRAS.496.3572P}{12 citations}]

\item[{\color{numcolor}\scriptsize121}] Nielsen, L. D.; Brahm, R.; Bouchy, F.; Espinoza, N.; \etal\ (incl.\ \textbf{jpdeleon}), 2020, \doi{10.1051/0004-6361/202037941}{Three short-period Jupiters from TESS. HIP 65Ab, TOI-157b, and TOI-169b}, Astronomy and Astrophysics, \textbf{639} (\arxiv{2003.05932}) [\href{https://ui.adsabs.harvard.edu/abs/2020A&A...639A..76N}{19 citations}]

\item[{\color{numcolor}\scriptsize120}] \textbf{de Leon, Jerome}; Licandro, J.; de la Fuente Marcos, C.; de la Fuente Marcos, R.; \etal, 2020, \doi{10.1093/mnras/staa1190}{Visible and near-infrared observations of interstellar comet 2I/Borisov with the 10.4-m GTC and the 3.6-m TNG telescopes}, Monthly Notices of the Royal Astronomical Society, \textbf{495}, 2053 (\arxiv{2005.00786}) [\href{https://ui.adsabs.harvard.edu/abs/2020MNRAS.495.2053D}{12 citations}]

\item[{\color{numcolor}\scriptsize119}] , 2020, \doi{10.3847/1538-3881/ab7245}{TOI-503: The First Known Brown-dwarf Am-star Binary from the TESS Mission}, The Astronomical Journal, \textbf{159}, 151 (\arxiv{1909.07984}) [\href{https://ui.adsabs.harvard.edu/abs/2020AJ....159..151S}{35 citations}]

\item[{\color{numcolor}\scriptsize118}] , 2020, \doi{10.1051/0004-6361/201937080}{Three planets transiting the evolved star EPIC 249893012: a hot 8.8-M<SUB>{\ensuremath{\oplus}}</SUB> super-Earth and two warm 14.7 and 10.2-M<SUB>{\ensuremath{\oplus}}</SUB> sub-Neptunes}, Astronomy and Astrophysics, \textbf{636} (\arxiv{2002.01755}) [\href{https://ui.adsabs.harvard.edu/abs/2020A&A...636A..89H}{13 citations}]

\item[{\color{numcolor}\scriptsize117}] , 2020, \doi{10.1016/j.icarus.2019.113463}{Analysis in the visible range of NASA Lucy mission targets: Eurybates, Polymele, Orus and Donaldjohanson.}, Icarus, \textbf{338}, 113463 (\arxiv{1907.11451}) [\href{https://ui.adsabs.harvard.edu/abs/2020Icar..33813463S}{6 citations}]

\item[{\color{numcolor}\scriptsize116}] , 2020, \doi{10.1016/j.icarus.2019.113473}{The spectroscopic properties of the Lixiaohua family, cradle of Main Belt Comets}, Icarus, \textbf{338}, 113473 [\href{https://ui.adsabs.harvard.edu/abs/2020Icar..33813473D}{3 citations}]

\item[{\color{numcolor}\scriptsize115}] Mansour, J. -A.; Popescu, M.; \textbf{de Leon, Jerome}; \& Licandro, J., 2020, \doi{10.1093/mnras/stz3284}{Distribution and spectrophotometric classification of basaltic asteroids}, Monthly Notices of the Royal Astronomical Society, \textbf{491}, 5966 (\arxiv{1911.10348}) [\href{https://ui.adsabs.harvard.edu/abs/2020MNRAS.491.5966M}{9 citations}]

\item[{\color{numcolor}\scriptsize114}] Parviainen, H.; Palle, E.; Zapatero-Osorio, M. R.; Montanes Rodriguez, P.; \etal\ (incl.\ \textbf{jpdeleon}), 2020, \doi{10.1051/0004-6361/201935958}{MuSCAT2 multicolour validation of TESS candidates: an ultra-short-period substellar object around an M dwarf}, Astronomy and Astrophysics, \textbf{633} (\arxiv{1911.04366}) [\href{https://ui.adsabs.harvard.edu/abs/2020A&A...633A..28P}{28 citations}]

\item[{\color{numcolor}\scriptsize113}] , 2020, \doi{10.3847/1538-3881/ab5850}{Subaru Near-infrared Imaging Polarimetry of Misaligned Disks around the SR 24 Hierarchical Triple System}, The Astronomical Journal, \textbf{159}, 12 (\arxiv{1911.10941}) [\href{https://ui.adsabs.harvard.edu/abs/2020AJ....159...12M}{7 citations}]

\item[{\color{numcolor}\scriptsize112}] , 2020, \doi{10.1016/j.icarus.2019.113427}{Near-infrared spectroscopy of the Klio primitive inner-belt asteroid family}, Icarus, \textbf{335}, 113427 [\href{https://ui.adsabs.harvard.edu/abs/2020Icar..33513427A}{3 citations}]

\item[{\color{numcolor}\scriptsize111}] , 2020, \doi{10.1038/s41550-020-1142-z}{An ultrahot Neptune in the Neptune desert}, Nature Astronomy, \textbf{4}, 1148 (\arxiv{2009.12832}) [\href{https://ui.adsabs.harvard.edu/abs/2020NatAs...4.1148J}{47 citations}]

\item[{\color{numcolor}\scriptsize110}] , 2020, \doi{10.1038/s41550-020-01252-5}{Author Correction: An ultrahot Neptune in the Neptune desert}, Nature Astronomy, \textbf{4}, 1202

\item[{\color{numcolor}\scriptsize109}] Lauretta, D. S.; Hergenrother, C. W.; Chesley, S. R.; Leonard, J. M.; \etal\ (incl.\ \textbf{jpdeleon}), 2019, \doi{10.1126/science.aay3544}{Episodes of particle ejection from the surface of the active asteroid (101955) Bennu}, Science, \textbf{366} [\href{https://ui.adsabs.harvard.edu/abs/2019Sci...366.3544L}{99 citations}]

\item[{\color{numcolor}\scriptsize108}] Barrag{\'a}n, O.; Aigrain, S.; Kubyshkina, D.; Gandolfi, D.; \etal\ (incl.\ \textbf{jpdeleon}), 2019, \doi{10.1093/mnras/stz2569}{Radial velocity confirmation of K2-100b: a young, highly irradiated, and low-density transiting hot Neptune}, Monthly Notices of the Royal Astronomical Society, \textbf{490}, 698 (\arxiv{1909.05252}) [\href{https://ui.adsabs.harvard.edu/abs/2019MNRAS.490..698B}{45 citations}]

\item[{\color{numcolor}\scriptsize107}] Fukui, A.; Suzuki, D.; Koshimoto, N.; Bachelet, E.; \etal\ (incl.\ \textbf{jpdeleon}), 2019, \doi{10.3847/1538-3881/ab487f}{Kojima-1Lb Is a Mildly Cold Neptune around the Brightest Microlensing Host Star}, The Astronomical Journal, \textbf{158}, 206 (\arxiv{1909.11802}) [\href{https://ui.adsabs.harvard.edu/abs/2019AJ....158..206F}{21 citations}]

\item[{\color{numcolor}\scriptsize106}] , 2019, \doi{10.1051/0004-6361/201935992}{The last pieces of the primitive inner belt puzzle: Klio, Chaldaea, Chimaera, and Svea}, Astronomy and Astrophysics, \textbf{630} [\href{https://ui.adsabs.harvard.edu/abs/2019A&A...630A.141M}{13 citations}]

\item[{\color{numcolor}\scriptsize105}] , 2019, \doi{10.3847/2515-5172/ab449c}{Interstellar Visitors: A Physical Characterization of Comet C/2019 Q4 (Borisov) with OSIRIS at the 10.4{\,}m GTC}, Research Notes of the American Astronomical Society, \textbf{3}, 131 [\href{https://ui.adsabs.harvard.edu/abs/2019RNAAS...3..131D}{45 citations}]

\item[{\color{numcolor}\scriptsize104}] Medeiros, H.; \textbf{de Leon, Jerome}; Lazzaro, D.; Popescu, M.; \etal, 2019, \doi{10.1093/mnras/stz2001}{Compositional characterization of V-type candidate asteroids identified using the MOVIS catalogue}, Monthly Notices of the Royal Astronomical Society, \textbf{488}, 3866 [\href{https://ui.adsabs.harvard.edu/abs/2019MNRAS.488.3866M}{9 citations}]

\item[{\color{numcolor}\scriptsize103}] , 2019, \doi{10.1051/0004-6361/201935505}{Greening of the brown-dwarf desert. EPIC 212036875b: a 51 M<SUB>J</SUB> object in a 5-day orbit around an F7 V star}, Astronomy and Astrophysics, \textbf{628} (\arxiv{1906.05048}) [\href{https://ui.adsabs.harvard.edu/abs/2019A&A...628A..64P}{20 citations}]

\item[{\color{numcolor}\scriptsize102}] , 2019, \doi{10.1016/j.icarus.2019.03.007}{Spectral clustering tools applied to Ceres in preparation for OSIRIS-REx color imaging of asteroid (101955) Bennu}, Icarus, \textbf{328}, 69 (\arxiv{1903.11559}) [\href{https://ui.adsabs.harvard.edu/abs/2019Icar..328...69R}{3 citations}]

\item[{\color{numcolor}\scriptsize101}] Popescu, M.; Vaduvescu, O.; \textbf{de Leon, Jerome}; Gherase, R. M.; \etal, 2019, \doi{10.1051/0004-6361/201935006}{Near-Earth asteroids spectroscopic survey at Isaac Newton Telescope}, Astronomy and Astrophysics, \textbf{627} (\arxiv{1905.12997}) [\href{https://ui.adsabs.harvard.edu/abs/2019A&A...627A.124P}{24 citations}]

\item[{\color{numcolor}\scriptsize100}] Licandro, J.; de la Fuente Marcos, C.; de la Fuente Marcos, R.; \textbf{de Leon, Jerome}; \etal, 2019, \doi{10.1051/0004-6361/201834902}{Spectroscopic and dynamical properties of comet C/2018 F4, likely a true average former member of the Oort cloud}, Astronomy and Astrophysics, \textbf{625} (\arxiv{1903.10838}) [\href{https://ui.adsabs.harvard.edu/abs/2019A&A...625A.133L}{14 citations}]

\item[{\color{numcolor}\scriptsize99}] Hjorth, M.; Justesen, A. B.; Hirano, T.; Albrecht, S.; \etal\ (incl.\ \textbf{jpdeleon}), 2019, \doi{10.1093/mnras/stz139}{K2-290: a warm Jupiter and a mini-Neptune in a triple-star system}, Monthly Notices of the Royal Astronomical Society, \textbf{484}, 3522 (\arxiv{1901.03716}) [\href{https://ui.adsabs.harvard.edu/abs/2019MNRAS.484.3522H}{18 citations}]

\item[{\color{numcolor}\scriptsize98}] , 2019, \doi{10.3847/1538-3881/ab0ae4}{A Tail Structure Associated with a Protoplanetary Disk around SU Aurigae}, The Astronomical Journal, \textbf{157}, 165 (\arxiv{1902.10306}) [\href{https://ui.adsabs.harvard.edu/abs/2019AJ....157..165A}{31 citations}]

\item[{\color{numcolor}\scriptsize97}] Barnouin, O. S.; Daly, M. G.; Palmer, E. E.; Gaskell, R. W.; \etal\ (incl.\ \textbf{jpdeleon}), 2019, \doi{10.1038/s41561-019-0330-x}{Shape of (101955) Bennu indicative of a rubble pile with internal stiffness}, Nature Geoscience, \textbf{12}, 247 [\href{https://ui.adsabs.harvard.edu/abs/2019NatGe..12..247B}{170 citations}]

\item[{\color{numcolor}\scriptsize96}] Esposito, M.; Armstrong, D. J.; Gandolfi, D.; Adibekyan, V.; \etal\ (incl.\ \textbf{jpdeleon}), 2019, \doi{10.1051/0004-6361/201834853}{HD 219666 b: a hot-Neptune from TESS Sector 1}, Astronomy and Astrophysics, \textbf{623} (\arxiv{1812.05881}) [\href{https://ui.adsabs.harvard.edu/abs/2019A&A...623A.165E}{29 citations}]

\item[{\color{numcolor}\scriptsize95}] , 2019, \doi{10.1093/mnras/sty3464}{K2-264: a transiting multiplanet system in the Praesepe open cluster}, Monthly Notices of the Royal Astronomical Society, \textbf{484}, 8 (\arxiv{1809.01968}) [\href{https://ui.adsabs.harvard.edu/abs/2019MNRAS.484....8L}{30 citations}]

\item[{\color{numcolor}\scriptsize94}] Aznar, Amadeo; \textbf{de Leon, Jerome}; Popescu, M.; Serra-Ricart, M.; \etal, 2019, \doi{10.1093/mnras/sty3250}{Physical properties of PHA 2014 JO<SUB>25</SUB> from a worldwide observational campaign}, Monthly Notices of the Royal Astronomical Society, \textbf{483}, 4820 [\href{https://ui.adsabs.harvard.edu/abs/2019MNRAS.483.4820A}{4 citations}]

\item[{\color{numcolor}\scriptsize93}] , 2019, \doi{10.1117/1.JATIS.5.1.015001}{MuSCAT2: four-color simultaneous camera for the 1.52-m Telescopio Carlos S{\'a}nchez}, Journal of Astronomical Telescopes, Instruments, and Systems, \textbf{5}, 15001 (\arxiv{1807.01908}) [\href{https://ui.adsabs.harvard.edu/abs/2019JATIS...5a5001N}{81 citations}]

\item[{\color{numcolor}\scriptsize92}] , 2019, \doi{10.1016/j.petrol.2019.106376}{Multiscale simulation of asphaltene deposition in pipeline flows}, Journal of Petroleum Science and Engineering, \textbf{183}, 106376

\item[{\color{numcolor}\scriptsize91}] DellaGiustina, D. N.; Bennett, C. A.; Becker, K.; Golish, D. R.; \etal\ (incl.\ \textbf{jpdeleon}), 2018, \doi{10.1029/2018EA000382}{Overcoming the Challenges Associated with Image-Based Mapping of Small Bodies in Preparation for the OSIRIS-REx Mission to (101955) Bennu}, Earth and Space Science, \textbf{5}, 929 (\arxiv{1810.10080}) [\href{https://ui.adsabs.harvard.edu/abs/2018E&SS....5..929D}{21 citations}]

\item[{\color{numcolor}\scriptsize90}] , 2018, \doi{10.3847/2041-8213/aae88b}{ALMA Reveals a Misaligned Inner Gas Disk inside the Large Cavity of a Transitional Disk}, The Astrophysical Journal, \textbf{868} (\arxiv{1810.06941}) [\href{https://ui.adsabs.harvard.edu/abs/2018ApJ...868L...3M}{31 citations}]

\item[{\color{numcolor}\scriptsize89}] Licandro, J.; Popescu, M.; \textbf{de Leon, Jerome}; Morate, D.; \etal, 2018, \doi{10.1051/0004-6361/201832853}{The visible and near-infrared spectra of asteroids in cometary orbits}, Astronomy and Astrophysics, \textbf{618} (\arxiv{1903.10880}) [\href{https://ui.adsabs.harvard.edu/abs/2018A&A...618A.170L}{22 citations}]

\item[{\color{numcolor}\scriptsize88}] \textbf{de Leon, Jerome}; Campins, H.; Morate, D.; De Pr{\'a}, M.; \etal, 2018, \doi{10.1016/j.icarus.2018.05.009}{Expected spectral characteristics of (101955) Bennu and (162173) Ryugu, targets of the OSIRIS-REx and Hayabusa2 missions}, Icarus, \textbf{313}, 25 (\arxiv{1805.08774}) [\href{https://ui.adsabs.harvard.edu/abs/2018Icar..313...25D}{22 citations}]

\item[{\color{numcolor}\scriptsize87}] Bowles, N. E.; Snodgrass, C.; Gibbings, A.; Sanchez, J. P.; \etal\ (incl.\ \textbf{jpdeleon}), 2018, \doi{10.1016/j.asr.2017.10.021}{CASTAway: An asteroid main belt tour and survey}, Advances in Space Research, \textbf{62}, 1998 (\arxiv{1710.10191}) [\href{https://ui.adsabs.harvard.edu/abs/2018AdSpR..62.1998B}{19 citations}]

\item[{\color{numcolor}\scriptsize86}] , 2018, \doi{10.18637/jss.v086.i07}{DNest4: Diffusive Nested Sampling in C++ and Python}, Journal of Statistical Software, \textbf{86}, 1 (\arxiv{1606.03757}) [\href{https://scholar.google.com/scholar?cites=789224875040810871}{42 citations}]

\item[{\color{numcolor}\scriptsize85}] Popescu, M.; Licandro, J.; Carvano, J. M.; Stoicescu, R.; \etal\ (incl.\ \textbf{jpdeleon}), 2018, \doi{10.1051/0004-6361/201833023}{Taxonomic classification of asteroids based on MOVIS near-infrared colors}, Astronomy and Astrophysics, \textbf{617} (\arxiv{1807.00713}) [\href{https://ui.adsabs.harvard.edu/abs/2018A&A...617A..12P}{43 citations}]

\item[{\color{numcolor}\scriptsize84}] , 2018, \doi{10.3847/1538-4357/aad2e1}{Near-infrared High-resolution Imaging Polarimetry of FU Ori-type Objects: Toward a Unified Scheme for Low-mass Protostellar Evolution}, The Astrophysical Journal, \textbf{864}, 20 (\arxiv{1807.03499}) [\href{https://ui.adsabs.harvard.edu/abs/2018ApJ...864...20T}{38 citations}]

\item[{\color{numcolor}\scriptsize83}] , 2018, \doi{10.1016/j.icarus.2017.11.012}{PRIMASS visits Hilda and Cybele groups}, Icarus, \textbf{311}, 35 (\arxiv{1711.02071}) [\href{https://ui.adsabs.harvard.edu/abs/2018Icar..311...35D}{22 citations}]

\item[{\color{numcolor}\scriptsize82}] , 2018, \doi{10.1051/0004-6361/201832780}{Color study of asteroid families within the MOVIS catalog}, Astronomy and Astrophysics, \textbf{617} [\href{https://ui.adsabs.harvard.edu/abs/2018A&A...617A..72M}{8 citations}]

\item[{\color{numcolor}\scriptsize81}] , 2018, \doi{10.3847/1538-3881/aaccde}{44 Validated Planets from K2 Campaign 10}, The Astronomical Journal, \textbf{156}, 78 (\arxiv{1806.11504}) [\href{https://ui.adsabs.harvard.edu/abs/2018AJ....156...78L}{51 citations}]

\item[{\color{numcolor}\scriptsize80}] , 2018, \doi{10.3847/1538-3881/aacbd1}{Subaru/HiCIAO HK <SUB>s</SUB> Imaging of LKHa 330: Multi-band Detection of the Gap and Spiral-like Structures}, The Astronomical Journal, \textbf{156}, 63 (\arxiv{1804.05934}) [\href{https://ui.adsabs.harvard.edu/abs/2018AJ....156...63U}{25 citations}]

\item[{\color{numcolor}\scriptsize79}] , 2018, \doi{10.1051/0004-6361/201731407}{Visible spectroscopy of the Sulamitis and Clarissa primitive families: a possible link to Erigone and Polana}, Astronomy and Astrophysics, \textbf{610} [\href{https://ui.adsabs.harvard.edu/abs/2018A&A...610A..25M}{17 citations}]

\item[{\color{numcolor}\scriptsize78}] , 2017, \doi{10.1016/j.icarus.2017.04.004}{Physical and dynamical properties of the anomalous comet 249P/LINEAR}, Icarus, \textbf{295}, 34 (\arxiv{1704.04639}) [\href{https://ui.adsabs.harvard.edu/abs/2017Icar..295...34F}{14 citations}]

\item[{\color{numcolor}\scriptsize77}] \textbf{de Leon, Jerome}; de la Fuente Marcos, C.; \& de la Fuente Marcos, R., 2017, \doi{10.1093/mnrasl/slx003}{Visible spectra of (474640) 2004 VN<SUB>112</SUB>-2013 RF<SUB>98</SUB> with OSIRIS at the 10.4 m GTC: evidence for binary dissociation near aphelion among the extreme trans-Neptunian objects}, Monthly Notices of the Royal Astronomical Society, \textbf{467} (\arxiv{1701.02534}) [\href{https://ui.adsabs.harvard.edu/abs/2017MNRAS.467L..66D}{13 citations}]

\item[{\color{numcolor}\scriptsize76}] Snodgrass, C.; A'Hearn, M. F.; Aceituno, F.; Afanasiev, V.; \etal\ (incl.\ \textbf{jpdeleon}), 2017, \doi{10.1098/rsta.2016.0249}{The 67P/Churyumov-Gerasimenko observation campaign in support of the Rosetta mission}, Philosophical Transactions of the Royal Society of London Series A, \textbf{375}, 20160249 (\arxiv{1705.10539}) [\href{https://ui.adsabs.harvard.edu/abs/2017RSPTA.37560249S}{37 citations}]

\item[{\color{numcolor}\scriptsize75}] Licandro, J.; Popescu, M.; Morate, D.; \& \textbf{de Leon, Jerome}, 2017, \doi{10.1051/0004-6361/201629465}{V-type candidates and Vesta family asteroids in the Moving Objects VISTA (MOVIS) catalogue}, Astronomy and Astrophysics, \textbf{600} (\arxiv{1701.04621}) [\href{https://ui.adsabs.harvard.edu/abs/2017A&A...600A.126L}{21 citations}]

\item[{\color{numcolor}\scriptsize74}] Popescu, M.; Licandro, J.; Morate, D.; \textbf{de Leon, Jerome}; \& Nedelcu, D. A., 2017, Minor Planet Science with the VISTA Hemisphere Survey, The Messenger, \textbf{167}, 16 [\href{https://ui.adsabs.harvard.edu/abs/2017Msngr.167...16P}{2 citations}]

\item[{\color{numcolor}\scriptsize73}] , 2017, \doi{10.3847/1538-3881/153/1/7}{Near-infrared Imaging Polarimetry of Inner Region of GG Tau A Disk}, The Astronomical Journal, \textbf{153}, 7 (\arxiv{1610.09134}) [\href{https://ui.adsabs.harvard.edu/abs/2017AJ....153....7Y}{14 citations}]

\item[{\color{numcolor}\scriptsize72}] , 2017, \doi{10.1039/C6CP06380B}{A stochastic method for asphaltene structure formulation from experimental data: avoidance of implausible structures}, Physical Chemistry Chemical Physics (Incorporating Faraday Transactions), \textbf{19}, 9934 [\href{https://ui.adsabs.harvard.edu/abs/2017PCCP...19.9934D}{2 citations}]

\item[{\color{numcolor}\scriptsize71}] Pinilla-Alonso, Noem{\'\i}; \textbf{de Leon, Jerome}; Walsh, K. J.; Campins, H.; \etal, 2016, \doi{10.1016/j.icarus.2016.03.022}{Portrait of the Polana-Eulalia family complex: Surface homogeneity revealed from near-infrared spectroscopy}, Icarus, \textbf{274}, 231 [\href{https://ui.adsabs.harvard.edu/abs/2016Icar..274..231P}{22 citations}]

\item[{\color{numcolor}\scriptsize70}] , 2016, \doi{10.3390/s16071018}{Heterogeneous Multi-Robot System for Mapping Environmental Variables of Greenhouses}, Sensors, \textbf{16}, 1018 [\href{https://ui.adsabs.harvard.edu/abs/2016Senso..16.1018R}{13 citations}]

\item[{\color{numcolor}\scriptsize69}] Popescu, M.; Licandro, J.; Morate, D.; \textbf{de Leon, Jerome}; \etal, 2016, \doi{10.1051/0004-6361/201628163}{Near-infrared colors of minor planets recovered from VISTA-VHS survey (MOVIS)}, Astronomy and Astrophysics, \textbf{591} (\arxiv{1605.05594}) [\href{https://ui.adsabs.harvard.edu/abs/2016A&A...591A.115P}{42 citations}]

\item[{\color{numcolor}\scriptsize68}] Al{\'\i}-Lagoa, V.; Licandro, J.; Gil-Hutton, R.; Ca{\~n}ada-Assandri, M.; \etal\ (incl.\ \textbf{jpdeleon}), 2016, \doi{10.1051/0004-6361/201527660}{Differences between the Pallas collisional family and similarly sized B-type asteroids}, Astronomy and Astrophysics, \textbf{591} [\href{https://ui.adsabs.harvard.edu/abs/2016A&A...591A..14A}{23 citations}]

\item[{\color{numcolor}\scriptsize67}] , 2016, \doi{10.21105/joss.00024}{corner.py: Scatterplot matrices in Python}, The Journal of Open Source Software, \textbf{1}, 2 [\href{https://scholar.google.com/scholar?cites=1835087844145558435,17325274697099535179,14220488595059618709,12820425635803494730,7284810048757141243,17415935839493019063}{1940 citations}]

\item[{\color{numcolor}\scriptsize66}] , 2016, \doi{10.1073/pnas.1511656113}{Modeling confounding by half-sibling regression}, PNAS, \textbf{113}, 27 [\href{https://scholar.google.com/scholar?cites=2429561747341807338}{68 citations}]

\item[{\color{numcolor}\scriptsize65}] \textbf{de Leon, Jerome}; Pinilla-Alonso, N.; Delbo, M.; Campins, H.; \etal, 2016, \doi{10.1016/j.icarus.2015.11.014}{Visible spectroscopy of the Polana-Eulalia family complex: Spectral homogeneity}, Icarus, \textbf{266}, 57 [\href{https://ui.adsabs.harvard.edu/abs/2016Icar..266...57D}{30 citations}]

\item[{\color{numcolor}\scriptsize64}] , 2016, \doi{10.1051/0004-6361/201527453}{Compositional study of asteroids in the Erigone collisional family using visible spectroscopy at the 10.4 m GTC}, Astronomy and Astrophysics, \textbf{586} (\arxiv{1701.03761}) [\href{https://ui.adsabs.harvard.edu/abs/2016A&A...586A.129M}{25 citations}]

\item[{\color{numcolor}\scriptsize63}] , 2016, \doi{10.1109/TPAMI.2015.2448083}{Fast Direct Methods for Gaussian Processes}, IEEE Transactions on Pattern Analysis and Machine Intelligence, \textbf{38}, 252 (\arxiv{1403.6015}) [\href{https://scholar.google.com/scholar?cites=7122560326210979193,5194420368165307096,3347404430934682534}{668 citations}]

\item[{\color{numcolor}\scriptsize62}] , 2015, \doi{10.1088/2041-8205/806/1/L10}{Near-IR High-resolution Imaging Polarimetry of the SU Aur Disk: Clues for Tidal Tails?}, The Astrophysical Journal, \textbf{806} (\arxiv{1505.03610}) [\href{https://ui.adsabs.harvard.edu/abs/2015ApJ...806L..10D}{15 citations}]

\item[{\color{numcolor}\scriptsize61}] Vaduvescu, O.; Hudin, L.; Tudor, V.; Char, F.; \etal\ (incl.\ \textbf{jpdeleon}), 2015, \doi{10.1093/mnras/stv266}{First EURONEAR NEA discoveries from La Palma using the INT}, Monthly Notices of the Royal Astronomical Society, \textbf{449}, 1614 [\href{https://ui.adsabs.harvard.edu/abs/2015MNRAS.449.1614V}{14 citations}]

\item[{\color{numcolor}\scriptsize60}] , 2015, Removing systematic errors for exoplanet search via latent causes, ICML, \textbf{37}, 2218 (\arxiv{1505.03036}) [\href{https://scholar.google.com/scholar?cites=11768165421845046384}{11 citations}]

\item[{\color{numcolor}\scriptsize59}] , 2014, \doi{10.1080/00207179.2014.904523}{High-order sliding mode observers and integral backstepping sensorless control of IPMS motor}, International Journal of Control, \textbf{87}, 2176

\item[{\color{numcolor}\scriptsize58}] , 2013, \doi{10.1088/0004-6256/146/2/26}{The Origin of Asteroid 162173 (1999 JU<SUB>3</SUB>)}, The Astronomical Journal, \textbf{146}, 26 [\href{https://ui.adsabs.harvard.edu/abs/2013AJ....146...26C}{45 citations}]

\item[{\color{numcolor}\scriptsize57}] \textbf{de Leon, Jerome}; Lorenzi, V.; Al{\'\i}-Lagoa, V.; Licandro, J.; \etal, 2013, \doi{10.1051/0004-6361/201321514}{Additional spectra of asteroid 1996 FG3, backup target of the ESA MarcoPolo-R mission}, Astronomy and Astrophysics, \textbf{556} (\arxiv{1307.5774}) [\href{https://ui.adsabs.harvard.edu/abs/2013A&A...556A..33D}{9 citations}]

\item[{\color{numcolor}\scriptsize56}] \textbf{de Leon, Jerome}; Ortiz, J. L.; Pinilla-Alonso, N.; Cabrera-Lavers, A.; \etal, 2013, \doi{10.1051/0004-6361/201321373}{Visible and near-infrared observations of asteroid 2012 DA<SUB>14</SUB> during its closest approach of February 15, 2013}, Astronomy and Astrophysics, \textbf{555} (\arxiv{1303.0554}) [\href{https://ui.adsabs.harvard.edu/abs/2013A&A...555L...2D}{13 citations}]

\item[{\color{numcolor}\scriptsize55}] Al{\'\i}-Lagoa, V.; \textbf{de Leon, Jerome}; Licandro, J.; Delb{\'o}, M.; \etal, 2013, \doi{10.1051/0004-6361/201220680}{Physical properties of B-type asteroids from WISE data}, Astronomy and Astrophysics, \textbf{554} (\arxiv{1303.5487}) [\href{https://ui.adsabs.harvard.edu/abs/2013A&A...554A..71A}{33 citations}]

\item[{\color{numcolor}\scriptsize54}] , 2013, \doi{10.1016/j.jas.2013.01.033}{Early Olmec obsidian trade and economic organization at San Lorenzo}, Journal of Archaeological Science, \textbf{40}, 2784 [\href{https://ui.adsabs.harvard.edu/abs/2013JArSc..40.2784H}{6 citations}]

\item[{\color{numcolor}\scriptsize53}] , 2013, \doi{10.3897/zookeys.312.4535}{A new species of Perinereis (Polychaeta, Nereididae) from Florida, USA, with a key to all Perinereis from the American continent}, ZooKeys, \textbf{312}, 1 [\href{https://ui.adsabs.harvard.edu/abs/2013ZooK..312....1D}{3 citations}]

\item[{\color{numcolor}\scriptsize52}] Pinilla-Alonso, N.; Lorenzi, V.; Campins, H.; \textbf{de Leon, Jerome}; \& Licandro, J., 2013, \doi{10.1051/0004-6361/201221015}{Near-infrared spectroscopy of 1999 JU<SUB>3</SUB>, the target of the Hayabusa 2 mission}, Astronomy and Astrophysics, \textbf{552} (\arxiv{1303.3512}) [\href{https://ui.adsabs.harvard.edu/abs/2013A&A...552A..79P}{16 citations}]

\item[{\color{numcolor}\scriptsize51}] Licandro, J.; Moreno, F.; \textbf{de Leon, Jerome}; Tozzi, G. P.; \etal, 2013, \doi{10.1051/0004-6361/201220080}{Exploring the nature of new main-belt comets with the 10.4 m GTC telescope: (300163) 2006 VW139}, Astronomy and Astrophysics, \textbf{550} (\arxiv{1212.1022}) [\href{https://ui.adsabs.harvard.edu/abs/2013A&A...550A..17L}{33 citations}]

\item[{\color{numcolor}\scriptsize50}] , 2013, \doi{10.3897/zookeys.269.4003}{A new species of Nicon Kinberg, 1866 (Polychaeta, Nereididae) from Ecuador, Eastern Pacific, with a key to all known species of the genus}, ZooKeys, \textbf{269}, 67 [\href{https://ui.adsabs.harvard.edu/abs/2013ZooK..269...67D}{3 citations}]

\item[{\color{numcolor}\scriptsize49}] , 2012, \doi{10.1016/j.pss.2012.06.017}{Spectra of asteroid families in support of Gaia}, Planetary and Space Science, \textbf{73}, 95 [\href{https://ui.adsabs.harvard.edu/abs/2012P&SS...73...95C}{5 citations}]

\item[{\color{numcolor}\scriptsize48}] , 2012, \doi{10.1186/1687-6180-2012-223}{Blind separation of overlapping partials in harmonic musical notes using amplitude and phase reconstruction}, EURASIP Journal on Applied Signal Processing, \textbf{2012}

\item[{\color{numcolor}\scriptsize47}] \textbf{de Leon, Jerome}; Pinilla-Alonso, N.; Campins, H.; Licandro, J.; \& Marzo, G. A., 2012, \doi{10.1016/j.icarus.2011.11.024}{Near-infrared spectroscopic survey of B-type asteroids: Compositional analysis}, Icarus, \textbf{218}, 196 [\href{https://ui.adsabs.harvard.edu/abs/2012Icar..218..196D}{65 citations}]

\item[{\color{numcolor}\scriptsize46}] , 2011, \doi{10.3897/zookeys.151.1964}{Review of the Capitellidae (Annelida, Polychaeta) from the Eastern Tropical Pacific region, with notes on selected species}, ZooKeys, \textbf{151}, 17 [\href{https://ui.adsabs.harvard.edu/abs/2011ZooK..151...17G}{2 citations}]

\item[{\color{numcolor}\scriptsize45}] , 2011, \doi{10.1080/00207179.2011.629684}{Sensorless speed control of PMSM via adaptive interconnected observer}, International Journal of Control, \textbf{84}, 1926 [\href{https://ui.adsabs.harvard.edu/abs/2011IJC....84.1926E}{3 citations}]

\item[{\color{numcolor}\scriptsize44}] , 2011, \doi{10.1126/science.1207325}{Images of Asteroid 21 Lutetia: A Remnant Planetesimal from the Early Solar System}, Science, \textbf{334}, 487 [\href{https://ui.adsabs.harvard.edu/abs/2011Sci...334..487S}{131 citations}]

\item[{\color{numcolor}\scriptsize43}] Moreno, F.; Lara, L. M.; Licandro, J.; Ortiz, J. L.; \etal\ (incl.\ \textbf{jpdeleon}), 2011, \doi{10.1088/2041-8205/738/1/L16}{The Dust Environment of Main-Belt Comet P/2010 R2 (La Sagra)}, The Astrophysical Journal, \textbf{738} [\href{https://ui.adsabs.harvard.edu/abs/2011ApJ...738L..16M}{41 citations}]

\item[{\color{numcolor}\scriptsize42}] Licandro, J.; Campins, H.; Tozzi, G. P.; \textbf{de Leon, Jerome}; \etal, 2011, \doi{10.1051/0004-6361/201117018}{Testing the comet nature of main belt comets. The spectra of 133P/Elst-Pizarro and 176P/LINEAR}, Astronomy and Astrophysics, \textbf{532} (\arxiv{1104.0879}) [\href{https://ui.adsabs.harvard.edu/abs/2011A&A...532A..65L}{51 citations}]

\item[{\color{numcolor}\scriptsize41}] \textbf{de Leon, Jerome}; Moth{\'e}-Diniz, T.; Licandro, J.; Pinilla-Alonso, N.; \& Campins, H., 2011, \doi{10.1051/0004-6361/201117041}{New observations of asteroid (175706) 1996 FG3, primary target of the ESA Marco Polo-R mission}, Astronomy and Astrophysics, \textbf{530} [\href{https://ui.adsabs.harvard.edu/abs/2011A&A...530L..12D}{22 citations}]

\item[{\color{numcolor}\scriptsize40}] \textbf{de Leon, Jerome}; Duffard, R.; Lara, L. M.; \& Lin, Z. -Y., 2011, \doi{10.1051/0004-6361/201015144}{Photometric and spectroscopic observations of asteroid (21) Lutetia three months before the Rosetta fly-by}, Astronomy and Astrophysics, \textbf{527} [\href{https://ui.adsabs.harvard.edu/abs/2011A&A...527A..42D}{2 citations}]

\item[{\color{numcolor}\scriptsize39}] , 2010, \doi{10.1073/pnas.1012362107}{From the Cover: Anaphase-promoting complex/cyclosome-Cdh1 coordinates glycolysis and glutaminolysis with transition to S phase in human T lymphocytes}, Proceedings of the National Academy of Science, \textbf{107}, 18868 [\href{https://ui.adsabs.harvard.edu/abs/2010PNAS..10718868C}{9 citations}]

\item[{\color{numcolor}\scriptsize38}] , 2010, \doi{10.1088/2041-8205/721/1/L53}{The Origin of Asteroid 101955 (1999 RQ<SUB>36</SUB>)}, The Astrophysical Journal, \textbf{721} [\href{https://ui.adsabs.harvard.edu/abs/2010ApJ...721L..53C}{71 citations}]

\item[{\color{numcolor}\scriptsize37}] \textbf{de Leon, Jerome}; Licandro, J.; Serra-Ricart, M.; Pinilla-Alonso, N.; \& Campins, H., 2010, \doi{10.1051/0004-6361/200913852}{Observations, compositional, and physical characterization of near-Earth and Mars-crosser asteroids from a spectroscopic survey}, Astronomy and Astrophysics, \textbf{517} [\href{https://ui.adsabs.harvard.edu/abs/2010A&A...517A..23D}{82 citations}]

\item[{\color{numcolor}\scriptsize36}] \textbf{de Leon, Jerome}; Campins, H.; Tsiganis, K.; Morbidelli, A.; \& Licandro, J., 2010, \doi{10.1051/0004-6361/200913609}{Origin of the near-Earth asteroid Phaethon and the Geminids meteor shower}, Astronomy and Astrophysics, \textbf{513} [\href{https://ui.adsabs.harvard.edu/abs/2010A&A...513A..26D}{89 citations}]

\item[{\color{numcolor}\scriptsize35}] , 2010, \doi{10.1080/00207170903193474}{A robust sensorless output feedback controller of the induction motor drives: new design and experimental validation}, International Journal of Control, \textbf{83}, 484 [\href{https://ui.adsabs.harvard.edu/abs/2010IJC....83..484G}{2 citations}]

\item[{\color{numcolor}\scriptsize34}] , 2009, \doi{10.1016/j.chaos.2009.04.055}{Synchronization in reduced-order of chaotic systems via control approaches based on high-order sliding-mode observer}, Chaos Solitons and Fractals, \textbf{42}, 3219 [\href{https://ui.adsabs.harvard.edu/abs/2009CSF....42.3219R}{9 citations}]

\item[{\color{numcolor}\scriptsize33}] , 2009, \doi{10.1016/j.cnsns.2009.03.007}{A dynamic parameter estimator to control chaos with distinct feedback schemes}, Communications in Nonlinear Science and Numerical Simulations, \textbf{14}, 4280

\item[{\color{numcolor}\scriptsize32}] , 2008, \doi{10.1007/s00442-008-1095-x}{Identifying the predator complex of Homalodisca vitripennis (Hemiptera: Cicadellidae): a comparative study of the efficacy of an ELISA and PCR gut content assay}, Oecologia, \textbf{157}, 629 [\href{https://ui.adsabs.harvard.edu/abs/2008Oecol.157..629F}{29 citations}]

\item[{\color{numcolor}\scriptsize31}] Licandro, J.; Alvarez-Candal, A.; \textbf{de Leon, Jerome}; Pinilla-Alonso, N.; \etal, 2008, \doi{10.1051/0004-6361:20078340e}{Spectral properties of asteroids in cometary orbits}, Astronomy and Astrophysics, \textbf{487}, 1195 [\href{https://ui.adsabs.harvard.edu/abs/2008A&A...487.1195L}{4 citations}]

\item[{\color{numcolor}\scriptsize30}] Licandro, J.; Alvarez-Candal, A.; \textbf{de Leon, Jerome}; Pinilla-Alonso, N.; \etal, 2008, \doi{10.1051/0004-6361:20078340}{Spectral properties of asteroids in cometary orbits}, Astronomy and Astrophysics, \textbf{481}, 861 [\href{https://ui.adsabs.harvard.edu/abs/2008A&A...481..861L}{35 citations}]

\item[{\color{numcolor}\scriptsize29}] , 2007, \doi{10.1086/519974}{Nuclear Spectra of Comet 28P Neujmin 1}, The Astronomical Journal, \textbf{134}, 1626 [\href{https://ui.adsabs.harvard.edu/abs/2007AJ....134.1626C}{9 citations}]

\item[{\color{numcolor}\scriptsize28}] Brunetto, R.; \textbf{de Leon, Jerome}; \& Licandro, J., 2007, \doi{10.1051/0004-6361:20077722}{Testing space weathering models on A-type asteroid (1951) Lick}, Astronomy and Astrophysics, \textbf{472}, 653 [\href{https://ui.adsabs.harvard.edu/abs/2007A&A...472..653B}{13 citations}]

\item[{\color{numcolor}\scriptsize27}] , 2007, Strong guiding of light in hollow nanowire structures, Chinese Optics Letters, \textbf{5}, 543

\item[{\color{numcolor}\scriptsize26}] , 2007, \doi{10.1016/j.chaos.2005.11.064}{Chaos suppression based on adaptive observer for a P -class of chaotic systems}, Chaos Solitons and Fractals, \textbf{32}, 1345

\item[{\color{numcolor}\scriptsize25}] Licandro, J.; Campins, H.; Moth{\'e}-Diniz, T.; Pinilla-Alonso, N.; \& \textbf{de Leon, Jerome}, 2007, \doi{10.1051/0004-6361:20065833}{The nature of comet-asteroid transition object (3200) Phaethon}, Astronomy and Astrophysics, \textbf{461}, 751 [\href{https://ui.adsabs.harvard.edu/abs/2007A&A...461..751L}{95 citations}]

\item[{\color{numcolor}\scriptsize24}] Licandro, J.; di Fabrizio, L.; Pinilla-Alonso, N.; \textbf{de Leon, Jerome}; \& Oliva, E., 2006, \doi{10.1051/0004-6361:20064906}{Trans-neptunian object (55636) 2002 TX<SUB>300</SUB>, a fresh icy surface in the outer solar system}, Astronomy and Astrophysics, \textbf{457}, 329 [\href{https://ui.adsabs.harvard.edu/abs/2006A&A...457..329L}{17 citations}]

\item[{\color{numcolor}\scriptsize23}] , 2006, \doi{10.1086/506253}{Nuclear Spectra of Comet 162P/Siding Spring (2004 TU12)}, The Astronomical Journal, \textbf{132}, 1346 [\href{https://ui.adsabs.harvard.edu/abs/2006AJ....132.1346C}{36 citations}]

\item[{\color{numcolor}\scriptsize22}] Duffard, R.; \textbf{de Leon, Jerome}; Licandro, J.; Lazzaro, D.; \& Serra-Ricart, M., 2006, \doi{10.1051/0004-6361:20054347}{Basaltic asteroids in the Near-Earth Objects population: a mineralogical analysis}, Astronomy and Astrophysics, \textbf{456}, 775 [\href{https://ui.adsabs.harvard.edu/abs/2006A&A...456..775D}{16 citations}]

\item[{\color{numcolor}\scriptsize21}] Licandro, J.; \textbf{de Leon, Jerome}; Pinilla, N.; \& Serra-Ricart, M., 2006, \doi{10.1016/j.asr.2005.11.015}{Multi-wavelength spectral study of asteroids in cometary orbits}, Advances in Space Research, \textbf{38}, 1991 [\href{https://ui.adsabs.harvard.edu/abs/2006AdSpR..38.1991L}{16 citations}]

\item[{\color{numcolor}\scriptsize20}] \textbf{de Leon, Jerome}, 2006, \doi{10.1007/s10714-005-0209-6}{Transition from decelerated to accelerated cosmic expansion in braneworld universes}, General Relativity and Gravitation, \textbf{38}, 61 (\arxiv{gr-qc/0412005}) [\href{https://ui.adsabs.harvard.edu/abs/2006GReGr..38...61D}{9 citations}]

\item[{\color{numcolor}\scriptsize19}] \textbf{de Leon, Jerome}; Licandro, J.; Duffard, R.; \& Serra-Ricart, M., 2006, \doi{10.1016/j.asr.2005.05.074}{Spectral analysis and mineralogical characterization of 11 olivine pyroxene rich NEAs}, Advances in Space Research, \textbf{37}, 178 [\href{https://ui.adsabs.harvard.edu/abs/2006AdSpR..37..178D}{26 citations}]

\item[{\color{numcolor}\scriptsize18}] \textbf{de Leon, Jerome}, 2006, \doi{10.1142/S0218271806008929}{An Analytical Model for the Transition from Decelerated to Accelerated Cosmic Expansion}, International Journal of Modern Physics D, \textbf{15}, 1237 (\arxiv{gr-qc/0511150}) [\href{https://ui.adsabs.harvard.edu/abs/2006IJMPD..15.1237D}{17 citations}]

\item[{\color{numcolor}\scriptsize17}] , 2006, \doi{10.1016/j.gexplo.2005.12.014}{Effects of artificial and natural recharge on chemical equilibrium in the Cerro Prieto reservoirs, Baja California, M{\'e}xico}, Journal of Geochemical Exploration, \textbf{89}, 339 [\href{https://ui.adsabs.harvard.edu/abs/2006JCExp..89..339P}{3 citations}]

\item[{\color{numcolor}\scriptsize16}] Lara, L. M.; \textbf{de Leon, Jerome}; Licandro, J.; \& Guti{\'e}rrez, P. J., 2005, \doi{10.1007/s11038-006-9067-9}{Dust Activity in Comet 67P/Churyumov Gerasimenko from February 20 to April 20, 2003}, Earth Moon and Planets, \textbf{97}, 165 [\href{https://ui.adsabs.harvard.edu/abs/2005EM&P...97..165L}{13 citations}]

\item[{\color{numcolor}\scriptsize15}] Duffard, R.; Lazzaro, D.; \& \textbf{de Leon, Jerome}, 2005, \doi{10.1111/j.1945-5100.2005.tb00393.x}{Revisiting spectral parameters of silicate-bearing meteorites}, Meteoritics and Planetary Science, \textbf{40}, 445 [\href{https://ui.adsabs.harvard.edu/abs/2005M&PS...40..445D}{22 citations}]

\item[{\color{numcolor}\scriptsize14}] \textbf{de Leon, Jerome}; Duffard, R.; Licandro, J.; \& Lazzaro, D., 2004, \doi{10.1051/0004-6361:20048009}{Mineralogical characterization of A-type asteroid (1951) Lick}, Astronomy and Astrophysics, \textbf{422} [\href{https://ui.adsabs.harvard.edu/abs/2004A&A...422L..59D}{18 citations}]

\item[{\color{numcolor}\scriptsize13}] \textbf{de Leon, Jerome}, 2004, \doi{10.1023/B:GERG.0000022391.57597.3b}{Extra Force from an Extra Dimension: Comparison Between Brane Theory, Space-Time-Matter Theory, and Other Approaches}, General Relativity and Gravitation, \textbf{36}, 1335 [\href{https://ui.adsabs.harvard.edu/abs/2004GReGr..36.1335D}{13 citations}]

\item[{\color{numcolor}\scriptsize12}] \textbf{de Leon, Jerome}, 2004, \doi{10.1023/B:GERG.0000022581.83991.b4}{Letter: Electromagnetic Mass-Models in General Relativity Reexamined}, General Relativity and Gravitation, \textbf{36}, 1453 [\href{https://ui.adsabs.harvard.edu/abs/2004GReGr..36.1453D}{7 citations}]

\item[{\color{numcolor}\scriptsize11}] \textbf{de Leon, Jerome}, 2004, \doi{10.1023/B:GERG.0000018082.54353.cd}{Brane-World Models Emerging from Collisions of Plane Waves in 5D}, General Relativity and Gravitation, \textbf{36}, 923 (\arxiv{gr-qc/0212058}) [\href{https://ui.adsabs.harvard.edu/abs/2004GReGr..36..923D}{15 citations}]

\item[{\color{numcolor}\scriptsize10}] Guti{\'e}rrez, P. J.; \textbf{de Leon, Jerome}; Jorda, L.; Licandro, J.; \etal, 2003, \doi{10.1051/0004-6361:20031066}{New spin period determination for comet 6P/d'Arrest}, Astronomy and Astrophysics, \textbf{407} [\href{https://ui.adsabs.harvard.edu/abs/2003A&A...407L..37G}{12 citations}]

\item[{\color{numcolor}\scriptsize9}] \textbf{de Leon, Jerome}, 2002, \doi{10.1142/S0217732302009143}{Variation of G, {\ensuremath{\Lambda}}<SUB>(4)</SUB> and Vacuum Energy from Brane-World Models}, Modern Physics Letters A, \textbf{17}, 2425 (\arxiv{gr-qc/0207001}) [\href{https://ui.adsabs.harvard.edu/abs/2002MPLA...17.2425D}{19 citations}]

\item[{\color{numcolor}\scriptsize8}] \textbf{de Leon, Jerome}, 2001, \doi{10.1016/S0370-2693(01)01349-1}{Does the force from an extra dimension contradict physics in 4D?}, Physics Letters B, \textbf{523}, 311 (\arxiv{gr-qc/0110063}) [\href{https://ui.adsabs.harvard.edu/abs/2001PhLB..523..311D}{28 citations}]

\item[{\color{numcolor}\scriptsize7}] \textbf{de Leon, Jerome}, 2001, \doi{10.1142/S0217732301005709}{Equivalence between Space-Time-Matter and Brane-World Theories}, Modern Physics Letters A, \textbf{16}, 2291 (\arxiv{gr-qc/0111011}) [\href{https://ui.adsabs.harvard.edu/abs/2001MPLA...16.2291D}{92 citations}]

\item[{\color{numcolor}\scriptsize6}] \textbf{de Leon, Jerome}, 2001, \doi{10.1142/S0217732301004595}{Inevitability of Space-Time Singularities in the Canonical Metric}, Modern Physics Letters A, \textbf{16}, 1405 [\href{https://ui.adsabs.harvard.edu/abs/2001MPLA...16.1405D}{14 citations}]

\item[{\color{numcolor}\scriptsize5}] Wesson, Paul S.; \& \textbf{de Leon, Jerome}, 1994, \doi{10.1007/BF02107998}{Kaluza-Klein theory and Machian cosmology}, General Relativity and Gravitation, \textbf{26}, 555 [\href{https://ui.adsabs.harvard.edu/abs/1994GReGr..26..555W}{20 citations}]

\item[{\color{numcolor}\scriptsize4}] Liu, Hongya; Wesson, P.; \& \textbf{de Leon, Jerome}, 1993, \doi{10.1063/1.530027}{Time-dependent Kaluza-Klein soliton solutions.}, Journal of Mathematical Physics, \textbf{34}, 4070 [\href{https://ui.adsabs.harvard.edu/abs/1993JMP....34.4070L}{30 citations}]

\item[{\color{numcolor}\scriptsize3}] \textbf{de Leon, Jerome}; \& Wesson, P. S., 1993, Exact solutions and the effective equation of state in Kaluza-Klein theory., Journal of Mathematical Physics, \textbf{34}, 4080 [\href{https://ui.adsabs.harvard.edu/abs/1993JMP....34.4080D}{24 citations}]

\item[{\color{numcolor}\scriptsize2}] Wesson, P. S.; \& \textbf{de Leon, Jerome}, 1992, \doi{10.1063/1.529834}{Kaluza-Klein equations, Einstein's equations, and an effective energy-momentum tensor.}, Journal of Mathematical Physics, \textbf{33}, 3883 [\href{https://ui.adsabs.harvard.edu/abs/1992JMP....33.3883W}{200 citations}]

\item[{\color{numcolor}\scriptsize1}] \textbf{de Leon, Jerome}, 1991, Cosmological models that describe particle creation in the early universe and evolve into the ''present-day'' universe., Journal of Mathematical Physics, \textbf{32}, 3546 [\href{https://ui.adsabs.harvard.edu/abs/1991JMP....32.3546D}{2 citations}]
  \end{list}

%   \subsubsection{Preprints \& white papers}
%   \begin{list}{}{\cvlist}
%     \item[{\color{numcolor}\scriptsize8}] , 2024, \doi{10.48550/arXiv.2401.03715}{Simultaneous multicolour transit photometry of hot Jupiters HAT-P-19b, HAT-P-51b, HAT-P-55b, and HAT-P-65b}, ArXiv (\arxiv{2401.03715})

\item[{\color{numcolor}\scriptsize7}] , 2024, \doi{10.48550/arXiv.2401.05923}{Migration and Evolution of giant ExoPlanets (MEEP) I: Nine Newly Confirmed Hot Jupiters from the TESS Mission}, ArXiv (\arxiv{2401.05923})

\item[{\color{numcolor}\scriptsize6}] , 2023, \doi{10.48550/arXiv.2311.01971}{Photometry of the Didymos system across the DART impact apparition}, ArXiv (\arxiv{2311.01971}) [\href{https://ui.adsabs.harvard.edu/abs/2023arXiv231101971M}{3 citations}]

\item[{\color{numcolor}\scriptsize5}] , 2023, \doi{10.48550/arXiv.2308.09617}{Identification of the Top TESS Objects of Interest for Atmospheric Characterization of Transiting Exoplanets with JWST}, ArXiv (\arxiv{2308.09617})

\item[{\color{numcolor}\scriptsize4}] Yumoto, K.; Tatsumi, E.; Kouyama, T.; Golish, D. R.; \etal\ (incl.\ \textbf{jpdeleon}), 2023, \doi{10.48550/arXiv.2306.13321}{Cross calibration between Hayabusa2/ONC-T and OSIRIS-REx/MapCam for comparative analyses between asteroids Ryugu and Bennu}, ArXiv (\arxiv{2306.13321})

\item[{\color{numcolor}\scriptsize3}] Rizos, J. L.; Asensio-Ramos, A.; Golish, D. R.; DellaGiustina, D. N.; \etal\ (incl.\ \textbf{jpdeleon}), 2021, \doi{10.48550/arXiv.2106.01363}{Using artificial neural networks to improve photometric modeling in airless bodies}, ArXiv (\arxiv{2106.01363})

\item[{\color{numcolor}\scriptsize2}] , 2019, \doi{10.48550/arXiv.1904.11831}{ASIME 2018 White Paper. In-Space Utilisation of Asteroids: Asteroid Composition -- Answers to Questions from the Asteroid Miners}, ArXiv (\arxiv{1904.11831}) [\href{https://ui.adsabs.harvard.edu/abs/2019arXiv190411831G}{3 citations}]

\item[{\color{numcolor}\scriptsize1}] , 2018, \doi{10.48550/arXiv.1809.01148}{Compositional Diversity Among Primitive Asteroids}, ArXiv (\arxiv{1809.01148}) [\href{https://ui.adsabs.harvard.edu/abs/2018arXiv180901148C}{4 citations}]
%   \end{list}
\fi

\subsection{Awards}
\begin{list}{}{\cvlist}
    \item \textbf{2022/12:} St.\ Joseph's Academy Centennial Achiever's Award
    \item \textbf{2021/3:} Philippine Embassy in Japan Medal for Distinguished Filipino Graduates
    \item \textbf{2018/4--2021/3:} MEXT PhD Scholarship, Ministry of Education, Culture, Sports, Science and Technology (JPY 8.4M)
    \item \textbf{2016/4--2018/3:} MEXT MSc Scholarship (JPY 5.6M)
    \item \textbf{2015/4--2016/3:} MEXT Japanese Language and Research Scholarship (JPY 2.8M)
    \item \textbf{2015/10--2016/3:} Research Grant, Graduate University for Advanced Studies (JPY 0.2M)
    \item \textbf{2014/9:} SanDisk Japan Scholarship (USD 3.75k, deferred)
    \item \textbf{2013/3:} International Space University Scholarship (EUR 9k, deferred)
    \item \textbf{2011--present:} Multiple travel grants for data science workshops and research conferences, including observation runs in the U.S., Europe, South Africa, and Southeast Asia
\end{list}

\subsection{Teaching and Mentorships}
\begin{list}{}{\cvlist}
    \item \textbf{Designed and Taught Courses:} Created and delivered two elective courses for 4th-year Aerospace Engineering undergraduates at the \href{http://sea.addu.edu.ph/programs/aerospace-engineering/}{Ateneo de Davao University} during the 2nd semester of 2021.
    
    \item \textbf{Research Collaborations:} Collaborated with several international teams on multi-year projects resulting in research publications in high-impact, peer-reviewed journals.
    
    \item \textbf{Pipeline Development:} Developed publicly available Python code to analyze data from the TESS telescope for research and instructional use (see \href{https://github.com/jpdeleon/quicklook}{github.com/jpdeleon/quicklook}).
    
    \item \textbf{Undergraduate Thesis Advisory:} Mentored K.M. Shariat on Bayesian Statistical Modeling; mentored K. Uno on estimating stellar ages using multiple techniques as part of a Research Introduction class; mentored H. Kobayashi on developing a pipeline to find young transiting planets and binary stars in nearby clusters (see \href{https://github.com/hiremasa/ytps}{github.com/hiremasa/ytps}). Also advised A. Nodado on an undergraduate thesis on variable star characterization.
    
    \item \textbf{MEXT Scholarship Guidance:} Guided A. Javier and C. Cabatlao from application to acceptance, supporting their successful pursuit of graduate studies in Japan.
\end{list}

\subsection{Trainings and Workshops}
\begin{list}{}{\cvlist}
    \item University of Tokyo Faculty Development (\href{https://utokyofd.com/en/ffdp/about/}{UTokyo FD Open}) Program, Tokyo, Japan, Oct 2025--Feb 2026
    \item \href{https://www.exoclock.space/annual_meetings}{5th ExoClock Meeting}, Madrid, Spain, Oct 4--6, 2025
    \item \href{https://eventos.upm.es/132958/detail/ariel-consortium-meeting-madrid.html}{ARIEL Consortium Meeting}, Madrid, Spain, Sep 30--Oct 4, 2025
    \item \href{https://sites.google.com/g.ecc.u-tokyo.ac.jp/young-transiting-planet-worksh/home}{Young Transiting Planet Workshop 2025}, Ishigaki, Japan, Mar 12--15, 2025
    \item \href{https://indico.narit.or.th/event/203/}{JWST Data Analysis Workshop}, Chiang Mai, Thailand, Jun 24--Jul 5, 2024
    \item Japan Planetary Atmosphere Research Society Workshop II, Atami, Japan, Mar 2023
    \item Japan Planetary Atmosphere Research Society Workshop, Matsushima, Miyagi, Japan, Mar 2022
    \item RESCEU Symposium: From Protoplanetary Disks through Planetary System Architecture to Planetary Atmospheres and Habitability, Okinawa, Japan, Oct 14--18, 2019
    \item \href{https://indico.narit.or.th/event/112/overview}{1st NARIT--EACOA Summer Workshop on Astrostatistics and Informatics}, Chiang Mai, Thailand, Aug 13--17, 2019
    \item Telluric Line Hack Week, Flatiron Institute, New York, USA, Feb 25--28, 2019
    \item \href{https://nexsci.caltech.edu/workshop/2018/}{NASA Sagan Exoplanet Summer Workshop}, NASA Exoplanet Science Institute, California Institute of Technology, Pasadena, CA, USA, Jul 23--27, 2018
    \item Japan Geoscience Union Meeting, Chiba, Japan, May 20--24, 2018
    \item 2nd Rencontres du Vietnam on Exoplanetary Science, Binh Dinh, Vietnam, Feb 2018
    \item TIARA Summer School on Astrostatistics and Big Data, Taipei, Taiwan, Sep 4--8, 2017
    \item Astronomy Hack Week, University of Washington, Seattle, WA, USA, Aug 28--Sep 1, 2017
    \item Protoplanetary Disk Dynamics and Planet Formation Workshop, Yokohama Institute for Earth Sciences (JAMSTEC), Kanagawa, Japan, Sep 29--Oct 2, 2015
    \item 27th Spring School in Particles and Fields, Chang Gung University, Taoyuan City, Taiwan, Apr 2--5, 2014
    \item NARIT--KASI Radio Astronomy Winter School, Chiang Mai, Thailand, Jan 27--30, 2014
    \item TEDx Diliman: Things That Matter, UP Diliman, Philippines, Oct 20, 2013
    \item 2nd Southeast Asian Astronomers Collaboration Meeting, Bandung, Indonesia, Nov 18--22, 2013
    \item Sokendai Asian Winter School on Exoplanets, National Astronomical Observatory of Japan (NAOJ), Mitaka, Tokyo, Japan, Dec 3--5, 2012
    \item 1st Southeast Asian Astronomers Collaboration Meeting, Palawan, Philippines, Nov 5--7, 2012
    \item Astronomy Training Workshop for Science Educators, Galileo Teacher Training Program (GTTP), UP Los Ba\~nos, Philippines, Apr 9--11, 2012
    \item The Universe Survives 2012, NIDO Science Discovery Center, SM Mall of Asia, Pasay City, Philippines, Feb 18 and 24, 2012
    \item Search for Undiscovered Main Belt Asteroids and Other Important NEO Observations, International Asteroid Campaign (IASC), Philippines, Apr 20--May 20, 2012
    \item 29th Samahang Pisika ng Pilipinas (Physics Society of the Philippines) National Physics Congress, National Institute of Physics, College of Science, UP Diliman, Philippines, Oct 24--26, 2011
    \item International Astronomical Union (IAU) Gala Night, NIDO Discovery Center, SM Mall of Asia, Pasay City, Philippines, Oct 21, 2011
    \item Developing Astronomy Education and Research in the Philippines, UP Los Ba\~nos, Philippines, Oct 20, 2011
\end{list}

\subsection{Field Work Experiences}
\begin{list}{}{\cvlist}
    \item 2022--present: Regular remote observations using the LCO Global Telescope Network
    \item 2018--present: Bi-monthly observations with the MuSCAT2/TCS telescope, Teide Observatory, Tenerife, Spain (remote)
    \item 2022--2023: Monthly remote observations with the Subaru/IRD telescope intensive program (remote; PI: Norio Narita)
    \item 2018--2020: MuSCAT2/TCS telescope, Teide Observatory, Tenerife, Spain (on-site)
    \item 2018: SIRIUS/IRSF telescope, Sutherland, South Africa (on-site, 2 weeks)
    \item 2018: IRCS/Subaru telescope, Mauna Kea Observatory, Hawaii, USA (on-site, 3 nights)
    \item 2017: CHARIS/Subaru telescope, Mauna Kea Observatory, Hawaii, USA (on-site, 2 nights)
    \item 2017: HDS/Gemini telescope, Mauna Kea Observatory, Hawaii, USA (on-site, 1 night)
    \item 2017--2019: MuSCAT2/OAO telescope, Japan (on-site, 20+ nights)
\end{list}

\subsection{Outreach Activities}
\begin{list}{}{\cvlist}
    \item Resource speaker for Junior High School Career Week, SJASQ, Philippines, Jan 24, 2026
    \item Guest on a podcast titled ``Are We Alone?'' (published on \href{https://open.spotify.com/episode/2hfoLrrP6TKyKE7mgu6pju?trackId=7BRD7x5pt8Lqa1eGYC4dzj}{Spotify}), Dec 2025
    \item Resource speaker for Philippine Astronomy Week 2024 led by PAGASA, Philippines, Feb 2024
    \item Featured by Pinoy Scientists (\href{https://www.facebook.com/pinoyscientists/posts/pfbid0Lj83pJZaaU35tdSpz5UKgkAodhBqMzudUjCS7z4UV32zqiqmkQzT7LdtpGTsu7Cql}{Jan 2023})
    \item Resource speaker for Science Week at De La Salle Integrated School, Philippines, Mar 2022
    \item Resource speaker for the Philippine Science High School--Western Visayas Campus, Philippines, Mar 2021
    \item Courtesy call with the Director of the Philippine Space Agency (PhilSA), Mar 2021
    \item Invited speaker at the National Institute of Physics, University of the Philippines Diliman (UPD), Jan 2020
    \item Astronomy Over Coffee (self-funded astronomy event), UP Diliman Hotel, Sep 2019
    \item Founder of the Filipino Astronomy Community (\href{https://www.facebook.com/groups/filastrocomm}{FAC}; 2016)
    \item Invited speaker at Philippine Rural High School, Los Ba\~nos, Philippines, Apr 2014
    \item Invited speaker at St.\ Joseph's Academy Sariaya (SJASQ) High School, Philippines, Apr 2013
\end{list}

% https://www.u-tokyo.ac.jp/focus/en/press/z0508_00003.html

% \subsection{Professional service \& activities}
% \begin{list}{}{\cvlist}
%   \item Active referee at the Monthly Notices of the Royal Astronomical Society (\href{https://academic.oup.com/mnras}{MNRAS}; Impact Factor: 4.8)
% \end{list}

\subsection{Affiliations and Services}
\begin{list}{}{\cvlist}
    \item Monthly Notices of the Royal Astronomical Society (\href{https://academic.oup.com/mnras}{MNRAS}; Impact Factor: 4.8; active referee)
    \item SPACETIDE Foundation (\href{https://spacetide.jp/en/}{spacetide.jp}; active intern, 2022--present)
    \item Science Integrated Direction for High School Investigators (\href{https://www.sidhi.org/}{SIDHI}; mentor, 2022--present)
    \item Science and Technology Advisory Council--Japan Chapter (\href{https://www.facebook.com/profile.php?id=100083271798519}{STAC-J}; member, 2019--present)
    \item Exoplanet Follow-up Observing Program (\href{https://exofop.ipac.caltech.edu/tess/}{ExoFOP}; member, 2018--present)
    \item Association of Filipino Students in Japan (\href{https://www.facebook.com/afsjpage}{AFSJ}; president, 2017--2018)
    \item Association of Filipino Scholars in Taiwan (\href{https://www.facebook.com/AssocIskolar}{AFST}; founding member, 2014)
    \item \href{https://gk1world.com/volunteer}{Gawad Kalinga} (volunteer, 2009--2013)
\end{list}

\end{document}
