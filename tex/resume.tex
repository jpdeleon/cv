\documentclass[11pt,letterpaper]{article}

\newcommand{\fullname}{Jerome Pitogo de Leon, Ph.D}
\newcommand{\currentposition}{Postdoctoral researcher at The University of Tokyo}
\newcommand{\address}{3-5-3 Nishi-koiwa Edogawa-ku Tokyo} %133-0057
\newcommand{\phonenumber}{+81 (0) 80 8712 7159}
\newcommand{\email}{jpdeleon@g.ecc.u-tokyo.ac.jp}
\newcommand{\githuburl}{https://jpdeleon.github.io}
\newcommand{\scholarurl}{https://scholar.google.com/citations?hl=en&user=_Z8ialwAAAAJ&view_op=list_works&sortby=pubdate}
\newcommand{\orcid}{0000-0002-6424-3410}
\newcommand{\orcidurl}{https://orcid.org/my-orcid?orcid=0000-0002-6424-3410}
\newcommand{\linkedinurl}{https://www.linkedin.com/in/jpdeleonbsap/}
\newcommand{\paperthree}{https://ui.adsabs.harvard.edu/abs/2023MNRAS.522..750D/abstract}
\newcommand{\papertwo}{https://ui.adsabs.harvard.edu/abs/2021MNRAS.508..195D/abstract}
\newcommand{\paperone}{https://ui.adsabs.harvard.edu/abs/2015ApJ...806L..10D/abstract}
\newcommand{\spieurl}{https://www.spiedigitallibrary.org/conference-proceedings-of-spie/10679/106790Z/Real-time-high-speed-motion-blur-compensation-method-using-galvanometer/10.1117/12.2306621.short}
\newcommand{\naritaurl}{https://www.u-tokyo.ac.jp/focus/en/people/k0001_00784.html}
\newcommand{\tamuraurl}{https://www.u-tokyo.ac.jp/focus/en/people/people003289.htm}
\newcommand{\seseurl}{https://unisec-global.org/rogel.html}
\newcommand{\takamiurl}{https://www.asiaa.sinica.edu.tw/people/cv.php?i=hiro}
\newcommand{\hayakawaurl}{https://ishikawa-vision.org/members/hayakawa/hayakawa-e.html}
\newcommand{\mayamaurl}{https://researchmap.jp/mayama_satoshi?lang=en}
\newcommand{\phdthesisurl}{https://repository.dl.itc.u-tokyo.ac.jp/record/2006678/files/A37944.pdf}
% \newcommand{}{}
\usepackage{color}
\usepackage{fancyhdr}
\usepackage{hyperref}
\usepackage{ifthen}

% \usepackage[yyyymmdd]{datetime}
% \renewcommand{\dateseparator}{-}

% Link formatting.
\definecolor{numcolor}{rgb}{0.5,0.5,0.5}
\definecolor{linkcolor}{rgb}{0,0,0.4}
\hypersetup{%
    colorlinks=true,        % false: boxed links; true: colored links
    linkcolor=linkcolor,    % color of internal links
    citecolor=linkcolor,    % color of links to bibliography
    filecolor=linkcolor,    % color of file links
    urlcolor=linkcolor      % color of external links
}

% Text formatting.
\newcommand{\foreign}[1]{\textit{#1}}
\newcommand{\etal}{\foreign{et~al.}}
\newcommand{\project}[1]{\textsl{#1}}
\definecolor{grey}{rgb}{0.5,0.5,0.5}
\newcommand{\deemph}[1]{\textcolor{grey}{\footnotesize{#1}}}

% literature links--use doi if you can
  \newcommand{\doi}[2]{\emph{\href{http://dx.doi.org/#1}{{#2}}}}
  \newcommand{\ads}[2]{\href{http://adsabs.harvard.edu/abs/#1}{{#2}}}
  \newcommand{\isbn}[1]{{\footnotesize(\textsc{isbn:}{#1})}}
  \newcommand{\arxiv}[1]{{\href{http://arxiv.org/abs/#1}{arXiv:{#1}}}}

% Section headings.
\renewcommand\familydefault{\sfdefault}
\usepackage{titlesec}

\titleformat{\subsection}
{\normalfont\sffamily\large\bfseries}
{}{0pt}{}

\titleformat{\subsubsection}
{\normalfont\sffamily\bfseries}
{}{0pt}{}

\titlespacing{\subsection}{0pt}{2\parskip}{0pt}
\titlespacing{\subsubsection}{0pt}{\parskip}{0pt}

\newcommand{\cvheading}[1]{\addvspace{1ex}\pagebreak[2]\par\textbf{#1}\nopagebreak\vspace{-0.4em}}

% Set up the custom unordered list.
\newcounter{refpubnum}
\newcommand{\cvlist}{%
    \rightmargin=0in
    \leftmargin=0.15in
    \topsep=0ex
    \partopsep=0pt
    \itemsep=0.2ex
    \parsep=0pt
    \itemindent=-1.0\leftmargin
    \listparindent=0.0\leftmargin
    \settowidth{\labelsep}{~}
    \usecounter{refpubnum}
}

% Margins and spaces.
\raggedright
\setlength{\oddsidemargin}{0in}
\setlength{\topmargin}{0in}
\setlength{\headsep}{0.20in}
\setlength{\headheight}{0.25in}
\setlength{\textheight}{9.1in}
\addtolength{\topmargin}{-\headsep}
\addtolength{\topmargin}{-\headheight}
\setlength{\textwidth}{6.50in}
\setlength{\parindent}{0in}
\setlength{\parskip}{1ex}

% Headings and footings.
\renewcommand{\headrulewidth}{0pt}
\pagestyle{fancy}
\lhead{\deemph{Jerome Pitogo de Leon}}
\chead{\deemph{Curriculum Vitae}}
\rhead{\deemph{\thepage}}
\cfoot{\deemph{Last updated: \today}}

% Journal names.
\newcommand{\aj}{AJ}
\newcommand{\apj}{ApJ}
\newcommand{\pasp}{PASP}
\newcommand{\mnras}{MNRAS}

\newcommand{\pubsdate}{2024-12-12}
\newcommand{\pubsfirst}{3}
\newcommand{\pubsnumber}{75}
\newcommand{\pubscitations}{1,468}
\newcommand{\pubshindex}{24}

\usepackage{fontawesome5}
\newcommand{\makefield}[2]{\makebox[1.5em]{#1} #2}

\addtolength{\topmargin}{-0.2in}
\addtolength{\textheight}{0.4in}
\addtolength{\oddsidemargin}{-0.2in}
\addtolength{\evensidemargin}{-0.2in}
\addtolength{\textwidth}{0.4in}

\begin{document}\thispagestyle{empty}\sloppy\sloppypar\raggedbottom

\textbf{\Large \fullname} \\[0.5ex]
\currentposition \\[1.0ex]
\textsf{\small 
    \makefield{\faMapPin}{\address} %
    \makefield{\faEnvelope[regular]}{\href{mailto:\email}{\texttt{jpdeleon[at]g.ecc.u-tokyo.ac.jp}}}\\%
    \makefield{\faPhone}{\url{\phonenumber}} %
    \makefield{\faBook}{\href{\scholarurl}}{Google scholar} %
    % \href{\orcidurl}{ORCID: \orcid} | %
    \makefield{\faLinkedin}{\href{\linkedinurl}{\texttt{linkedin.com/in/jpdeleonbsap}}}\\%
    \makefield{\faGlobe}{\url{\githuburl}} %
}\\[0.5ex]


\subsection{Overview}
\begin{list}{}{\cvlist}
    \item \textbf{Research:} Published \pubsnumber\ refereed papers mainly about exoplanets with \pubscitations\ total citations, and an h-index of \pubshindex, as of \pubsdate\footnote{Publication data collected from \href{https://ui.adsabs.harvard.edu/}{NASA Astrophysics Data System}}. The complete list is available on \href{\scholarurl}{Google Scholar}.
    \item \textbf{Teaching:} Conducted 2 introductory courses in Astronomy for undergraduates; half of the class pursued graduate studies both in the Philippines and abroad.
    \item \textbf{Mentorship:} Guided a dynamic research group consisting of 1 PhD, 3 master, and 2 undergraduate students in cutting-edge exoplanet research initiatives. Developed publicly available code for Astronomy research and instruction.
    \item \textbf{Outreach:} Presented several invited talks over the years, contributing to both research and public outreach. Founded Astronomy group for fostering a passion for the cosmos.
\end{list}

\subsection{Education}
\begin{list}{}{\cvlist}
    \item
        \textbf{PhD, 2021} Department of Astronomy, The University of Tokyo, Japan.\\
        Title: Discovery and Characterization of Transiting Exoplanets with Diverse Radii and Ages (\href{\phdthesisurl}{online version}).
        Advisor: \href{\tamuraurl}{Prof. Motohide Tamura}
    \item
        \textbf{MSc, 2018} Department of Astronomy, The University of Tokyo, Japan.\\
        Title: Multi-color Simultaneous Transit Observations of Low Density Hot Jupiters (\href{https://github.com/jpdeleon/thesis-master/tree/master}{online version}).
        Advisor: Prof. Motohide Tamura
    \item
        \textbf{BSc, 2013}\\
        Department of Physics, University of the Philippines, Los Ba\~nos, Laguna, Philippines.\\
        Title: Thermal Differential and Image Analysis of a Fabricated 11-cm Solar Telescope using an Active-Passive Cooling System.
        Advisor: \href{\seseurl}{Dr. Rogel Mari Sese}      
\end{list}

\subsection{Full-time Positions}
\begin{list}{}{\cvlist}
    \item
        \textbf{Project Assistant Professor (2025/4 - Present)} \\
        \textit{Komaba Institute for Science, Graduate School of Arts and Sciences, The University of Tokyo, Japan} \\
        \textit{Role: Management of observations and data analysis with \href{\muscatLCOurl}{MuSCAT4} under the \href{\kibanSurl}{Kiban S} project.}\\
        
    \item
        \textbf{Project Research Assistant (2023/4 - 2025/3)} \\
        \textit{Department of Multi-Disciplinary Sciences, Graduate School of Arts and Sciences, The University of Tokyo, Japan} \\
        \textit{Role: Spearheading the discovery and in-depth characterization of young transiting exoplanets using the TESS telescope.}
        \textit{Advisor: \href{\naritaurl}{Prof. Norio Narita}} (Refer to \href{\paperthree}{de Leon et al. 2023})\\

    \item
        \textbf{Project Research Assistant (2021/4 - 2023/3)} \\
        \textit{Department of Astronomy, Graduate School of Science, The University of Tokyo, Japan} \\
        \textit{Role: Led an international group of astronomers towards the discovery of 37 new transiting exoplanets using the Kepler telescope.}
        \textit{Advisor: \href{tamuraurl}{Prof. Motohide Tamura}} (Refer to \href{\papertwo}{de Leon et al. 2021})\\
        
    \item
        \textbf{Research Development Assistant (2013/4-2014/6)}\\
        \textit{Regulus SpaceTech, Inc., Los Ba\~nos, Laguna, Philippines}\\
        \textit{Role: Initiated the software development for automating solar telescope control and data analysis.}
        \textit{Advisor: \href{\seseurl}{Dr. Rogel Sese}} \\
\end{list}

\subsection{Temporary Positions}
\begin{list}{}{\cvlist}
    \item
        \textbf{Remote Lecturer (2021/6 - 2021/10)} \\
        \textit{Faculty of Aerospace Engineering, School of Engineering \& Architecture, Ateneo de Davao University, Philippines} \\
        \textit{Role: Crafted and conducted lectures on Introduction to Astronomy \& Astronomical Data Analysis.}
        
    \item 
        \textbf{Technical Assistant (2018/10-2019/3)}\\
        \textit{Engineering Department, The University of Tokyo, Japan}\\
        \textit{Role: Developed innovative image processing software for stitching featureless high-resolution images 
        \textit{Advisor: \href{hayakawaurl}{Assoc. Prof. Tomohiko Hayakawa}} (Refer to \href{\spieurl}{Murakami et al. 2018})}

    \item 
        \textbf{Research Assistant (2017/4-2018/3)}\\	
        \textit{Department of Astronomy, The University of Tokyo, Japan}\\
        \textit{Role: Developed a robust pipeline for modeling and analyzing multi-wavelength transit light curves and provided mentorship to new research group members.}

    \item 
        \textbf{Research Student (2015/9-2016/3)}\\
        \textit{Department of Astronomy, Graduate University for Advanced Studies (SOKENDAI), Japan}\\
        \textit{Role: Worked on data reduction and analysis of high-resolution astronomical images and contributed to scientific report writing.}
        \textit{Advisor: \href{\mayamaurl}{Dr. Satoshi Mayama}} \\

    \item 
        \textbf{Intern (2014/7-2014/9)}\\
        \textit{Academia Sinica Institute for Astronomy and Astrophysics (ASIAA), Taipei, Taiwan}\\	
        \textit{Role: Developed an image processing pipeline for the analysis of high-resolution images of a young forming star using the 8-m Subaru telescope.}
        \textit{Advisor: \href{takamiurl}{Dr. Michihiro Takami} (Refer to \href{\paperone}{de Leon et al. 2015})} \\
\end{list}


\subsection{Education}
\begin{list}{}{\cvlist}
    \item
        \textbf{PhD, 2021} Department of Astronomy, The University of Tokyo, Japan.\\
        Title: Discovery and Characterization of Transiting Exoplanets with Diverse Radii and Ages (\href{\phdthesisurl}{online version}).
        Advisor: \href{\tamuraurl}{Prof. Motohide Tamura}
    \item
        \textbf{MSc, 2018} Department of Astronomy, The University of Tokyo, Japan.\\
        Title: Multi-color Simultaneous Transit Observations of Low Density Hot Jupiters (\href{https://github.com/jpdeleon/thesis-master/tree/master}{online version}).
        Advisor: Prof. Motohide Tamura
    \item
        \textbf{BSc, 2013}\\
        Department of Physics, University of the Philippines, Los Ba\~nos, Laguna, Philippines.\\
        Title: Thermal Differential and Image Analysis of a Fabricated 11-cm Solar Telescope using an Active-Passive Cooling System.
        Advisor: \href{\seseurl}{Dr. Rogel Mari Sese}      
\end{list}

\subsection{Awards}
\begin{list}{}{\cvlist}
    \item \textbf{2022/12:} St. Joseph's Academy, Centennial Achiever's Award
    \item \textbf{2021/3:} Philippine Embassy in Japan, Medal for distinguished Filipino Graduates
    \item \textbf{2018/4-2021/3:} Ministry of Education, Culture, Sports (MEXT) PhD Scholarship (JPY 8.4M)
    \item \textbf{2016/4-2018/3:} MEXT MSc Scholarship (JPY 5.6M)
    \item \textbf{2015/4-2016/3:} MEXT Japanese Language and Research Scholarship (JPY 2.8M)
    \item \textbf{2015/10-2016/3:} Research Grant from the Graduate University for Advanced Studies (JPY 0.2M)
    \item \textbf{2014/9:} SanDisk Japan Scholarship (USD 3.75k, deferred)
    \item \textbf{2013/3:} International Space University Scholarship (EUR 9k, deferred)
    \item \textbf{2011-Present:} Received multiple travel grants to attend workshops in the US and Taiwan related to data science, and participated in research-related conferences and observation runs in the U.S., Europe, South Africa, and South-east Asia
\end{list}

\subsection{Teaching and Mentorships}
\begin{list}{}{\cvlist}
    \item \textbf{Designed and Taught Courses:} Created and delivered two elective courses for 4th-year undergraduate Aerospace Engineering students at the \href{http://sea.addu.edu.ph/programs/aerospace-engineering/}{Ateneo de Davao University} during the 2nd semester of 2021.
    \item \textbf{Research Collaborations:} Collaborated with M. Mori (Current postdoc) on the discovery and validation of a rare planet in the Neptune desert (Refer to \href{https://ui.adsabs.harvard.edu/abs/2022AJ....163..298M/abstract}{Mori et al. 2022}).%; \href{https://github.com/jpdeleon/toi1696}{github.com/jpdeleon/toi1696}).
    \item \textbf{Pipeline Development:} Wrote a public Python code for easily analyzing data from TESS telescope appropriate for research and instruction (Refer to \href{https://github.com/jpdeleon/quicklook}{github.com/jpdeleon/quicklook})
    \item \textbf{Thesis Advisory:} Worked with H. Kobayashi (Former undergraduate student) to develop a pipeline for finding young transiting planets and binary stars in nearby star clusters (Refer to \href{https://github.com/hiremasa/ytps}{github.com/hiremasa/ytps}). Also advised A. Nodado (Former student) on her undergraduate thesis focused on Mira variables.
    \item \textbf{MEXT Scholarship Guidance:} Provided guidance to A. Javier and C. Cabatlao from their scholarship application to acceptance, supporting their successful pursuit of graduate studies in Japan.
\end{list}

\subsection{Trainings and Workshops}
\begin{list}{}{\cvlist}
    \item JWST Data Analysis Workshop, Chiang Mai, Thailand, Jun 24 - Jul 5 2024
    \item Japan Planetary Atmosphere Research Society workshop II, Atami, Japan, Mar 2023
    \item Japan Planetary Atmosphere Research Society workshop, Matsushima, Miyagi, Japan, Mar 2022
    \item RESCEU Symposium, From Protoplanetary Disks through Planetary System Architecture to Planetary Atmospheres and Habitability, Okinawa, Japan, Oct 14-18, 2019
    \item 1st NARIT-EACOA Summer workshop on Astrostatistics and Informatics, Chiang Mai, Thailand, Aug 13-17, 2019
    \item Telluric Line Hack Week, Flatiron Institute, New York, USA, Feb 25-28, 2019
    \item NASA Sagan Exoplanet Summer Workshop, NASA Exoplanet Science Institute, California Institute of Technology, Pasadena, California, USA, Jul 23-27, 2018
    \item Japan Geoscience Union Meeting, Chiba, Japan, May 20-24, 2018
    \item 2nd Rencontres du Vietnam on Exoplanetary Science, Binh Dinh, Vietnam, Feb 2018
    \item TIARA Summer School on Astrostatistics and Big Data, Taipei, Taiwan, Sep 4-8, 2017
    \item Astronomy Hack Week, University of Washington, Seattle, USA, Aug 28- Sep 1, 2017
    \item Protoplanetary Disk Dynamics and Planet Formation Workshop, Yokohama Institute for Earth Sciences, JAMSTEC, Kanagawa, Japan, Sep 29-Oct 2, 2015
    \item 27th Spring School in Particle and Fields, Chang Gung University, Taoyuan City, Taiwan, Apr 2-5, 2014
    \item NARIT-KASI Radio Astronomy Winter School, Chiang Mai, Thailand, Jan 27-30, 2014
    \item TEDx Diliman: Things that Matter, UP Diliman, Philippines, Oct 20, 2013
    \item 2nd Southeast Asian Astronomers Collaboration Meeting, Bandung, Indonesia, Nov 18-22, 2013
    \item Sokendai Asian Winter School on Exoplanets, National Astronomical Observatory of Japan (NAOJ), Mitaka, Tokyo, Japan, Dec 3-5, 2012
    \item 1st Southeast Asian Astronomers Collaboration Meeting, Palawan, Philippines, Nov 5-7, 2012
    \item Astronomy Training Workshop for Science Educators, Galileo Teacher Training Program (GTTP), UP Los Baños, Philippines, Apr 9-11, 2012
    \item The Universe Survives– 2012, NIDO Science Discovery Center, SM Mall of Asia, Pasay City, Philippines, Feb 18 and 24, 2012
    \item Search for Undiscovered Main Belt asteroids and other important NEO Observations International Asteroid Campaign (IASC), Philippines, Apr 20—May 20, 2012
    \item 29th Samahang Pisika ng Pilipinas (Physics Society of the Philippines) National Physics Congress, National Institute of Physics, College of Science, UP Diliman, Philippines, Oct 24-26, 2011
    \item International Astronomical Union (IAU) Gala Night, NIDO Discovery Center, SM Mall of Asia, Pasay City, Philippines, Oct 21, 2011
    \item Developing Astronomy Education and Research in the Philippines,‖ UP Los Baños, Philippines, Oct 20, 2011
\end{list}

\subsection{Field Work Experiences}
\begin{list}{}{\cvlist}
    \item 2018-present: bi-monthly observations with MuSCAT2/TCS telescope, Teide Observatory, Tenerife, Spain (remote)
    \item 2022-2023: monthly remote observations with Subaru/IRD telescope intensive program (remote; PI: Norio Narita)
    \item 2018-2020: MuSCAT2/TCS telescope, Teide Observatory, Tenerife, Spain (on-site) 
    \item 2018: SIRIUS/IRSF telescope, Sutherland, South Africa (on-site, 2 weeks)
    \item 2018: IRCS/Subaru telescope, Mauna Kea observatory, Hawaii, USA (on-site, 3 nights)
    \item 2017: CHARIS/Subaru telescope, Mauna Kea observatory, Hawaii, USA (on-site, 2 nights)
    \item 2017: HDS/Gemini telescope, Mauna Kea observatory, Hawaii, USA (on-site, 1 night)
    \item 2017-2019: MuSCAT2/OAO telescope, Japan (on-site, 20+ nights)
\end{list}

\subsection{Outreach Activities}
\begin{list}{}{\cvlist}
    \item Resource speaker for the Philippine Astronomy Week 2024 led by PAGASA, Feb 2024
    \item Featured at Pinoy Scientists in \href{https://www.facebook.com/pinoyscientists/posts/pfbid0Lj83pJZaaU35tdSpz5UKgkAodhBqMzudUjCS7z4UV32zqiqmkQzT7LdtpGTsu7Cql}{Jan 2023}
    \item Resource speaker for Science Week at De La Salle Integrated School, Mar 2022
    \item Resource speaker for the Philippine Science High School-Western Visayas Campus, Mar 2021
    \item Courtesy call by the Director of the Philippine Space Agency (PhilSA), Mar 2021
    \item Invited speaker at National Insitute of Physics, University of the Philippines Diliman (UPD), Jan 2020
    \item Astronomy Over Coffee (self-funded Astronomy event) at UP Diliman Hotel, Sep 2019
    \item Founding of Filipino Astronomy Community (\href{https://www.facebook.com/groups/filastrocomm}{FAC}; 2016)
    \item Invited speaker at Philippine Rural High School, Los Banos, Apr 2014
    \item Invited speaker at St. Joseph's Academy Sariaya High School, Apr 2013
\end{list}

\subsection{Affiliations and Services}
\begin{list}{}{\cvlist}
    \item Monthly Notices of the Royal Astronomical Society (\href{https://academic.oup.com/mnras}{MNRAS}; Impact Factor: 4.8; Active referee)
    \item SPACETIDE Foundation (\href{https://spacetide.jp/en/}{spacetide.org}; Active intern, 2022-present)
    \item Science Integrated Direction for High School Investigators (\href{https://www.sidhi.org/}{SIDHI}; mentor, 2022-present)
    \item Science and Technology Advisory Council- Japan chapter (\href{https://www.facebook.com/profile.php?id=100083271798519}{STAC-J}; Member, 2019-present)
    \item Exoplanet Follow-up Observing Program (\href{https://exofop.ipac.caltech.edu/tess/}{ExoFOP}; Member, 2018-present)
    \item Association of Filipino Students in Japan (\href{https://www.facebook.com/afsjpage}{AFSJ}; President, 2017-2018)
    \item Association of Filipino Scholars in Taiwan (\href{https://www.facebook.com/AssocIskolar}{AFST}; Founding member, 2014)
    \item \href{https://gk1world.com/volunteer}{Gawad Kalinga} (Volunteer, 2009-2013)
\end{list}

% \subsection{Skills}
% \begin{list}{}{\cvlist}
%     \item Languages: Filipino (native), English, Japanese
%     \item Scientific software development
% \end{list}

\end{document}
