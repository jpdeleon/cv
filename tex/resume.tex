\documentclass[11pt,letterpaper]{article}

\newcommand{\fullname}{Jerome Pitogo de Leon, Ph.D}
\newcommand{\currentposition}{Postdoctoral researcher at The University of Tokyo}
\newcommand{\address}{3-5-3 Nishi-koiwa Edogawa-ku Tokyo} %133-0057
\newcommand{\phonenumber}{+81 (0) 80 8712 7159}
\newcommand{\email}{jpdeleon@g.ecc.u-tokyo.ac.jp}
\newcommand{\githuburl}{https://jpdeleon.github.io}
\newcommand{\scholarurl}{https://scholar.google.com/citations?hl=en&user=_Z8ialwAAAAJ&view_op=list_works&sortby=pubdate}
\newcommand{\orcid}{0000-0002-6424-3410}
\newcommand{\orcidurl}{https://orcid.org/my-orcid?orcid=0000-0002-6424-3410}
\newcommand{\linkedinurl}{https://www.linkedin.com/in/jpdeleonbsap/}
\newcommand{\paperthree}{https://ui.adsabs.harvard.edu/abs/2023MNRAS.522..750D/abstract}
\newcommand{\papertwo}{https://ui.adsabs.harvard.edu/abs/2021MNRAS.508..195D/abstract}
\newcommand{\paperone}{https://ui.adsabs.harvard.edu/abs/2015ApJ...806L..10D/abstract}
\newcommand{\spieurl}{https://www.spiedigitallibrary.org/conference-proceedings-of-spie/10679/106790Z/Real-time-high-speed-motion-blur-compensation-method-using-galvanometer/10.1117/12.2306621.short}
\newcommand{\naritaurl}{https://www.u-tokyo.ac.jp/focus/en/people/k0001_00784.html}
\newcommand{\tamuraurl}{https://www.u-tokyo.ac.jp/focus/en/people/people003289.htm}
\newcommand{\seseurl}{https://unisec-global.org/rogel.html}
\newcommand{\takamiurl}{https://www.asiaa.sinica.edu.tw/people/cv.php?i=hiro}
\newcommand{\hayakawaurl}{https://ishikawa-vision.org/members/hayakawa/hayakawa-e.html}
\newcommand{\mayamaurl}{https://researchmap.jp/mayama_satoshi?lang=en}
\newcommand{\phdthesisurl}{https://repository.dl.itc.u-tokyo.ac.jp/record/2006678/files/A37944.pdf}
% \newcommand{}{}
\usepackage{color}
\usepackage{fancyhdr}
\usepackage{hyperref}
\usepackage{ifthen}

% \usepackage[yyyymmdd]{datetime}
% \renewcommand{\dateseparator}{-}

% Link formatting.
\definecolor{numcolor}{rgb}{0.5,0.5,0.5}
\definecolor{linkcolor}{rgb}{0,0,0.4}
\hypersetup{%
    colorlinks=true,        % false: boxed links; true: colored links
    linkcolor=linkcolor,    % color of internal links
    citecolor=linkcolor,    % color of links to bibliography
    filecolor=linkcolor,    % color of file links
    urlcolor=linkcolor      % color of external links
}

% Text formatting.
\newcommand{\foreign}[1]{\textit{#1}}
\newcommand{\etal}{\foreign{et~al.}}
\newcommand{\project}[1]{\textsl{#1}}
\definecolor{grey}{rgb}{0.5,0.5,0.5}
\newcommand{\deemph}[1]{\textcolor{grey}{\footnotesize{#1}}}

% literature links--use doi if you can
  \newcommand{\doi}[2]{\emph{\href{http://dx.doi.org/#1}{{#2}}}}
  \newcommand{\ads}[2]{\href{http://adsabs.harvard.edu/abs/#1}{{#2}}}
  \newcommand{\isbn}[1]{{\footnotesize(\textsc{isbn:}{#1})}}
  \newcommand{\arxiv}[1]{{\href{http://arxiv.org/abs/#1}{arXiv:{#1}}}}

% Section headings.
\renewcommand\familydefault{\sfdefault}
\usepackage{titlesec}

\titleformat{\subsection}
{\normalfont\sffamily\large\bfseries}
{}{0pt}{}

\titleformat{\subsubsection}
{\normalfont\sffamily\bfseries}
{}{0pt}{}

\titlespacing{\subsection}{0pt}{2\parskip}{0pt}
\titlespacing{\subsubsection}{0pt}{\parskip}{0pt}

\newcommand{\cvheading}[1]{\addvspace{1ex}\pagebreak[2]\par\textbf{#1}\nopagebreak\vspace{-0.4em}}

% Set up the custom unordered list.
\newcounter{refpubnum}
\newcommand{\cvlist}{%
    \rightmargin=0in
    \leftmargin=0.15in
    \topsep=0ex
    \partopsep=0pt
    \itemsep=0.2ex
    \parsep=0pt
    \itemindent=-1.0\leftmargin
    \listparindent=0.0\leftmargin
    \settowidth{\labelsep}{~}
    \usecounter{refpubnum}
}

% Margins and spaces.
\raggedright
\setlength{\oddsidemargin}{0in}
\setlength{\topmargin}{0in}
\setlength{\headsep}{0.20in}
\setlength{\headheight}{0.25in}
\setlength{\textheight}{9.1in}
\addtolength{\topmargin}{-\headsep}
\addtolength{\topmargin}{-\headheight}
\setlength{\textwidth}{6.50in}
\setlength{\parindent}{0in}
\setlength{\parskip}{1ex}

% Headings and footings.
\renewcommand{\headrulewidth}{0pt}
\pagestyle{fancy}
\lhead{\deemph{Jerome Pitogo de Leon}}
\chead{\deemph{Curriculum Vitae}}
\rhead{\deemph{\thepage}}
\cfoot{\deemph{Last updated: \today}}

% Journal names.
\newcommand{\aj}{AJ}
\newcommand{\apj}{ApJ}
\newcommand{\pasp}{PASP}
\newcommand{\mnras}{MNRAS}

\newcommand{\pubsdate}{2024-12-12}
\newcommand{\pubsfirst}{3}
\newcommand{\pubsnumber}{75}
\newcommand{\pubscitations}{1,468}
\newcommand{\pubshindex}{24}

\addtolength{\topmargin}{-0.2in}
\addtolength{\textheight}{0.4in}
\addtolength{\oddsidemargin}{-0.2in}
\addtolength{\evensidemargin}{-0.2in}
\addtolength{\textwidth}{0.4in}

\begin{document}\thispagestyle{empty}\sloppy\sloppypar\raggedbottom

\textbf{\Large \fullname} \\[0.5ex]
\currentposition \\
\textsf{\small 
    \href{mailto:\email}{jpdeleon [at] g.ecc.u-tokyo.ac.jp} | %
    Phone: {\phonenumber} | %
    % \href{\scholarurl}{google scholar} | %
    \href{\orcidurl}{ORCID: \orcid} | %
    \href{\githuburl}{github.com/jpdeleon} | %
    % \href{\scholarurl}{scholar.google.com} | %
    \href{\linkedinurl}{linkedin.com/in/jpdeleonbsap}
}\\[0.5ex]

\subsection{Overview}
\begin{list}{}{\cvlist}
    \item As of \pubsdate\footnote{Publication data collected from \href{https://ui.adsabs.harvard.edu/}{NASA Astrophysics Data System}}, published \pubsnumber\ refereed papers with \pubscitations\ total citations, and an h-index of \pubshindex; a complete listing can be found on \href{\scholarurl}{Google Scholar}.
    \item Taught 2 introductory courses in Astronomy for undergraduates. 
    \item Led a research group composed of 1 PhD, 3 master and 2 undergraduate students. 
    \item Gave several invited colloquium talks for research and public outreach.
\end{list}

\subsection{Work Experience}
\begin{list}{}{\cvlist}
    \item
        Project Research Assistant (2023/4 - current) \\
        Advisor: Asst prof. Norio Narita \\
        Department of Multi-Disciplinary Sciences, Graduate School of Arts and Sciences, The University of Tokyo, 3-8-1 Komaba, Meguro, Tokyo 153-8902, Japan \\
        Theme: Discovery and characterization of young transiting exoplanets \\
    \item
        Remote Lecturer (2021/6 - 2021/10) \\
        Faculty of Aerospace Engineering, School of Engineering \& Architecture, Ateneo de Davao University, 8016 Davao City, Philippines \\
        Subjects: Introduction to Astronomy \& Astronomical Data Analysis \\
    \item
        Project Research Assistant (2021/4 - 2023/3) \\
        Department of Astronomy, Graduate School of Science, The University of Tokyo, 7-3-1 Hongo, Bunkyo-ku, Tokyo 113-0033, Japan \\
        Advisor: Prof. Motohide Tamura \\
        Theme: Discovery and characterization of transiting exoplanets \\
    \item 
        Technical Assistant (2018/10-2019/3)\\
        Engineering Department, The University of Tokyo, Japan\\
        Developed image processing software for high resolution mosaic/stitching and analysis (See \href{\spieurl}{Murakami et al. 2018})
    \item 
        Research Assistant (2017/4-2018/3)\\	
        Astronomy Department, The University of Tokyo, Japan\\
        Developed pipeline for modeling and analysis of multi-wavelength transit light curves used in our research group and mentored new members
    \item 
        Research Student (2015/9-2016/3)\\
        Department of Astronomy, Graduate University for Advanced Studies (aka SOKENDAI), Japan\\
        Image processing and data reduction of high resolution astronomical images and scientific report writing
    \item 
        Intern (2014/7-2014/9)\\
        Academia Sinica Institute for Astronomy and Astrophysics, Taipei, Taiwan\\	
        Developed image processing pipeline for analysis of high resolution astronomical images (See \href{\paperone}{de Leon et al. 2015})
    \item 
        Research Development Assistant (2013/4-2014/6)\\
        Regulus SpaceTech, Inc., Los Ba\~nos, Laguna, Philippines\\
        Theme: Data pipeline development for basic operation and data analysis of automated solar telescope
\end{list}

\subsection{Education}
\begin{list}{}{\cvlist}
    \item
        PhD 2021, Department of Astronomy, The University of Tokyo, Japan.\\Advisor: \href{\tamuraurl}{Prof. Motohide Tamura}
    \item
        MSc 2018, Department of Astronomy, The University of Tokyo, Japan.\\Advisor: Prof. Motohide Tamura
    \item
        BSc 2013, Department of Physics, University of the Philippines, Los Ba\~nos, Laguna, Philippines.\\Advisor: \href{\seseurl}{Dr. Rogel Mari Sese}      
\end{list}

\subsection{Scholarships}
\begin{list}{}{\cvlist}
    \item 2015/4-2021/3: Ministry of Education, Culture, Sports (MEXT) Scholarship (JPY 40.4M)
    \item 2014/9: SanDisk Japan Scholarship (USD 3.75k, deferred)
    \item 2013/3: International Space University Scholarship (EUR 9k, deferred)
    \item 2011-now: now multiple travel grants to attend workshops in the US and Taiwan related to data science and research-related conferences and observation runs in the U.S., Europe, and South-east Asia
\end{list}

\subsection{Teaching and Mentorships}
\begin{list}{}{\cvlist}
    \item I designed, wrote, and taught 2 elective courses to 4th year undergraduate Aerospace Engineering students at the \href{http://sea.addu.edu.ph/programs/aerospace-engineering/}{Ateneo de Davao University} in 2nd semester of 2021. \\
    \item M. Mori (Current postdoc) and I worked on the discovery and validation of a rare planet found in the so-called Neptune desert (See \href{https://ui.adsabs.harvard.edu/abs/2022AJ....163..298M/abstract}{Mori et al. 2022}; \href{https://github.com/jpdeleon/toi1696}{github.com/jpdeleon/toi1696})
    \item H. Kobayashi (Former undergraduate student) and I worked on creating a pipeline for finding young transiting planets in nearby star clusters and found no planets but several young binary stars (See \href{https://github.com/hiremasa/ytps}{github.com/hiremasa/ytps})
    \item I adviced A. Nodado (Former student) on her undergraduate thesis on Mira variables
    \item I adviced A. Javier and C. Cabatlao for 1 year from application to acceptance of MEXT scholarship for their graduate studies in Japan
\end{list}


\subsection{Affiliations}
\begin{list}{}{\cvlist}
    \item Association of Filipino Students in Japan (\href{https://www.facebook.com/afsjpage/}{AFSJ}; President, 2017-2018)
    \item Association of Filipino Scholars in Taiwan  (\href{https://www.facebook.com/AssocIskolar/}{AFST}; Founding member)
    \item Science and Technology Advisory Council- Japan chapter (\href{https://www.facebook.com/profile.php?id=100083271798519}{STAC-J}; Member)
\end{list}

\end{document}
